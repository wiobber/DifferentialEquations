\documentclass{ximera}
\input{../preamble.tex}

\title{Exercises} \license{CC BY-NC-SA 4.0}

\begin{document}

\begin{abstract}
\end{abstract}
\maketitle

\begin{onlineOnly}
\section*{Exercises}
\end{onlineOnly}
\begin{problem}\label{exer:4.1.1}
The half-life of a radioactive substance is 3200
years.  Find the quantity $Q(t)$ of the substance left at
time $t > 0$ if $Q(0)=20$ g.
\end{problem}

\begin{problem}\label{exer:4.1.2}
The half-life of a radioactive substance is 2 days.  Find
the time required for a given amount of the material to
decay to 1/10 of its original mass.
\begin{solution}
    $k\tau=\ln2$ and $\tau=2\Rightarrow k=\frac{\ln2}{2}$;
$Q(t)=Q_0e^{-t\ln2/2}$; if $Q(T)=\frac{Q_0}{10}$, then
$\frac{Q_0}{10}=Q_0e^{-T\ln2/2}$; $\ln10=\frac{T\ln2}{2}$;
$T=\frac{2\ln10}{\ln2}$ days.
\end{solution}
\end{problem}

\begin{problem}\label{exer:4.1.3}
A radioactive material loses 25\% of its mass
in 10 minutes.  What is its half-life?
\end{problem}

\begin{problem}\label{exer:4.1.4}
A  tree contains a known percentage $p_0$ of
a radioactive substance with half-life $\tau$.  When the
tree dies the substance decays and isn't  replaced.  If the
percentage of the substance in the fossilized remains of
such a tree is found to be $p_1$, how long has the tree been
dead?
\begin{solution}
    Let $t_1$ be the elapsed time since the tree died.
Since $p(t)=e^{-(t\ln2)\tau}$, it follows that
$p_1=p_0e^{-(t_1\ln2)/\tau}$, so $\ln\left(\frac{p_1}{p_0}\right)
=-\frac{t_1}{\tau}\ln2$ and
 $t_1=\tau \frac{\ln(p_0/p_1)}{\ln2}$.
\end{solution}
\end{problem}

\begin{problem}\label{exer:4.1.5}
If $t_p$ and $t_q$ are the times required for a radioactive
material to decay to $1/p$ and $1/q$  times
its original mass (respectively), how are $t_p$ and $t_q$ related?
\end{problem}

\begin{problem}\label{exer:4.1.6}
Find the decay constant $k$ for a radioactive substance,
given that the mass of the substance is $Q_1$ at time $t_1$
and $Q_2$ at time $t_2$.
\begin{solution}
$Q=Q_0e^{-kt}$;
$Q_1=Q_0e^{-kt_1}$;
$Q_2=Q_0e^{-kt_2}$; $\frac{Q_2}{Q_1}=e^{-k(t_2-t_1)}$;
    $\ln\left(\frac{Q_1}{Q_2}\right)=k(t_2-t_1)$; $k=\frac{1}{t_2-t_1}\ln\left(\frac{Q_1}{Q_2}\right)$.
\end{solution}
\end{problem}

\begin{problem}\label{exer:4.1.7}
A process creates a radioactive substance at the rate of 2
g/hr and the substance decays at a rate proportional to
its mass, with constant of proportionality
$k=.1 \text{  hr}^{-1}$.  If $Q(t)$ is the mass of the
substance at time $t$, find $\lim_{t\to\infty}Q(t)$\end{problem}

\begin{problem}\label{exer:4.1.8}
A bank pays interest continuously at the rate of 6\%.
How long does it take for a deposit of $Q_0$ to grow in
value to $2Q_0$?
\begin{solution}
$Q'=.06Q,\ Q(0)=Q_0$;\ $Q=Q_0e^{.06t}$. We must find $\tau$
such that $Q(\tau)=2Q_0$; that is, $Q_0e^{.06\tau}=2Q_0$, so
$.06\tau=\ln2$ and $\tau=\frac{\ln2}{.06} =\frac{50\ln2}{3}$ yr.
\end{solution}
\end{problem}



\begin{problem}\label{exer:4.1.9}
At what rate of interest, compounded continuously, will a
bank deposit double in value in 8 years?
\end{problem}

\begin{problem}\label{exer:4.1.10}
A savings account pays 5\% per annum interest compounded
continuously. The initial deposit is $Q_0$ dollars. Assume that there
are no subsequent withdrawals or deposits.

\begin{enumerate}

\item % (a)
How long will it take for the value of the account to triple?
\begin{solution}
If $T$ is the time to triple the value, then
$Q(T)=Q_0e^{.05T}=3Q_0$, so $e^{.05T}=3$.
Therefore, $.05T=\ln3$ and $T=20\ln3$.
\end{solution}

\item % (b)
What is $Q_0$ if the value of the account after 10 years is \$100,000
dollars?

\begin{solution}
If $Q(10)=100000$, then  $Q_0e^{.5}=100000$,
so $Q_0=100000e^{-.5}$
\end{solution}
\end{enumerate}
\end{problem}

\begin{problem}\label{exer:4.1.11}
A candymaker makes 500 pounds of candy per week, while
his large family eats the candy at a rate equal to $Q(t)/10$
pounds per week, where $Q(t)$ is the amount of candy present
at time $t$.
\begin{enumerate}
\item %(a)
Find $Q(t)$ for $t > 0$ if the candymaker has 250 pounds of
candy at $t=0$.
\item %(b)
Find $\lim_{t\to\infty} Q(t)$.
\end{enumerate}
\end{problem}

\begin{problem}\label{exer:4.1.12}
Suppose a substance decays at a yearly rate equal
to half the square of the mass of the substance present.  If
we start with 50 g of the substance, how long will it be
until only 25 g remain?
\begin{solution}
$Q'=-\frac{Q^2}{2},\ Q(0)=50$;\, $\frac{Q'}{Q^2}=-\frac{1}{2}$;\,$-\frac{1}{Q}=-\frac{t}{2}+c$;\,
$Q(0)=50\Rightarrow c=-\frac{1}{50}$;\, $\frac{1}{Q}=\frac{t}{2}+\frac{1}{50}=\frac{1+25t}{50}$;\,
$Q=\frac{50}{1+25t}$. Now $Q(T)=25\Rightarrow1+25T=2\Rightarrow
25T=1\Rightarrow T=\frac{1}{25}$ years.
\end{solution}
\end{problem}

\begin{problem}\label{exer:4.1.13}
A super bread dough increases in volume at a rate
proportional to the volume $V$ present.  If $V$ increases by
a factor of 10 in 2 hours and $V(0)=V_0$, find $V$ at any
time $t$.  How long will it take for $V$ to increase to $100
V_0$?
\end{problem}

\begin{problem}\label{exer:4.1.14}
A radioactive substance decays at a rate proportional to the
amount present, and half the original quantity $Q_0$ is left
after 1500 years. In how many years would the original
amount be reduced to $3Q_0/4$?  How much will be
left after 2000 years?
\begin{solution}
Since $\tau=1500$, $k=\frac{\ln2}{1500}$; hence
$Q=Q_0e^{-(t\ln2)/1500}$. If $Q(t_1)=\frac{3Q_0}{4}$, then
$e^{-(t_1\ln2)/1500}=\frac{3}{4}$;\,
$-t_1\frac{\ln2}{1500}=\ln\left(\frac{3}{4}\right)=
-\ln\left(\frac{4}{3}\right)$;\,
$t_1=1500\ln\left(\frac{4}{3}\right)\over\ln2$. Finally,
$Q(2000)=Q_0e^{-\frac{4}{3}\ln2}=2^{-4/3}Q_0$.
\end{solution}
\end{problem}

\begin{problem}\label{exer:4.1.15}
A wizard creates gold continuously at the rate of 1 ounce
per hour, but an assistant steals it continuously at the
rate of 5\% of however much  is there per hour.  Let $W(t)$
be the number of ounces that the wizard has at time $t$.
Find $W(t)$ and $\lim_{t\to\infty}W(t)$ if  $W(0)=1$.
\end{problem}

\begin{problem}\label{exer:4.1.16}
A process creates a radioactive substance at the rate of 1
g/hr, and the substance decays at an hourly rate
 equal to 1/10 of the mass present (expressed in
grams).  Assuming that there are initially 20 g, find
the mass $S(t)$ of the substance present at time $t$, and
find $\lim_{t\to\infty} S(t)$.
\begin{solution}
We have $S'=1-\frac{S}{10},\ S(0)=20$. Rewrite this as $S'+\frac{S}{10}=1$. Since
$S_1=e^{-t/10}$ is a solution of the complementary equation, the
solutions are given by $S=ue^{-t/10}$, where $u'e^{-t/10}=1$.
Therefore, $u'=e^{t/10}$;\ $u=10e^{t/10}+c$;\;
$S=10+ce^{-t/10}$. Now $S(0)=20\Rightarrow c=10$, so
$S=10+10e^{-t/10}$ and
 $\lim_{t\to\infty}S(t)=10$ g.
\end{solution}
\end{problem}

\begin{problem}\label{exer:4.1.17}
A tank is empty at $t=0$.  Water is added to the tank at
the rate of 10 gal/min, but it leaks out at a rate
 (in gallons per minute)  equal to the number of
gallons in the tank.  What is the smallest capacity the tank
can have if this process is to continue forever?
\end{problem}

\begin{problem}\label{exer:4.1.18}
A person deposits \$25,000 in a bank that pays 5\% per
year interest, compounded continuously.  The person
continuously withdraws from the account at the rate of \$750
per year.  Find $V(t)$, the value of the account at time $t$
after the initial deposit.
\begin{solution}
We have $V'=-750+\frac{V}{20},\ V(0)=25000$. We can rewrite this as $V'-\frac{V}{20}=-750$. Since
$V_1=e^{t/20}$ is a solution of the complementary equation, the
solutions are given by $V=ue^{t/20}$, where $u'e^{t/20}=-750$.
Therefore, $u'=-750e^{-t/20}$;\ $u=15000e^{-t/20}+c$;\;
$V=15000+ce^{t/20}$;\ $V(0)=25000\Rightarrow c=10000$. Therefore,
$V=15000+10000e^{t/20}$.
\end{solution}
\end{problem}

\begin{problem}\label{exer:4.1.19}
  A person has a fortune that grows at rate proportional to the
square root of its worth. Find the  worth $W$ of the fortune as a function
of $t$ if it was  \$1 million 6 months ago and
is \$4 million today.
\end{problem}

\begin{problem}\label{exer:4.1.20}
Let $p=p(t)$ be the quantity of a  product present at
time $t$. The product is manufactured continuously at a
rate proportional to $p$, with proportionality constant 1/2,
and it's consumed continuously at a rate proportional to
$p^2$, with proportionality constant 1/8.  Find $p(t)$ if
$p(0)=100$.

\begin{solution}
$p'=\frac{p}{2}-\frac{p^2}{8}=-\frac{1}{8}p(p-4)$ 

$\frac{p'}{p(p-4)}=-\frac{1}{8}$ 

$\frac{1}{4}\left[
\frac{1}{p-4}-\frac{1}{p}\right]p'=-\frac{1}{8}$ 

 $\left[
\frac{1}{p-4}-\frac{1}{p}\right]p'=-\frac{1}{2}$ 

$\frac{p-4}{p}=-\frac{t}{2}+k$ 

$\frac{p-4}{p}=ce^{-t/2}$

$p(0)=100 \Rightarrow c=\frac{24}{25}$ 

$\frac{p-4}{p}=\frac{24}{25}e^{-t/2}$ 

$p-4=\frac{24}{25}pe^{-t/2}$ 

$p\left(1-\frac{24}{25} e^{-t/2}\right)=4$ 

$p=\frac{4}{1-\frac{24}{25}e^{-t/2}}=\frac{100}{24-24e^{-t/2}}$.
\end{solution}
\end{problem}

\begin{problem}\label{exer:4.1.21}

\begin{enumerate}
\item % (a)
 In the situation of Example~\ref{example:4.1.6}
find the exact value $P(t)$ of the person's account after $t$
years, where $t$ is an integer.  Assume that each year has
exactly 52 weeks, and include the year-end
deposit in the computation.

\begin{hint} 
At time $t$ the initial $\$1000$
has been on deposit for $t$ years.  There have been $52t$
deposits of $\$50$ each.  The first $\$50$ has been on deposit
for $t-1/52$ years, the second for $t-2/52$ years $\cdots$
in general, the $j$th $\$50$ has been on deposit for $t-j/52$
years $(1 \le j \le 52t)$.  Find the present value of each
$\$50$ deposit assuming $6$\% interest compounded
continuously, and use the formula
$$
1+x+x^2+\cdots+x^n=\frac{1-x^{n+1}}{1-x}\
(x \ne 1)
$$
to find their total value.
\end{hint}

\item % (b)
 Let
$$
p(t)=\frac{Q(t)-P(t)}{P(t)}
$$
be the relative error after $t$ years.  Find
$$
p(\infty)=\lim_{t\to\infty}p(t).
$$
\end{enumerate}
\end{problem}

\begin{problem}\label{exer:4.1.22}
A homebuyer borrows $P_0$ dollars at an annual interest rate $r$,
agreeing to repay the loan with equal monthly payments of $M$ dollars
per month over $N$ years.

\begin{enumerate} \item % (a)
Derive a
differential equation for the loan principal (amount that the
homebuyer owes) $P(t)$ at time $t>0$, making the simplifying
assumption that the homebuyer repays the loan continuously rather than
in discrete steps. (See Example~\ref{example:4.1.6} .)
\begin{solution}
    $P'=rP-12M$.
\end{solution}

\item %(b)
Solve the equation you just derived.
\begin{solution}
$P=ue^{rt}$

$u'e^{rt}=-12M$

$u'=-12Me^{rt}$

$u=\frac{12M}{r}e^{-rt}+c$

$P=\frac{12M}{r}+ce^{rt}$

$P(0)=P_0\Rightarrow c=P_0-\frac{12M}{r}$

$P=\frac{12M}{r}(1-e^{rt})+P_0e^{rt}$.
\end{solution}

\item % (c)
 Use your solution to determine an approximate
value for $M$  assuming that each year has exactly 12 months of
equal length.
\begin{solution}
Since $P(N)=0$, the previous argument implies that
 $M=\frac{rP_0}{12(1-e^{-rN})}$
\end{solution}

\item % (d)
 It can be shown that the exact value of $M$  is given by
$$
M=\frac{rP_0 }{ 12}\left(1-(1+r/12)^{-12N}\right)^{-1}.
$$
Compare the value of $M$ obtained from the problem you just did to the
exact value  if
 (i) $P_0=\$50,000$, $r=7\frac{1}{2}$\%, $N=20$
 (ii) $P_0=\$150,000$, $r=9.0$\%, $N=30$.
\end{enumerate}
\end{problem}

\begin{problem}\label{exer:4.1.23}
Assume that the homebuyer of Exercise \ref{exer:4.1.22} elects to repay the
loan continuously at the rate of $\alpha M$ dollars per month,
where $\alpha$ is a constant greater than 1. (This is called \emph{
accelerated payment}.)
\begin{enumerate}
\item % (a)
 Determine the time $T(\alpha)$ when the loan
will be paid off and the amount $S(\alpha)$ that the homebuyer will save.

\item % (b)
 Suppose $P_0=\$50,000$, $r=8$\%, and $N=15$.
Compute
the savings realized by accelerated payments with $\alpha=1.05,1.10$, and
$1.15$.
\end{enumerate}
\end{problem}

\begin{problem}\label{exer:4.1.24} A benefactor wishes to establish a trust fund to
pay a researcher's salary for $T$ years. The salary is to start at
$S_0$ dollars per year and increase at a fractional rate of $a$ per
year. Find the amount of money $P_0$ that the benefactor must deposit
in a trust fund paying interest at a rate $r$ per year. Assume that
the researcher's salary is paid continuously, the interest is
compounded continuously, and the salary increases are granted
continuously.
\begin{solution}
The researcher's salary is the solution of the initial value problem
$S'=aS,\ S(0)=S_0$. Therefore, $S=S_0e^{at}$. If $P=P(t)$ is the value
of the trust fund, then $P'=-S_0e^{at}+rP$, or $P'-rP=-S_0e^{at}$.
Therefore, 
\begin{equation}\label{solneqn:4.1.24A}
P=ue^{rt},  
\end{equation}
where $u'e^{rt}=-S_0e^{at}$,
 so 
\begin{equation}\label{solneqn:4.1.24B}
u'=-S_0e^{(a-r)t}.  
\end{equation}
If $a\ne r$, then \ref{solneqn:4.1.24B} implies that
$u=\frac{S_0}{r-a}e^{(a-r)t}+c$, so \ref{solneqn:4.1.24A} implies that
$P=\frac{S_0}{r-a}e^{at}+ce^{rt}$. Now $P(0)=P_0\Rightarrow
c=P_0-\frac{S_0}{r-a}$; therefore
$P=\frac{S_0}{r-a}e^{at}+\left(P_0-\frac{S_0}{r-a}\right)e^{rt}$.
We must choose $P_0$ so that $P(T)=0$; that is,
$P=\frac{S_0}{r-a}e^{aT}+\left(P_0-{S_0\over r-a}\right)e^{rT}=0$.
Solving this for $P_0$ yields
$P_0=\frac{S_0(1-e^{(a-r)T})}{r-a}$.
If $a=r$, then \ref{solneqn:4.1.24B} becomes
$u'=-S_0$, so $u=-S_0t+c$ and \ref{solneqn:4.1.24A} implies that
$P=(-S_0t+c)e^{rt}$. Now $P(0)=P_0\Rightarrow
c=P_0$; therefore
$P=(-S_0t+P_0)e^{rt}$.
To make $P(T)=0$ we must take $P_0=S_0T$.
\end{solution}
\end{problem}

\begin{problem}\label{exer:4.1.25}  
A radioactive substance with decay constant $k$ is produced
at the rate of
$$
\frac{at}{1+btQ(t)}
$$
 units of mass per unit time, where
$a$ and $b$ are positive constants and $Q(t)$ is the mass of the
substance present at time $t$;   thus, the rate of production is small
at the start
 and tends to slow when $Q$ is large.
\begin{enumerate}
\item % (a)
Set up a differential equation for $Q$.
\item % (b)
Choose your own positive values for $a$, $b$, $k$, and
$Q_0=Q(0)$.
Use a numerical method
to discover what happens to $Q(t)$
as $t\to\infty$. (Be precise, expressing your conclusions in terms of
$a$, $b$, $k$. However,  no proof is required.)
\end{enumerate}
\end{problem}

\begin{problem}\label{exer:4.1.26}  
Follow the instructions of Exercise~\ref{exer:4.1.25}, assuming that
the substance is produced at the rate of   $at/(1+bt(Q(t))^2)$
units of mass per unit of time.
\begin{solution}
$Q'=\frac{at}{1+btQ^2}-kQ$;\  $\lim_{t\to\infty}Q(t)=(a/bk)^{1/3}$.
\end{solution}
\end{problem}

\begin{problem}\label{exer:4.1.27}  
Follow the instructions of Exercise~\ref{exer:4.1.25}, assuming that
the substance is produced at the rate of   $at/(1+bt)$
units of mass per unit of time.
\end{problem}

\end{document}