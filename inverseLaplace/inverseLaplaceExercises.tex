\documentclass{ximera}
\input{../preamble.tex}

\title{Exercises} \license{CC BY-NC-SA 4.0}

\begin{document}

\begin{abstract}
\end{abstract}
\maketitle

\begin{onlineOnly}
\section*{Exercises}
\end{onlineOnly}

\begin{problem}\label{exer:8.2.1}
 Use the table of Laplace transforms  to find the inverse Laplace
transform.

\begin{enumerate}
    \item $\frac{3}{(s-7)^4}$
    \item $\frac{2s-4}{s^2-4s+13}$
    \item $\frac{1}{s^2+4s+20}$
    \item $\frac{2}{s^2+9}$
    \item $\frac{s^2-1}{(s^2+1)^2}$
    \item $\frac{1}{(s-2)^2-4}$
    \item $\frac{12s-24}{(s^2-4s+85)^2}$
    \item $\frac{2}{(s-3)^2-9}$
    \item $\frac{s^2-4s+3}{(s^2-4s+5)^2}$
\end{enumerate}
\end{problem}

\begin{problem}\label{exer:8.2.2}
 Use Theorem~\ref{thmtype:8.2.1} and the table of Laplace transforms
  to find the inverse Laplace transform.

\begin{enumerate}
    \item $\frac{2s+3}{(s-7)^4}$
    \item $\frac{s^2-1}{(s-2)^6}$
    \item $\frac{s+5}{s^2+6s+18}$
    \item $\frac{2s+1}{s^2+9}$
    \item $\frac{s}{s^2+2s+1}$
    \item $\frac{s+1}{s^2-9}$
    \item $\frac{s^3+2s^2-s-3}{(s+1)^4}$
    \item $\frac{2s+3}{(s-1)^2+4}$
    \item $\frac{1}{s}-\frac{s}{s^2+1}$
    \item $\frac{3s+4}{s^2-1}$
    \item $\frac{3}{s-1}+\frac{4s+1}{s^2+9}$
    \item $\frac{3}{(s+2)^2}-\frac{2s+6}{s^2+4}$
\end{enumerate}
\end{problem}

\begin{problem}\label{exer:8.2.3}
 Use Heaviside's method to find the inverse Laplace transform.

\begin{enumerate}
    \item $\frac{3-(s+1)(s-2)}{(s+1)(s+2)(s-2)}$
    \item $\frac{7+(s+4)(18-3s)}{(s-3)(s-1)(s+4)}$
    \item $\frac{2+(s-2)(3-2s)}{(s-2)(s+2)(s-3)}$
    \item $\frac{3-(s-1)(s+1)}{(s+4)(s-2)(s-1)}$
    \item $\frac{3+(s-2)(10-2s-s^2)}{(s-2)(s+2)(s-1)(s+3)}$
    \item $\frac{3+(s-3)(2s^2+s-21)}{(s-3)(s-1)(s+4)(s-2)}$
\end{enumerate}
\end{problem}

\begin{problem}\label{exer:8.2.4}
 Find the inverse Laplace transform.

\begin{enumerate}
    \item $\frac{2+3s}{(s^2+1)(s+2)(s+1)}$
    \item $\frac{3s^2+2s+1}{(s^2+1)(s^2+2s+2)}$
    \item $\frac{3s+2}{(s-2)(s^2+2s+5)}$
    \item $\frac{3s^2+2s+1}{(s-1)^2(s+2)(s+3)}$
    \item $\frac{2s^2+s+3}{(s-1)^2(s+2)^2}$
    \item $\frac{3s+2}{(s^2+1)(s-1)^2}$
\end{enumerate}
\end{problem}

\begin{problem}\label{exer:8.2.5}
 Use the method of Example~\ref{example:8.2.9}  to find the
inverse Laplace transform.

\begin{enumerate}
    \item $\frac{3s+2}{(s^2+4)(s^2+9)}$
    \item $\frac{-4s+1}{(s^2+1)(s^2+16)}$
    \item $\frac{5s+3}{(s^2+1)(s^2+4)}$
    \item $\frac{-s+1}{(4s^2+1)(s^2+1)}$
    \item $\frac{17s-34}{(s^2+16)(16s^2+1)}$
    \item $\frac{2s-1}{(4s^2+1)(9s^2+1)}$
\end{enumerate}
\end{problem}

\begin{problem}\label{exer:8.2.6}
 Find the inverse Laplace transform.

\begin{enumerate}
    \item $\frac{17 s-15}{(s^2-2s+5)(s^2+2s+10)}$
    \item $\frac{8s+56}{(s^2-6s+13)(s^2+2s+5)}$
    \item $\frac{s+9}{(s^2+4s+5)(s^2-4s+13)}$
    \item $\frac{3s-2}{(s^2-4s+5)(s^2-6s+13)}$
    \item $\frac{3s-1}{(s^2-2s+2)(s^2+2s+5)}$
    \item $\frac{20s+40}{(4s^2-4s+5)(4s^2+4s+5)}$
\end{enumerate}
\end{problem}

\begin{problem}\label{exer:8.2.7}
Find the inverse Laplace transform.

\begin{enumerate}
    \item $\frac{1}{s(s^2+1)}$
    \item $\frac{1}{(s-1)(s^2-2s+17)}$
    \item $\frac{3s+2}{(s-2)(s^2+2s+10)}$
    \item $\frac{34-17s}{(2s-1)(s^2-2s+5)}$
    \item $\frac{s+2}{(s-3)(s^2+2s+5)}$
    \item $\frac{2s-2}{(s-2)(s^2+2s+10)}$
\end{enumerate}
\end{problem}

\begin{problem}\label{exer:8.2.8}
 Find the inverse Laplace transform.
\begin{enumerate}
    \item $\frac{2s+1}{(s^2+1)(s-1)(s-3)}$
    \item $\frac{s+2}{(s^2+2s+2)(s^2-1)}$
    \item $\frac{2s-1}{(s^2-2s+2)(s+1)(s-2)}$
    \item $\frac{s-6}{(s^2-1)(s^2+4)}$
    \item $\frac{2s-3}{s(s-2)(s^2-2s+5)}$
    \item $\frac{5s-15}{(s^2-4s+13)(s-2)(s-1)}$
\end{enumerate}
\end{problem}

\begin{problem}\label{exer:8.2.9}
 Given that $f(t)\leftrightarrow F(s)$, find the inverse
Laplace transform of $F(as-b)$, where $a>0$.
\end{problem}

\begin{problem}\label{exer:8.2.10}
\begin{enumerate}
\item  % (a)
If $s_1$, $s_2$, \dots, $s_n$ are distinct and $P$ is a polynomial of
degree less than $n$, then
$$
\frac{P(s)}{(s-s_1)(s-s_2)\cdots(s-s_n)}=
\frac{A_1}{s-s_1}+\frac{A_2}{s-s_2}+\cdots+\frac{A_n}{s-s_n}.
$$
Multiply through by $s-s_i$ to show that
 $A_i$ can be obtained by ignoring the factor $s-s_i$ on the
left and setting $s=s_i$ elsewhere.
\item % (b)
Suppose $P$ and $Q_1$ are polynomials such that
$\text{degree}(P)\le\text{degree}(Q_1)$ and $Q_1(s_1)\ne0$.
Show that the coefficient of $1/(s-s_1)$ in the partial fraction
expansion of
$$
F(s)=\frac{P(s)}{(s-s_1)Q_1(s)}
$$
is $P(s_1)/Q_1(s_1)$.
\item % (c)
Explain how these two results above are related.
\end{enumerate}
\end{problem}

\end{document}