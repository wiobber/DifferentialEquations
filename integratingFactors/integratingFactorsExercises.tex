\documentclass{ximera}
\input{../preamble.tex}

\title{Exercises} \license{CC BY-NC-SA 4.0}

\begin{document}

\begin{abstract}
\end{abstract}
\maketitle

\begin{onlineOnly}
\section*{Exercises}
\end{onlineOnly}



\begin{problem}\label{exer:2.6.1}
\begin{enumerate}
\item % (a)
Verify that $\mu(x,y)=y$ is an integrating factor for
\begin{equation}\label{eqA:2.6.1}
y\,dx+\left(2x+\frac{1}{y}\right)\,dy=0
\end{equation}
on any open rectangle that does not intersect  the $x$ axis
or, equivalently,  that
\begin{equation}\label{eqB:2.6.1}
y^2\,dx+(2xy+1)\,dy=0
\end{equation}
is exact on any such rectangle.
\item % (b)
Verify that $y\equiv0$  is a solution of  (\ref{eqB:2.6.1}),
but not of  (\ref{eqA:2.6.1}).
\item % (b)
Show that
\begin{equation}\label{eqC:2.6.1}
y(xy+1)=c
\end{equation}
is an implicit solution of (\ref{eqB:2.6.1}), and explain why
every differentiable function $y=y(x)$ other than $y\equiv0$
that satisfies  (\ref{eqC:2.6.1})
 is also a solution of~(\ref{eqA:2.6.1}).
\end{enumerate}
\end{problem}

\begin{problem}\label{exer:2.6.2}
\begin{enumerate}
\item % (a)
Verify that $\mu(x,y)=1/(x-y)^2$ is an integrating factor for
\begin{equation}\label{eqA:2.6.2}
-y^2\,dx+x^2\,dy=0
\end{equation}
on any open rectangle  that does not intersect the line $y=x$
or, equivalently, that
\begin{equation}\label{eqB:2.6.2}
-\frac{y^2}{(x-y)^2}\,dx+\frac{x^2}{(x-y)^2}\,dy=0
\end{equation}
is exact on any such rectangle.
\item % (b)
Use Theorem~\ref{thmtype:2.2.1} to show that
\begin{equation}\label{eqC:2.6.2}
\frac{xy}{(x-y)}=c
\end{equation}
is an implicit solution of (\ref{eqB:2.6.2}), and explain why
it's also an implicit solution of (\ref{eqA:2.6.2})



\begin{solution}
    To show that $\mu(x,y)=\frac{1}{(x-y)^2}$
is an integrating
factor for (\ref{eqA:2.6.2}) and that (\ref{eqB:2.6.2}) is exact, it suffices to observe that
$\frac{\partial }{\partial x}\left(\frac{xy}{ x-y}\right)=-
\frac{y^2}{(x-y)^2}$ and
$\frac{\partial }{\partial y}\left(\frac{xy}{ x-y}\right)=
\frac{x^2}{(x-y)^2}$. By Theorem~\ref{thmtype:2.5.1} this also shows that
(\ref{eqC:2.6.2}) is an implicit solution of (\ref{eqB:2.6.2}). Since
$\mu(x,y)$ is never zero, any solution of (\ref{eqB:2.6.2})  is a solution of~(\ref{eqA:2.6.2}).
\end{solution}

\item % (c)
Verify that $y=x$  is a solution of  (\ref{eqA:2.6.2}),
even though it can't be obtained from  (\ref{eqC:2.6.2}).



\begin{solution}
    If we interpret (\ref{eqA:2.6.2}) as $-y^2+x^2y'=0$, then substituting
$y=x$ yields $-x^2+x^2\cdot1=0$.
\end{solution}
\end{enumerate}
\end{problem}

\begin{problem}\label{exer:2.6.3} Find an integrating factor; that is a function of only one variable, and solve the given equation.
$$y\,dx-x\,dy=0$$
\end{problem}

\begin{problem}\label{exer:2.6.4} Find an integrating factor; that is a function of only one variable, and solve the given equation.
$$3x^2y\,dx+2x^3\,dy=0$$



\begin{solution}
    $M(x,y)=3x^2y$;\;
$N(x,y)=2x^3$;\;
$M_y(x,y)-N_x(x,y)=3x^2-6x^2=-3x^2$;\;
$p(x)=\frac{M_y(x,y)-N_x(x,y)}{
N(x,y)}=-\frac{3x^2}{2x^3}=-\frac{3}{2x}$;\;
$\int p(x)\,dx=-\frac{3}{2}\ln|x|$;\;
$\mu(x)=P(x)=x^{-3/2}$;
therefore
$3x^{1/2}y\,dx+2x^{3/2}\,dy=0$
is exact.
We must find $F$ such that
(A) $F_x(x,y)=3x^{1/2}y$ and
(B) $F_y(x,y)=2x^{3/2}$.
Integrating (A) with respect to $x$ yields
(C) $F(x,y)=2x^{3/2}y+\phi(y)$.
Differentiating (C) with respect to $y$  yields
(D) $F_y(x,y)=2x^{3/2}+\phi'(y)$.
Comparing (D) with (B)  shows that
$\phi'(y)=0$, so we take
$\phi(y)=0$.
Substituting this into (C) yields
$F(x,y)=2x^{3/2}y$,
so $x^{3/2}y=c$.
\end{solution}
\end{problem}

\begin{problem}\label{exer:2.6.5} Find an integrating factor; that is a function of only one variable, and solve the given equation.
$$2y^3\,dx+3y^2\,dy=0$$
\end{problem}

\begin{problem}\label{exer:2.6.6} Find an integrating factor; that is a function of only one variable, and solve the given equation.
$$(5xy+2y+5)\,dx+2x\,dy=0$$



\begin{solution}
    $M(x,y)=5xy+2y+5$;\;
$N(x,y)=2x$;\;
$M_y(x,y)-N_x(x,y)=(5x+2)-2=5x$;\;
$p(x)=\frac{M_y(x,y)-N_x(x,y)}{
N(x,y)}=\frac{5x}{2x}=\frac{5}{2}$;\;
$\int p(x)\,dx=\frac{5x}{2}$;\;
$\mu(x)=P(x)=e^{5x/2}$;
therefore
$e^{5x/2}(5xy+2y+5)\,dx+2xe^{5x/2}\,dy=0$
is exact.
We must find $F$ such that
(A) $F_x(x,y)=e^{5x/2}(5xy+2y+5)$ and
(B) $F_y(x,y)=2xe^{5x/2}$.
Integrating (B) with respect to $y$ yields
(C) $F(x,y)=2xye^{5x/2}+\psi(x)$.
Differentiating (C) with respect to $x$  yields
(D) $F_x(x,y)=5xye^{5x/2}+2ye^{5x/2}+\psi'(x)$.
Comparing (D) with (A)  shows that
$\psi'(x)=5e^{5x/2}$, so we take
$\psi(x)=2e^{5x/2}$.
Substituting this into (C) yields
$F(x,y)=2e^{5x/2}(xy+1)$,
so $e^{5x/2}(xy+1)=c$.
\end{solution}
\end{problem}

\begin{problem}\label{exer:2.6.7} Find an integrating factor; that is a function of only one variable, and solve the given equation.
$$(xy+x+2y+1)\,dx+(x+1)\,dy=0$$
\end{problem}

\begin{problem}\label{exer:2.6.8} Find an integrating factor; that is a function of only one variable, and solve the given equation.
$$(27xy^2+8y^3)\,dx+(18x^2y+12xy^2)\,dy=0$$



\begin{solution}
    $M(x,y)=27xy^2+8y^3$;\;
$N(x,y)=18x^2y+12xy^2$;\;
$M_y(x,y)-N_x(x,y)=(54xy+24y^2)-(36xy+12y^2)=18xy+12y^2$;\;
$p(x)=\frac{M_y(x,y)-N_x(x,y)}{ N(x,y)}=\frac{18xy+12y^2}{
18x^2y+12y^2x}=\frac{1}{ x}$;\;
$\int p(x)\,dx=\ln|x|$;\;
$\mu(x)=P(x)=x$;
therefore
$(27x^2y^2+8xy^3)\,dx+(18x^3y+12x^2y^2)\,dy=0$
is exact.
We must find $F$ such that
(A) $F_x(x,y)=27x^2y^2+8xy^3$ and
(B) $F_y(x,y)=18x^3y+12x^2y^2$.
Integrating (A) with respect to $x$ yields
(C) $F(x,y)=9x^3y^2+4x^2y^3+\phi(y)$.
Differentiating (C) with respect to $y$  yields
(D) $F_y(x,y)=18x^3y+12x^2y^2+\phi'(y)$.
Comparing (D) with (B)  shows that
$\phi'(y)=0$, so we take
$\phi(y)=0$.
Substituting this into (C) yields
$F(x,y)=9x^3y^2+4x^2y^3$,
so $x^2y^2(9x+4y)=c$.
\end{solution}
\end{problem}

\begin{problem}\label{exer:2.6.9} Find an integrating factor; that is a function of only one variable, and solve the given equation.
$$(6xy^2+2y)\,dx+(12x^2y+6x+3)\,dy=0$$
\end{problem}

\begin{problem}\label{exer:2.6.10} Find an integrating factor; that is a function of only one variable, and solve the given equation.
$$y^2\,dx+\left(xy^2+3xy+\frac{1}{y}\right)\,dy=0$$



\begin{solution}
    $M(x,y)=y^2$;\;
$N(x,y)= \left(xy^2+3xy+\frac{1}{ y}\right)$;\;
$M_y(x,y)-N_x(x,y)=2y-(y^2+3y)=-y(y+1)$;\;
$q(y)=\frac{N_x(x,y)-M_y(x,y)}{ M(x,y)}=\frac{y(y+1)}{ y^2}=
1+\frac{1}{ y}$;\;
$\int q(y)\,dy=y\ln|y|$;\;
$\mu(y)=Q(y)=ye^y$;
therefore
$y^3e^y\,dx+e^y(xy^3+3xy^2+1)\,dy=0$
is exact.
We must find $F$ such that
(A) $F_x(x,y)=y^3e^y$ and
(B) $F_y(x,y)=e^y(xy^3+3xy^2+1)$.
Integrating (A) with respect to $x$ yields
(C) $F(x,y)=xy^3e^y+\phi(y)$.
Differentiating (C) with respect to $y$  yields
(D) $F_y(x,y)=xy^3e^y+3xy^2e^y+\phi'(y)$.
Comparing (D) with (B)  shows that
$\phi'(y)=e^y$, so we take
$\phi(y)=e^y$.
Substituting this into (C) yields
$F(x,y)=xy^3e^y+e^y$,
so $e^y(xy^3+1)=c$.
\end{solution}
\end{problem}

\begin{problem}\label{exer:2.6.11} Find an integrating factor; that is a function of only one variable, and solve the given equation.
$$(12x^3y+24x^2y^2)\,dx+(9x^4+32x^3y+4y)\,dy=0$$
\end{problem}

\begin{problem}\label{exer:2.6.12} Find an integrating factor; that is a function of only one variable, and solve the given equation.
$$(x^2y+4xy+2y)\,dx+(x^2+x)\,dy=0$$



\begin{solution}
    $M(x,y)=x^2y+4xy+2y$;\;
$N(x,y)=x^2+x$;\;
$M_y(x,y)-N_x(x,y)=(x^2+4x+2)-(2x+1)=x^2+2x+1=(x+1)^2$;\;
$p(x)=\frac{M_y(x,y)-N_x(x,y)}{ N(x,y)}=\frac{(x+1)^2}{
x(x+1)}=1+\frac{1}{ x}$;\;
$\int p(x)\,dx=x+\ln|x|$;\;
$\mu(x)=P(x)=xe^x$;
therefore
$e^x(x^3y+4x^2y+2xy)\,dx+e^x(x^3+x^2)\,dy=0$
is exact.
We must find $F$ such that
(A) $F_x(x,y)=e^x(x^3y+4x^2y+2xy)$ and
(B) $F_y(x,y)=e^x(x^3+x^2)$.
Integrating (B) with respect to $y$ yields
(C) $F(x,y)=y(x^3+x^2)e^x+\psi(x)$.
Differentiating (C) with respect to $x$  yields
(D) $F_x(x,y)=e^x(x^3y+4x^2y+2xy)+\psi'(x)$.
Comparing (D) with (A)  shows that
$\psi'(x)=0$, so we take
$\psi(x)=0$.
Substituting this into (C) yields
$F(x,y)=y(x^3+x^2)e^x=x^2y(x+1)e^x$,
so $x^2y(x+1)e^x=c$.
\end{solution}
\end{problem}

\begin{problem}\label{exer:2.6.13} Find an integrating factor; that is a function of only one variable, and solve the given equation.
$$-y\,dx+(x^4-x)\,dy=0$$
\end{problem}

\begin{problem}\label{exer:2.6.14} Find an integrating factor; that is a function of only one variable, and solve the given equation.
$$\cos x\cos y\,dx+(\sin x\cos y-\sin x\sin y+y)\,dy=0$$



\begin{solution}
    $M(x,y)=\cos x\cos y$;\;
$N(x,y)=\sin x\cos y-\sin x\sin y+y$;\;
$M_y(x,y)-N_x(x,y)=-\cos x\sin y-(\cos x\cos y-\cos x\sin y)
=-\cos x\cos y$;\;
$q(y)=\frac{N_x(x,y)-M_y(x,y)}{ M(x,y)}=\frac{\cos x\cos y
}{\cos x\cos y}=1$;\;
$\int q(y)\,dy=1$;\;
$\mu(y)=Q(y)=e^y$;
therefore
$e^y\cos x\cos y\,dx+e^y(\sin x\cos y-\sin x\sin y+y)\,dy=0$
is exact.
We must find $F$ such that
(A) $F_x(x,y)=e^y\cos x\cos y$ and
(B) $F_y(x,y)=e^y(\sin x\cos y-\sin x\sin y+y)$.
Integrating (A) with respect to $x$ yields
(C) $F(x,y)=e^y\sin x\cos y+\phi(y)$.
Differentiating (C) with respect to $y$  yields
(D) $F_y(x,y)=e^y(\sin x\cos y-\sin x\sin y)+\phi'(y)$.
Comparing (D) with (B)  shows that
$\phi'(y)=ye^y$, so we take
$\phi(y)=e^y(y-1)$.
Substituting this into (C) yields
$F(x,y)=e^y(\sin x\cos y+y-1)$,
so $e^y(\sin x\cos y+y-1)=c$.
\end{solution}
\end{problem}

\begin{problem}\label{exer:2.6.15} Find an integrating factor; that is a function of only one variable, and solve the given equation.
$$(2xy+y^2)\,dx+(2xy+x^2-2x^2y^2-2xy^3)\,dy=0$$
\end{problem}

\begin{problem}\label{exer:2.6.16} Find an integrating factor; that is a function of only one variable, and solve the given equation.
$$y\sin y\,dx+x(\sin y-y\cos y)\,dy=0$$



\begin{solution}
    $M(x,y)=y\sin y$;\;
$N(x,y)=x(\sin y-y\cos y)$;\;
 $M_y(x,y)-N_x(x,y)=(y\cos y+\sin y)-(\sin y-y\cos y)=2y\cos y$;
$q(y)=\frac{N_x(x,y)-M_y(x,y)}{ N(x,y)}=-\frac{2\cos y}{\sin y}$;\;
$\int q(y)\,dy=-2\ln|\sin y|$;\;
$\mu(y)=Q(y)=\frac{1}{\sin^2y}$;
therefore
$\left({y}{\sin y}\right)\,dx+
x\left(\frac{1}{\sin y}-\frac{y\cos y}{\sin^2y}\right)\,dy=0$
is exact. We must find $F$ such that
(A) $F_x(x,y)=\frac{y}{\sin y}$ and
(B) $F_y(x,y)=x\left(\frac{1}{\sin y}-\frac{y\cos y}{\sin^2y}\right)$.
Integrating (A) with respect to $x$ yields
(C) $F(x,y)=\frac{xy}{\sin y}+\phi(y)$.
Differentiating (C) with respect to $y$  yields
(D) $F_y(x,y)=x\left(\frac{1}{\sin y}-\frac{y\cos y}{\sin^2y}\right)
+\phi'(y)$.
Comparing (D) with (B)  shows that
$\phi'(y)=0$, so we take
$\phi(y)=0$.
Substituting this into (C) yields
$F(x,y)=\frac{xy}{\sin y}$,
so $\frac{xy}{\sin y}=c$. In addition, the given equation has the
constant solutions $y=k\pi$, where $k$ is an integer.
\end{solution}
\end{problem}

\begin{problem}\label{exer:2.6.17} Find an integrating factor of the form $\mu(x,y)=P(x)Q(y)$ and solve the given equation.
$$y(1+5\ln|x|)\,dx+4x\ln|x|\,dy=0$$
\end{problem}

 \begin{problem}\label{exer:2.6.18} Find an integrating factor of the form $\mu(x,y)=P(x)Q(y)$ and solve the given equation.
 $$(\alpha y+ \gamma xy)\,dx+(\beta x+ \delta xy)\,dy=0$$

 

 \begin{solution}
     $M(x,y)=\alpha y+\gamma xy$;\;
$N(x,y)=\beta x+ \delta xy$;\;
 $M_y(x,y)-N_x(x,y)=(\alpha+\gamma x)-(\beta+\delta y)$;
and  $p(x)N(x,y)-q(y)M(x,y)=p(x)x(\beta + \delta y)-
q(y)y(\alpha +\gamma x)$.
so exactness requires that
 $(\alpha+\gamma x)-(\beta+\delta y)=
p(x)x(\beta+\delta y)-
q(y)y(\alpha+\gamma x)$,
which holds if
  $p(x)x=-1$ and $q(y)y=-1$. Thus
 $p(x)=-\frac{1}{ x}$;\;
 $q(y)=-\frac{1}{ y}$;\;
$\int p(x)\,dx=-\ln|x|$;\;
$\int q(y)\,dy=-\ln|y|$;\;
$P(x)=\frac{1}{ x}$;
$Q(y)=\frac{1}{ y}$;
$\mu(x,y)=\frac{1}{ xy}$.
Therefore,
$\left(\frac{\alpha}{ x}+\gamma\right)\,dx+
\left(\frac{\beta}{ y}+\delta\right)\,dy=0$
is exact.
We must find $F$ such that
(A) $F_x(x,y)=\frac{\alpha}{ x}+\gamma$ and
(B) $F_y(x,y)=\frac{\beta}{ y}+\delta$.
Integrating (A) with respect to $x$ yields
(C) $F(x,y)=\alpha\ln|x|+\gamma x+\phi(y)$.
Differentiating (C) with respect to $y$  yields
(D) $F_y(x,y)=\phi'(y)$.
Comparing (D) with (B)  shows that
$\phi'(y)=\frac{\beta}{ y}+\delta$, so we take
$\phi(y)=\beta\ln|y|+\delta y$.
Substituting this into (C) yields
$F(x,y)=\alpha\ln|x|+\gamma x+\beta\ln|y|+\delta y$,
so $|x|^\alpha|y|^\beta e^{\gamma x}e^{\delta y}=c$.
The given equation also has the solutions $x\equiv0$
and $y\equiv0$.

 \end{solution}

 \end{problem}

\begin{problem}\label{exer:2.6.19}Find an integrating factor of the form $\mu(x,y)=P(x)Q(y)$ and solve the given equation.
$$(3x^2y^3-y^2+y)\,dx+(-xy+2x)\,dy=0$$
\end{problem}

\begin{problem}\label{exer:2.6.20} Find an integrating factor of the form $\mu(x,y)=P(x)Q(y)$ and solve the given equation.
$$2y\,dx+ 3(x^2+x^2y^3)\,dy=0$$



\begin{solution}
    $M(x,y)=2y$;\;
$N(x,y)=3(x^2+x^2y^3)$;\;
 $M_y(x,y)-N_x(x,y)=2-(6x+6xy^3)$;
and  $p(x)N(x,y)-q(y)M(x,y)=3p(x)(x^2+x^2y^3)-2q(y)y$.
so exactness requires that
(A) $2-6x-6xy^3=3p(x)x(x+xy^3)-2q(y)y$.
To obtain  similar terms on the two sides of (A)
we let  $p(x)x=a$ and $q(y)y=b$  where $a$ and $b$ are constants
such that $2-6x-6xy^3=3a(x+xy^3)-2b$, which  holds if
$a=-2$ and $b=-1$. Thus,
 $p(x)=-\frac{2}{ x}$;\;
 $q(y)=-\frac{1}{ y}$;\;
$\int p(x)\,dx=-2\ln|x|$;\;
$\int q(y)\,dy=-\ln|y|$;\;
$P(x)=\frac{1}{ x^2}$;
$Q(y)=\frac{1}{ y}$;
$\mu(x,y)=\frac{1}{ x^2y}$.
Therefore,
$\frac{2}{ x^2}\,dx+3\left(\frac{1}{ y}+y^2\right)\,dy=0$
is exact.
We must find $F$ such that
(B) $F_x(x,y)=\frac{2}{ x^2}$ and
(C) $F_y(x,y)=3\left(\frac{1}{ y}+y^2\right)$.
Integrating (B) with respect to $x$ yields
(D) $F(x,y)=-\frac{2}{ x}+\phi(y)$.
Differentiating (D) with respect to $y$  yields
(E) $F_y(x,y)=\phi'(y)$.
Comparing (E) with (C)  shows that
$\phi'(y)=3\left(\frac{1}{ y}+y^2\right)$, so we take
$\phi(y)=y^3+3\ln|y|$.
Substituting this into (D) yields
$F(x,y)=-\frac{2}{ x}+y^3+3\ln|y|$,
so $-\frac{2}{ x}+y^3+3\ln|y|=c$.
The given equation also has the solutions $x\equiv0$
and $y\equiv0$.
\end{solution}
\end{problem}

\begin{problem}\label{exer:2.6.21}Find an integrating factor of the form $\mu(x,y)=P(x)Q(y)$ and solve the given equation.
$$(a\cos xy-y\sin xy)\,dx+(b\cos xy-x\sin xy)\,dy=0$$
\end{problem}

\begin{problem}\label{exer:2.6.22}Find an integrating factor of the form $\mu(x,y)=P(x)Q(y)$ and solve the given equation.
$$x^4y^4\,dx+x^5y^3\,dy=0$$



\begin{solution}
    $M(x,y)=x^4y^4$;\;
$N(x,y)=x^5y^3$;\;
 $M_y(x,y)-N_x(x,y)=4x^4y^3-5x^4y^3=-x^4y^3$;
and  $p(x)N(x,y)-q(y)M(x,y)=p(x)x^5y^3-q(y)x^4y^4$.
so exactness requires that
 $-x^4y^3=p(x)x^5y^3-q(y)x^4y^4$, which is equivalent to
 $p(x)x-q(y)y=-1$. This holds if
 $p(x)x=a$ and $q(y)y=a+1$  where $a$ is an arbitrary real number.
Thus,
 $p(x)=\frac{a}{x}$;\;
 $q(y)=\frac{a+1}{y}$;\;
$\int p(x)\,dx=a\ln|x|$;\;
$\int q(y)\,dy=(a+1)\ln|y|$;\;
$P(x)=|x|^a$;
$Q(y)=|y|^{a+1}$;
$\mu(x,y)=|x^a||y|^{a+1}$.
Therefore,
$|x|^a|y|^{a+1}\left(x^4y^4\,dx+x^5y^3\,dy\right)=0$
is exact for any choice of $a$. For simplicity we let $a=-4$,
so (A)  is equivalent to $y\,dx+x\,dy=0$.
We must find $F$ such that
(B) $F_x(x,y)=y$ and
(C) $F_y(x,y)=x$.
Integrating (B) with respect to $x$ yields
(D) $F(x,y)=xy+\phi(y)$.
Differentiating (D) with respect to $y$  yields
(E) $F_y(x,y)=x+\phi'(y)$.
Comparing (E) with (C)  shows that
$\phi'(y)=0$, so we take
$\phi(y)=0$.
Substituting this into (D) yields
$F(x,y)=xy$,
so $xy=c$.

\end{solution}
\end{problem}

\begin{problem}\label{exer:2.6.23} Find an integrating factor of the form $\mu(x,y)=P(x)Q(y)$ and solve the given equation.
$$y(x\cos x+2\sin x)\,dx+x(y+1)\sin x\,dy=0$$
\end{problem}

\begin{problem}\label{exer:2.6.24} Find an integrating factor and solve the  equation. Plot a direction field and some integral curves for the equation in the indicated rectangular region.
 $$(x^4y^3+y)\,dx+(x^5y^2-x)\,dy=0;   \quad \{-1\leq x\leq 1,-1\leq  y\leq 1\}$$

 

 \begin{solution}
     $M(x,y)=x^4y^3+y$;\;
$N(x,y)=x^5y^2-x$;\;
$M_y(x,y)-N_x(x,y)=(3x^4y^2+1)-(5x^4y^2-1)=-2x^4y^2+2$;\;
$p(x)=\frac{M_y(x,y)-N_x(x,y)}{ N(x,y)}=-\frac{2x^4y^2-2
}{ x^5y^2-x}=-\frac{2}{ x}$;\;
$\int p(x)\,dx=-2\ln|x|$;\;
$\mu(x)=P(x)=\frac{1}{ x^2}$;
therefore
$\left(x^2y^3+\frac{y}{ x^2}\right)\,dx+
\left(x^3y^2-\frac{1}{ x}\right)\,dy=0$
is exact.
We must find $F$ such that
(A) $F_x(x,y)=\left(x^2y^3+\frac{y}{ x^2}\right)$ and
(B) $F_y(x,y)=\left(x^3y^2-\frac{1}{ x}\right)$.
Integrating (A) with respect to $x$ yields
(C) $F(x,y)=\frac{x^3y^3}{3}-\frac{y}{ x}+\phi(y)$.
Differentiating (C) with respect to $y$  yields
(D) $F_y(x,y)=x^3y^2-\frac{1}{ x}+\phi'(y)$.
Comparing (D) with (B)  shows that
$\phi'(y)=0$, so we take
$\phi(y)=0$.
Substituting this into (C) yields
$F(x,y)=\frac{x^3y^3}{3}-\frac{y}{ x}$,
so $\frac{x^3y^3}{3}-\frac{y}{ x}=c$.

 \end{solution}
\end{problem}

\begin{problem}\label{exer:2.6.25}Find an integrating factor and solve the  equation. Plot a direction field and some integral curves for the equation in the indicated rectangular region.
$$(3xy+2y^2+y)\,dx+(x^2+2xy+x+2y)\,dy=0;   \quad \{-2\leq x\leq 2,-2\leq y\leq 2\}$$
\end{problem}

\begin{problem}\label{exer:2.6.26} Find an integrating factor and solve the  equation. Plot a direction field and some integral curves for the equation in the indicated rectangular region.
$$(12 xy+6y^3)\,dx+(9x^2+10xy^2)\,dy=0;   \quad \{-2\leq x\leq 2,-2\leq y\leq 2\}$$



\begin{solution}
    $M(x,y)=12xy+6y^3$;\;
$N(x,y)=9x^2+10xy^2$;\;
 $M_y(x,y)-N_x(x,y)=(12x+18y^2)-(18x+10y^2)=-6x+8y^2$;
and  $p(x)N(x,y)-q(y)M(x,y)=p(x)x(9x+10y^2)-q(y)y(12x+6y^2)$,
so exactness requires that
(A) $-6x+8y^2=p(x)x(9x+10y^2)-q(y)y(12x+6y^2)$.
To obtain  similar terms on the two sides of (A)
we let  $p(x)x=a$ and $q(y)y=b$  where $a$ and $b$ are constants
such that $-6x+8y^2=a(9x+10y^2)-b(12x+6y^2)$, which holds if
  $9a-12b=-6$, $10a-6b=8$; that is, $a=b=2$. Thus
 $p(x)=\frac{2}{ x}$;\;
 $q(y)=\frac{2}{ y}$;\;
$\int p(x)\,dx=2\ln|x|$;\;
$\int q(y)\,dy=2\ln|y|$;\;
$P(x)=x^2$;
$Q(y)=y^2$;
$\mu(x,y)=x^2y^2$.
Therefore,
$(12x^3y^3+6x^2y^5)\,dx+(9x^4y^2+10x^3y^4)\,dy=0$
is exact.
We must find $F$ such that
(B) $F_x(x,y)=12x^3y^3+6x^2y^5$ and
(C) $F_y(x,y)=9x^4y^2+10x^3y^4$.
Integrating (B) with respect to $x$ yields
(D) $F(x,y)=3x^4y^3+2x^3y^5+\phi(y)$.
Differentiating (D) with respect to $y$  yields
(E) $F_y(x,y)=9x^4y^2+10x^3y^4+\phi'(y)$.
Comparing (E) with (C)  shows that
$\phi'(y)=0$, so we take
$\phi(y)=0$.
Substituting this into (D) yields
$F(x,y)=3x^4y^3+2x^3y^5$,
so $x^3y^3(3x+2y^2)=c$.

\end{solution}
\end{problem}

\begin{problem}\label{exer:2.6.27} Find an integrating factor and solve the  equation. Plot a direction field and some integral curves for the equation in the indicated rectangular region.
$$(3x^2y^2+2y)\,dx+ 2x\,dy=0;   \quad \{-4\leq x\leq 4,-4\leq y\leq 4\}$$
\end{problem}

\begin{problem}\label{exer:2.6.28}
Suppose $a$, $b$, $c$, and $d$ are constants such that
$ad-bc\ne0$, and let $m$ and $n$ be arbitrary real numbers.
Show that
$$
(ax^my+by^{n+1})\,dx+(cx^{m+1}+dxy^n)\,dy=0
$$
has an integrating factor $\mu(x,y)=x^\alpha y^\beta$.

Click below to see answer.

\begin{solution}
    $M(x,y)=ax^my+by^{n+1}$;\ $N(x,y)=cx^{m+1}+dxy^n$;\;
$M_y(x,y)-N_x(x,y)=\left[ax^{m+1}+(n+1)by^n\right]-\left[(m+1)cx^m
+dy^n\right]$;\;
$p(x)N(x,y)-q(y)M(x,y)=xp(x)(cx^m+dy^n)-yp(y)(ax^m+by^n)$.
Let (A)  $xp(x)=\alpha$ and (B) $yp(y)=\beta$, where $\alpha$
and $\beta$ are to be chosen so that
$\left[ax^{m+1}+(n+1)by^n\right]-\left[(m+1)cx^m
+dy^n\right]=\alpha(cx^m+dy^n)-\beta(ax^m+by^n)$, which will hold if
$$
\begin{array}{rclcl}
c\alpha-a\beta&=&\phantom{-}a-(m+1)c&=_{\mbox{df}}&A\\
d\alpha-b\beta&=&-d+\,(n+1)b&=_{\mbox{df}}&B.
\end{array}
\eqno{\rm (C)}
$$
Since $ad-bc\neq 0$ it can be verified that
$\alpha=\frac{aB-bA}{ ad-bc}$ and $\beta=\frac{cB-dA}{ ad-bc}$
satisfy (C). From (A) and (B),
$p(x)=\frac{\alpha}{ x}$ and $q(y)=\frac{\beta}{ y}$, so
$\mu(x,y)=x^\alpha y^\beta$  is an integrating factor for the given
equation.
\end{solution}
\end{problem}

\begin{problem}\label{exer:2.6.29}
Suppose $M$, $N$, $M_x$, and $N_y$  are continuous for all
$(x,y)$, and $\mu=\mu(x,y)$  is an integrating factor for
\begin{equation}\label{eqA:2.6.29}
M(x,y)\,dx+N(x,y)\,dy=0.
\end{equation}

Assume that $\mu_x$ and $\mu_y$ are continuous for all $(x,y)$, and
suppose   $y=y(x)$ is a differentiable function such that
$\mu(x,y(x))=0$ and $\mu_x(x,y(x))\ne0$ for all $x$ in some interval $I$. Show that $y$ is a solution of (\ref{eqA:2.6.29}) on $I$.
\end{problem}

\begin{problem}\label{exer:2.6.30}
According to Theorem~\ref{thmtype:2.1.2}, the general solution of the linear nonhomogeneous equation
\begin{equation}\label{eqA:2.6.30}
y'+p(x)y=f(x)
\end{equation}
is
\begin{equation}\label{eqB:2.6.30}
y=y_1(x)\left(c+\int f(x)/y_1(x)\,dx\right),
\end{equation}

where $y_1$ is any nontrivial solution of the
complementary equation $y'+p(x)y=0$. In this exercise we obtain this conclusion in a different way.
You may find it instructive to apply the method suggested here
to solve some of the exercises in Section~2.1.
\begin{enumerate}
\item % (a)
Rewrite (\ref{eqA:2.6.30}) as
\begin{equation}\label{eqC:2.6.30}
[p(x)y-f(x)]\,dx+\,dy=0,
\end{equation}

and show that $\mu=\pm e^{\int p(x)\,dx}$ is an integrating factor
for (\ref{eqC:2.6.30}).



\begin{solution}
    Since $M(x,y)=p(x)y-f(x)$  and $N(x,y)=1$,
$\frac{M_y(x,y)-N_x(x,y)}{N(x,y)}=p(x)$ and Theorem~\ref{thmtype:2.6.1}
implies that $\mu(x)\pm e^{\int p(x)\,dx}$  is an integrating factor
for (\ref{eqC:2.6.30}).
\end{solution}

\item % (b)
Multiply (\ref{eqA:2.6.30}) through by $\mu=\pm e^{\int p(x)\,dx}$  and verify that the resulting equation can be rewritten as
$$
(\mu(x)y)'=\mu(x)f(x).
$$
Then integrate both sides of this equation and solve for $y$
to show that the general solution of (\ref{eqA:2.6.30}) is
$$
y=\frac{1}{\mu(x)}\left(c+\int f(x)\mu(x)\,dx\right).
$$
Why is this form of the general solution
 equivalent to (\ref{eqB:2.6.30})?

 

 \begin{solution}
     Multiplying (\ref{eqA:2.6.30}) through $\mu=\pm e^{\int p(x)\,dx}$
yields  (D) $\mu(x)y'+\mu'(x)y=\mu(x)f(x)$, which is equivalent to
$(\mu(x)y)'=\mu(x)f(x)$. Integrating this yields
$\mu(x)y=c+\int\mu(x)f(x)\,dx$, so
$y=\frac{1}{\mu(x)}\left(c+\int
\mu(x)f(x)\,dx\right)$, which  is equivalent to (\ref{eqB:2.6.30}) since
$y_1=\frac{1}{\mu}$ is a nontrivial solution of $y'+p(x)y=0$.
 \end{solution}
\end{enumerate}
\end{problem}

\end{document}


