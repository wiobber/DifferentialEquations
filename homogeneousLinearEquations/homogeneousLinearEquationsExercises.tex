\documentclass{ximera}
\input{../preamble.tex}

\title{Exercises} \license{CC BY-NC-SA 4.0}

\begin{document}

\begin{abstract}
\end{abstract}
\maketitle

\begin{onlineOnly}
\section*{Exercises}
\end{onlineOnly}


\begin{problem}\label{exer:5.1.1}

\begin{enumerate}
\item % (a)
Verify that $y_1=e^{2x}$ and $y_2=e^{5x}$ are solutions of
\begin{equation}\label{eq:eqA5.1.1}
    y''-7y'+10y=0
\end{equation}
on  $(-\infty,\infty)$.
\item % (b)
Verify that if $c_1$ and $c_2$ are arbitrary constants then
$y=c_1e^{2x}+c_2e^{5x}$ is a solution of \ref{eq:eqA5.1.1} on
$(-\infty,\infty)$.
\item % (c)
Solve the initial value problem
$$
y''-7y'+10y=0,\quad  y(0)=-1,\quad y'(0)=1.
$$
\item % (d)
Solve the initial value problem
$$
y''-7y'+10y=0,\quad  y(0)=k_0,\quad y'(0)=k_1.
$$
\end{enumerate}
\end{problem}

\begin{problem}\label{exer:5.1.2}
\begin{enumerate}
\item % (a)
Verify that $y_1=e^x\cos x$ and $y_2=e^x\sin x$ are solutions of
\begin{equation}\label{eq:eqA5.1.2}
    y''-2y'+2y=0
\end{equation}
on  $(-\infty,\infty)$.

\begin{solution}
    If $y_1=e^x\cos x$, then $y_1'=e^x(\cos x -\sin x)$  and
$y_1''=e^x(\cos x-\sin x-\sin x-\cos x)=-2e^x\sin x$, so
$y_1''-2y_1'+2y_1=e^x(-2\sin x-2\cos x+2\sin x+2\cos x)=0$. If
$y_2=e^x\sin x$, then
$y_2'=e^x(\sin x+\cos x)$ and
$y_2''=e^x(\sin x+\cos x+\cos x-\sin x)=2e^x\cos x$, so
$y_2''-2y_2'+2y_2=e^x(2\cos x-2\sin x-2\cos x+2\sin x)=0$.
\end{solution}

\item % (b)
Verify that if $c_1$ and $c_2$ are arbitrary constants then
$y=c_1e^x\cos x+c_2e^x\sin x$ is a solution of \ref{eq:eqA5.1.2}
on
$(-\infty,\infty)$.

\begin{solution}
    If (B) $y=e^x(c_1\cos x+c_2\sin x)$, then
$$
y'=e^x\left(c_1(\cos x-\sin x)+c_2(\sin x+\cos x)\right)
\eqno{\rm (C)}
$$
 and
\begin{eqnarray*}
y''&=&c_1e^x(\cos x-\sin x-\sin x-\cos x)\\&&+c_2e^x(\sin x+\cos
x+\cos x-\sin x)\\ &=&2e^x(-c_1\sin x+c_2\cos x),
\end{eqnarray*}
 so
\begin{eqnarray*}
y''-2y'+2y&=&c_1e^x(-2\sin x-2\cos x+2\sin x+2\cos x)\\
&&+c_2e^x(2\cos x-2\sin x-2\cos x+2\sin x)=0.
\end{eqnarray*}
\end{solution}


\item % (c)
Solve the initial value problem
$$
y''-2y'+2y=0,\quad  y(0)=3,\quad y'(0)=-2.
$$

\begin{solution}
    We must choose $c_1$ and $c_2$ in (B) so that $y(0)=3$
and $y'(0)=-2$. Setting $x=0$ in (B) and (C) (see solution to the previous part) shows that
$c_1=3$ and $c_1+c_2=-2$, so $c_2=-5$. Therefore,
 $y=e^x(3\cos x-5\sin x)$.
\end{solution}


\item % (d)
Solve the initial value problem
$$
y''-2y'+2y=0,\quad  y(0)=k_0,\quad y'(0)=k_1.
$$

\begin{solution}
    We must choose $c_1$ and $c_2$ in (B) so that $y(0)=k_0$
and $y'(0)=k_1$. Setting $x=0$ in (B) and (C) shows that
$c_1=k_0$ and $c_1+c_2=k_1$, so $c_2=k_1-k_0$.  Therefore,
$y=e^x\left(k_0\cos x+(k_1-k_0)\sin x\right)$.
\end{solution}
\end{enumerate}
\end{problem}

\begin{problem}\label{exer:5.1.3}
\begin{enumerate}
\item % (a)
Verify that $y_1=e^x$ and $y_2=xe^x$ are solutions of
\begin{equation}\label{eq:eqA5.1.3}
    y''-2y'+y=0
\end{equation}
on  $(-\infty,\infty)$.
\item % (b)
Verify that if $c_1$ and $c_2$ are arbitrary constants then
$y=e^x(c_1+c_2x)$ is a solution of \ref{eq:eqA5.1.3} on
$(-\infty,\infty)$.
\item % (c)
Solve the initial value problem
$$
y''-2y'+y=0,\quad  y(0)=7,\quad y'(0)=4.
$$
\item % (d)
Solve the initial value problem
$$
y''-2y'+y=0,\quad  y(0)=k_0,\quad y'(0)=k_1.
$$
\end{enumerate}
\end{problem}

\begin{problem}\label{exer:5.1.4}
\begin{enumerate}
\item % (a)
Verify that $y_1=1/(x-1)$ and $y_2=1/(x+1)$ are solutions of
\begin{equation}\label{eq:eqA5.1.4}
    (x^2-1)y''+4xy'+2y=0
\end{equation}
on  $(-\infty,-1)$, $(-1,1)$, and $(1,\infty)$.
What is the general solution of \ref{eq:eqA5.1.4} on each of
these intervals?

\begin{solution}
    If $y_1=\frac{1}{ x-1}$, then $y_1'=-\frac{1}{(x-1)^2}$
and
$y_1''=\frac{2}{(x-1)^3}$, so
\begin{eqnarray*}
(x^2-1)y_1''+4xy_1'+2y_1&=&\frac{2(x^2-1)}{(x-1)^3}-\frac{4x}{(x-1)^2}
+\frac{2}{ x-1}\\
&=&\frac{2(x+1)-4x+2(x-1)}{(x-1)^2}=0.
\end{eqnarray*}
Similar manipulations show that
$(x^2-1)y_2''+4xy_2'+2y_2=0$. The general solution on each of
the intervals $(-\infty,-1)$, $(-1,1)$, and $(1,\infty)$ is
(B) $y=\frac{c_1}{ x-1}+\frac{c_2}{ x+1}$.
\end{solution}

\item % (b)
Solve the initial value problem
$$
(x^2-1)y''+4xy'+2y=0,\quad  y(0)=-5,\quad y'(0)=1.
$$
What is the interval of validity of the solution?

\begin{solution}
    Differentiating (B) yields (C) (see previous solution)
$y'=-\frac{c_1}{(x-1)^2}-\frac{c_2}{(x+1)^2}$.
 We must choose $c_1$ and $c_2$ in (B) so that $y(0)=-5$
and $y'(0)=1$. Setting $x=0$ in (B) and (C) shows that
$-c_1+c_2=-5,\ -c_1-c_2=1$. Therefore,$c_1=2$ and $c_2=-3$,
so $y=\frac{2}{ x-1}-\frac{3}{ x+1}$ on $(-1,1)$.
\end{solution}

\item % (c)
Graph the solution of the initial value problem.
\item % (d)
Verify Abel's formula for $y_1$ and $y_2$, with $x_0=0$.

\begin{solution}
    The Wronskian of  $\{y_1,y_2\}$ is
$$
W(x)=\left| \begin{array}{cc}
\phantom{-}\frac{1}{x-1} & \phantom{-}\frac{1}{x+1}\\
 -\frac{1}{(x-1)^2} &  - \frac{1}{(x+1)^2} \end{array} \right|
=\frac{2}{(x^2-1)^2},
\eqno{\rm (D)}
$$
so $W(0)=2$. Since
$p(x)=\frac{4x}{ x^2-1}$, so $\int_0^x p(t)\,dt=
\int_0^x \frac{4t}{ t^2-1}\,dt=\ln(x^2-1)^2$, Abel's
formula implies that
$W(x)=W(0)e^{-\ln(x^2-1)^2}=\frac{2}{(x^2-1)^2}$, consistent
with (D).
\end{solution}
\end{enumerate}
\end{problem}

\begin{problem}\label{exer:5.1.5}
Compute the Wronskians of the given sets of functions.

\begin{enumerate}
    \item $\{1, e^x\}$
    \item$\{e^x, e^x \sin x\}$
    \item $\{x+1,  x^2+2\}$
    \item $\{ x^{1/2}, x^{-1/3}\}$
    \item $\frac{\sin x}{x},  \frac{\cos x}{x}$
    \item $\{ x \ln|x|, x^2\ln|x|\}$
    \item $\{e^x\cos\sqrt x, e^x\sin\sqrt x\}$
\end{enumerate}
\end{problem}

\begin{problem}\label{exer:5.1.6}
Find the Wronskian  of a given set $\{y_1,y_2\}$ of solutions of
$$
y''+3(x^2+1)y'-2y=0,
$$
given that $W(\pi)=0$.

\begin{solution}
     From Abel's formula,
$W(x)=W(\pi)e^{-3\int_\pi^x (t^2+1)\,dt}=0\cdot
e^{-3\int_\pi^x (t^2+1)\,dt}=0$.
\end{solution}
\end{problem}

\begin{problem}\label{exer:5.1.7}
Find the Wronskian of a given set $\{y_1,y_2\}$ of solutions of
$$
(1-x^2)y''-2xy'+\alpha(\alpha+1)y=0,
$$
given that $W(0)=1$.  (This is
\href{http://www-history.mcs.st-and.ac.uk/PictDisplay/Legendre.html}{Legendre's equation}.)
\end{problem}

\begin{problem}\label{exer:5.1.8}
Find the Wronskian of a given set $\{y_1,y_2\}$ of solutions of
$$
x^2y''+xy'+(x^2-\nu^2)y=0 ,
$$
given that $W(1)=1$.  (This is
\href{http://www-history.mcs.st-and.ac.uk/Mathematicians/Bessel.html}
{Bessel's equation}.)

\begin{solution}
    $p(x)=\frac{1}{ x}$; therefore $\int_1^x
p(t)\,dt=\int_1^x\frac{dt}{ t}=\ln x$, so Abel's
formula yields $W(x)=W(1)e^{-\ln x}=\frac{1}{ x}$.
\end{solution}
\end{problem}

\begin{problem}\label{exer:5.1.9}  %\exerciseabel
(This exercise shows that if you know one
nontrivial solution of $y''+p(x)y'+q(x)y=0$, you can use Abel's
formula to find another.)

Suppose $p$ and $q$ are continuous and $y_1$ is a solution
of
\begin{equation}\label{eq:eqA5.1.9}
y''+p(x)y'+q(x)y=0
\end{equation}
that has no zeros  on  $(a,b)$.
Let $P(x)=\int p(x)\,dx$ be any
antiderivative of $p$ on  $(a,b)$.
\begin{enumerate}
\item % (a)
Show that if $K$ is an arbitrary nonzero constant and
$y_2$  satisfies
$$
y_1y_2'-y_1'y_2=Ke^{-P(x)}
$$
on $(a,b)$, then $y_2$ also satisfies \ref{eq:eqA5.1.9}
on $(a,b)$, and $\{y_1,y_2\}$ is a fundamental set of solutions on
(5) on $(a,b)$.

\item % (b)
Conclude from part (a) that if $y_2=uy_1$ where
$u'=K\frac{e^{-P(x)}}{y_1^2(x)}$, then  $\{y_1,y_2\}$
is a fundamental set of solutions of (A)
on  $(a,b)$.
\end{enumerate}
\end{problem}

\begin{problem}\label{exer:5.1.10}
For $y''-2y'-3y=0$; Use the
method suggested by Exercise~\ref{exer:5.1.9} to find a second solution
$y_2$  that isn't  a constant multiple of the  solution $y_1=e^{3x}$.

\begin{solution}
    $p(x)=-2$;\;
$P(x)=-2x$;\;
$y_2=uy_1=ue^{3x}$;\;
$u'=\frac{Ke^{-P(x)}}{ y_1^2(x)}=\frac{Ke^{2x}}{
e^{6x}}=Ke^{-4x}$;\;
$u=-\frac{K}{4}e^{-4x}$.
Choose $K=-4$;  then
$y_2=e^{-4x}e^{3x}=e^{-x}$.
\end{solution}
\end{problem}

\begin{problem}\label{exer:5.1.11}
For $y''-6y'+9y=0$; Use the
method suggested by Exercise~\ref{exer:5.1.9} to find a second solution
$y_2$  that isn't  a constant multiple of the  solution $y_1=e^{3x}$.
\end{problem}

\begin{problem}\label{exer:5.1.12}%
For $y''-2ay'+a^2y=0$\; ($a=$ constant); Use the
method suggested by Exercise~\ref{exer:5.1.9} to find a second solution
$y_2$  that isn't  a constant multiple of the  solution $y_1=e^{ax}$.

\begin{solution}
    $p(x)=-2a$;\;
$P(x)=-2ax$;\;
$y_2=uy_1=ue^{ax}$;\;
$u'=\frac{Ke^{-P(x)}}{ y_1^2(x)}=\frac{Ke^{2ax}}{ e^{2ax}}=K$;\;
$u=Kx$.
Choose $K=1$; then
$y_2=xe^{ax}$.
\end{solution}
\end{problem}


\begin{problem}\label{exer:5.1.13}
For $x^2y''+xy'-y=0$; Use the
method suggested by Exercise~\ref{exer:5.1.9} to find a second solution
$y_2$  that isn't  a constant multiple of the  solution $y_1=x$.
\end{problem}

\begin{problem}\label{exer:5.1.14}
For $x^2y''-xy'+y=0$; Use the
method suggested by Exercise~\ref{exer:5.1.9} to find a second solution
$y_2$  that isn't  a constant multiple of the  solution $y_1=x$.

\begin{solution}
    $p(x)=-\frac{1}{ x}$;\;
$P(x)=-\ln x$;\;
$y_2=uy_1=ux$;\;
$u'=\frac{Ke^{-P(x)}}{ y_1^2(x)}=\frac{Kx}{
x^2}=\frac{K}{ x}$;\;
$u=K\ln x$.
Choose $K=1$; then
$y_2=x\ln x$.
\end{solution}
\end{problem}


\begin{problem}\label{exer:5.1.15}
For $x^2y''-(2a-1)xy'+a^2y=0$\; ($a=$ nonzero constant);\, $x>0$; Use the
method suggested by Exercise~\ref{exer:5.1.9} to find a second solution
$y_2$  that isn't  a constant multiple of the  solution $y_1=x^a$.
\end{problem}

\begin{problem}\label{exer:5.1.16}
For $4x^2y''-4xy'+(3-16x^2)y=0$; Use the
method suggested by Exercise~\ref{exer:5.1.9} to find a second solution
$y_2$  that isn't  a constant multiple of the  solution $y_1=x^{1/2}e^{2x}$.

\begin{solution}
    $p(x)=-\frac{1}{ x}$;\;
$P(x)=-\ln|x|$;\;
$y_2=uy_1=ux^{1/2}e^{2x}$;\;
$u'=\frac{Ke^{-P(x)}}{ y_1^2(x)}=\frac{Kx}{ xe^{4x}}=e^{-4x}$;\;
$u=-\frac{Ke^{-4x}}{4}$.
Choose $K=-4$; then
$y_2=e^{-4x}(x^{1/2}e^{2x})=x^{1/2}e^{-2x}$.
\end{solution}
\end{problem}

\begin{problem}\label{exer:5.1.17}
For $(x-1)y''-xy'+y=0$; Use the
method suggested by Exercise~\ref{exer:5.1.9} to find a second solution
$y_2$  that isn't  a constant multiple of the  solution $y_1=e^x$.
\end{problem}

\begin{problem}\label{exer:5.1.18}
For $x^2y''-2xy'+(x^2+2)y=0$; Use the
method suggested by Exercise~\ref{exer:5.1.9} to find a second solution
$y_2$  that isn't  a constant multiple of the  solution $y_1=x\cos x$.

\begin{solution}
    $p(x)=-\frac{2}{ x}$;\;
$P(x)=-2\ln|x|$;\;
$y_2=uy_1=ux\cos x$;\;
$u'=\frac{Ke^{-P(x)}}{ y_1^2(x)}=\frac{Kx^2}{ x^2\cos^2
x}=K\sec^2x$;\;
$u=K\tan x$.
Choose $K=1$; then
$y_2=\tan x(x\cos x)=x\sin x$.
\end{solution}
\end{problem}

\begin{problem}\label{exer:5.1.19}
For $4x^2(\sin x)y''-4x(x\cos x+\sin x)y'+(2x\cos x+3\sin x)y=0$; Use the
method suggested by Exercise~\ref{exer:5.1.9} to find a second solution
$y_2$  that isn't  a constant multiple of the  solution $y_1=x^{1/2}$.
\end{problem}

\begin{problem}\label{exer:5.1.20}
For $(3x-1)y''-(3x+2)y'-(6x-8)y=0$; Use the
method suggested by Exercise~\ref{exer:5.1.9} to find a second solution
$y_2$  that isn't  a constant multiple of the  solution $y_1=e^{2x}$.

\begin{solution}
    $p(x)=-\frac{3x+2}{3x-1}=-1-\frac{3}{3x-1}$;\;
$P(x)=-x-\ln|3x-1|$;\;
$y_2=uy_1=ue^{2x}$;\;
$u'=\frac{Ke^{-P(x)}}{ y_1^2(x)}=\frac{K(3x-1)e^x}{
e^{4x}}=K(3x-1)e^{-3x}$;\;
$u=-Kxe^{-3x}$.
Choose $K=-1$; then
$y_2=xe^{-3x}e^{2x}=xe^{-x}$.
\end{solution}
\end{problem}


\begin{problem}\label{exer:5.1.21}
For $(x^2-4)y''+4xy'+2y=0$;  Use the
method suggested by Exercise~\ref{exer:5.1.9} to find a second solution
$y_2$  that isn't  a constant multiple of the  solution $y_1=\frac{1}{x-2}$.
\end{problem}


\begin{problem}\label{exer:5.1.22}
For $(2x+1)xy''-2(2x^2-1)y'-4(x+1)y=0$;  Use the
method suggested by Exercise~\ref{exer:5.1.9} to find a second solution
$y_2$  that isn't  a constant multiple of the  solution $y_1=\frac{1}{x}$.

\begin{solution}
    $p(x)=-\frac{2(2x^2-1)}{
x(2x+1)}=-2-\frac{2}{2x+1}+\frac{2}{ x}$;\;
$P(x)=-2x-\ln|2x+1|+2\ln|x|$;\;
$y_2=uy_1=\frac{u}{ x}$;\;
$u'=\frac{Ke^{-P(x)}}{ y_1^2(x)}=\frac{K(2x+1)e^{2x}}{
x^2}x^2=K(2x+1)e^{2x}$;\;
$u=Kxe^{2x}$.
Choose $K=1$; then
$y_2=\frac{xe^{2x}}{ x}=e^{2x}$.
\end{solution}
\end{problem}

\begin{problem}\label{exer:5.1.23}
For $(x^2-2x)y''+(2-x^2)y'+(2x-2)y=0$;  Use the
method suggested by Exercise~\ref{exer:5.1.9} to find a second solution
$y_2$  that isn't  a constant multiple of the  solution $y_1=e^x$.
\end{problem}


\begin{problem}\label{exer:5.1.24}
 Suppose  $p$ and $q$ are continuous on an open interval $(a,b)$
and let $x_0$ be in $(a,b)$.
Use   Theorem~\ref{thmtype:5.1.1} to show that
 the only solution of the initial value problem
$$
y''+p(x)y'+q(x)y=0,\quad  y(x_0)=0,\quad y'(x_0)=0
$$
on  $(a,b)$ is the trivial solution $y\equiv0$.

\begin{solution}
    Suppose that $y\equiv0$ on $(a,b)$. Then $y'\equiv0$ and $y''\equiv0$
on $(a,b)$, so $y$ is a solution of (A) $y''+p(x)y'+q(x)y=0,\
y(x_0)=0,\ y'(x_0)=0$ on $(a,b)$. Since Theorem \ref{thmtype:5.1.1}
implies that (A) has only one solution on $(a,b)$, the conclusion
follows.
\end{solution}
\end{problem}

\begin{problem}\label{exer:5.1.25}
Suppose $P_0$, $P_1$, and $P_2$ are continuous on  $(a,b)$
and let $x_0$ be in  $(a,b)$. Show that if either of
the following statements is true then $P_0(x)=0$ for some $x$ in
$(a,b)$.
\begin{enumerate}
\item % (a)
The initial value problem
$$
P_0(x)y''+P_1(x)y'+P_2(x)y=0,\quad  y(x_0)=k_0,\quad y'(x_0)=k_1
$$
has more than one  solution on  $(a,b)$.
\item % (b)
The initial value problem
$$
P_0(x)y''+P_1(x)y'+P_2(x)y=0,\quad y(x_0)=0,\quad y'(x_0)=0
$$
has a nontrivial solution on  $(a,b)$.
\end{enumerate}
\end{problem}

\begin{problem}\label{exer:5.1.26}
Suppose  $p$ and $q$ are continuous  on  $(a,b)$ and
 $y_1$ and $y_2$ are solutions of
\begin{equation}\label{eq:eqA5.1.26}
y''+p(x)y'+q(x)y=0
\end{equation}
on  $(a,b)$. Let
$$
z_1=\alpha y_1+\beta y_2\mbox{\quad  and \quad} z_2=\gamma y_1+\delta
y_2,
$$
where $\alpha$, $\beta$, $\gamma$, and $\delta$ are constants.
Show that if  $\{z_1,z_2\}$
is a fundamental set of solutions of
\ref{eq:eqA5.1.26} on $(a,b)$ then so is  $\{y_1,y_2\}$.

\begin{solution}
    If $\{z_1,z_2\}$ is a fundamental set of solutions of
(A) on $(a,b)$, then every solution $y$ of
(A) on $(a,b)$ is a linear combination of
$\{z_1,z_2\}$; that is, $y=c_1z_1+c_2z_2=c_1(\alpha y_1+\beta
y_2)+c_2(\gamma y_1+\delta y_2)
=(c_1\alpha+c_2\gamma)y_1+(c_1\beta+c_2\delta)y_2$, which shows that
every solution of (A) on $(a,b)$ can be written as
a linear combination of $\{y_1,y_2\}$. Therefore,$\{y_1,y_2\}$ is a
fundamental set of solutions of (A) on $(a,b)$.
\end{solution}
\end{problem}

\begin{problem}\label{exer:5.1.27}
Suppose $p$ and $q$ are continuous on  $(a,b)$ and
$\{y_1,y_2\}$ is a fundamental set of solutions of
\begin{equation}\label{eq:eqA5.1.27}
y''+p(x)y'+q(x)y=0
\end{equation}
on $(a,b)$. Let
$$
z_1=\alpha y_1+\beta y_2$$
and 
$$z_2=\gamma y_1+\delta
y_2,
$$
where $\alpha,\beta,\gamma$, and $\delta$ are constants.
 Show that  $\{z_1,z_2\}$ is a fundamental set of solutions
of \ref{eq:eqA5.1.27} on $(a,b)$ if and only if
$ \alpha\gamma-\beta\delta\ne0$.
\end{problem}

\begin{problem}\label{exer:5.1.28}
Suppose $y_1$ is differentiable on an interval  $(a,b)$
and $y_2=ky_1$, where $k$ is a constant. Show that the Wronskian of
$\{y_1,y_2\}$ is identically zero on  $(a,b)$.

\begin{solution}
    The Wronskian of  $\{y_1,y_2\}$ is
$$
W=\left| \begin{array}{cc}
y_1 & y_2 \\
y'_1 & y'_2
\end{array} \right|=
\left| \begin{array}{cc}
y_1 & ky_1 \\
y'_1 & ky'_1
\end{array} \right|=k(y_1y_1'-y_1'y_1)=0.
$$

nor $y_2$ can be a solution of $y''+p(x)y'+q(x)y=0$ on  $(a,b)$.
\end{solution}
\end{problem}

\begin{problem}\label{exer:5.1.29}
Let
$$
y_1=x^3\quad\mbox{ and }\quad y_2=\left\{\begin{array}{rl}
x^3,&x\ge 0,\\ -x^3,&x<0.\end{array}\right.
$$
\begin{enumerate}
\item % (a)
Show that the Wronskian of $\{y_1,y_2\}$ is defined and identically
zero on
$(-\infty,\infty)$.
\item % (b)
Suppose $a<0<b$.
Show that $\{y_1,y_2\}$ is linearly independent on  $(a,b)$.
\item % (b)
Use Exercise~\ref{exer:5.1.25} part (b) to show that these results don't
contradict Theorem~\ref{thmtype:5.1.5}, because neither $y_1$ nor $y_2$
can be a solution of an equation
$$
y''+p(x)y'+q(x)y=0
$$
on  $(a,b)$ if $p$ and $q$ are continuous on  $(a,b)$.
\end{enumerate}
\end{problem}

\begin{problem}\label{exer:5.1.30}
Suppose  $p$ and $q$ are continuous  on  $(a,b)$ and
$\{y_1,y_2\}$ is a  set of solutions of
$$
y''+p(x)y'+q(x)y=0
$$
on $(a,b)$ such that either $y_1(x_0)=y_2(x_0)=0$ or
$y_1'(x_0)=y_2'(x_0)=0$  for some $x_0$ in $(a,b)$. Show that
$\{y_1,y_2\}$ is linearly dependent on $(a,b)$.

\begin{solution}
    $W(x_0)=\left(y_1(x_0)y_2'(x_0)-y_1'(x_0)y_2(x_0)\right)=0$
if either $y_1(x_0)=y_2(x_0)=0$ or
$y_1'(x_0)=y_2'(x_0)=0$, and  Theorem \ref{thmtype:5.1.6}
implies that  $\{y_1,y_2\}$ is linearly dependent on  $(a,b)$.
\end{solution}
\end{problem}

\begin{problem}\label{exer:5.1.31}
Suppose  $p$ and $q$ are continuous  on  $(a,b)$ and
  $\{y_1,y_2\}$ is
a fundamental set of solutions of
$$
y''+p(x)y'+q(x)y=0
$$
on $(a,b)$. Show that if $y_1(x_1)=y_1(x_2)=0$, where $a<x_1<x_2<b$,
then $y_2(x)=0$ for some $x$ in $(x_1,x_2)$. 
\begin{hint}
    Show that
if $y_2$ has no zeros in $(x_1,x_2)$, then $y_1/y_2$ is either strictly
increasing or strictly decreasing on $(x_1,x_2)$, and deduce a
contradiction.
\end{hint}
\end{problem}

\begin{problem}\label{exer:5.1.32}
Suppose $p$ and $q$ are continuous on  $(a,b)$ and
 every solution of
\begin{equation}\label{eq:eqA5.1.32}
y''+p(x)y'+q(x)y=0
\end{equation}
on $(a,b)$ can be
written as a linear combination of the twice differentiable
functions $\{y_1,y_2\}$. Use Theorem~\ref{thmtype:5.1.1} to show that
$y_1$ and $y_2$ are themselves solutions of \ref{eq:eqA5.1.32} on
$(a,b)$.

\begin{solution}
    Let $x_0$ be an arbitrary point in  $(a,b)$. By the motivating
argument preceding  Theorem~5.1.4,
(B) $W(x_0)=y_1(x_0)y_2'(x_0)-y_1'(x_0)y_2(x_0)\ne0$. Now let $y$ be
the solution of $y''+p(x)y'+q(x)y=0,\ y(x_0)=y_1(x_0),\
y'(x_0)=y_1'(x_0)$. By assumption,  $y$ is a linear combination
of
 $\{y_1,y_2\}$ on  $(a,b)$; that is,  $y=c_1y_1+c_2y_2$, where
\begin{eqnarray*}
c_1y_1(x_0)+c_2y_2(x_0)&=&y_1(x_0)\\
c_1y_1'(x_0)+c_2y_2'(x_0)&=&y_1'(x_0).
\end{eqnarray*}
Solving this system by Cramers' rule yields
$$
c_1=\frac{1}{ W(x_0)}
\left| \begin{array}{cc}
y_1(x_0) & y_2(x_0) \\
y_1'(x_0) & y'_2(x_0)
\end{array} \right|=1
\mbox{ and }
c_2=\frac{1}{ W(x_0)}
\left| \begin{array}{cc}
y_1(x_0) & y_1(x_0)\\
y'_1(x_0) &y_1(x_0)
\end{array} \right|=0.
$$
Therefore,$y=y_1$, which
shows that $y_1$ is a solution of (A). A similar argument shows that
$y_2$ is a solution of (A).
\end{solution}
\end{problem}

\begin{problem}\label{exer:5.1.33}
Suppose  $p_1$, $p_2$, $q_1$, and $q_2$ are continuous on
$(a,b)$  and the equations
$$
y''+p_1(x)y'+q_1(x)y=0$$ 
and 
$$y''+p_2(x)y'+q_2(x)y=0
$$
have the same solutions on $(a,b)$. Show that $p_1=p_2$ and $q_1=q_2$
on $(a,b)$.
\begin{hint}
    Use Abel's formula.
\end{hint}
\end{problem}

\begin{problem}\label{exer:5.1.34}
(For this exercise you have to know about $3\times 3$
determinants.)
Show that if $y_1$ and $y_2$ are twice continuously differentiable on
$(a,b)$ and the Wronskian $W$ of $\{y_1,y_2\}$ has no zeros in
$(a,b)$ then the equation
$$\frac{1}{W}\begin{bmatrix}y & y_1 & y_2\\
y' & y'_1 & y'_2 \\
y'' & y_1'' & y_2''\end{bmatrix}=0$$
can be written as
\begin{equation}\label{eq:eqA5.1.34}
y''+p(x)y'+q(x)y=0,
\end{equation}
where $p$ and $q$ are continuous on $(a,b)$ and $\{y_1,y_2\}$ is a
fundamental set of solutions of (A) on $(a,b)$.
\begin{hint}
    Expand the determinant by cofactors of its first column.
\end{hint}

\begin{solution}
    Expanding  the determinant by cofactors of its first column shows that
the first equation in the exercise can be written as
$$
\frac{y}{ W}\left| \begin{array}{cc}
y_1' & y_2' \\
y_1'' & y_2''
\end{array} \right|-
\frac{y'}{ W}\left| \begin{array}{cc}
y_1 & y_2 \\
y_1'' & y_2''
\end{array} \right|+
\frac{y''}{ W}\left| \begin{array}{cc}
y_1 & y_2 \\
y'_1 & y'_2
\end{array} \right|=0,
$$
which is of the form (A) with
$$
p=-\frac{1}{ W}\left| \begin{array}{cc}
y_1 & y_2 \\
y_1'' & y_2''
\end{array} \right|
\mbox{\quad and \quad}
q=\frac{1}{ W}\left| \begin{array}{cc}
y_1' & y_2' \\
y_1'' & y_2''
\end{array} \right|.
$$
\end{solution}
\end{problem}

\begin{problem}\label{exer:5.1.35}
Use the method suggested by Exercise~\ref{exer:5.1.34} to find a linear
homogeneous equation for which the given functions form a fundamental
set of solutions on some interval.


\begin{enumerate}
\item $e^x \cos 2x, \quad e^x \sin 2x$
\item $x, \quad e^{2x}$
\item $x, \quad x \ln x$
\item $\cos (\ln x), \quad \sin (\ln x)$
\item $\cosh x, \quad \sinh x$
\item $ x^2-1, \quad x^2+1$
\end{enumerate}
\end{problem}

\begin{problem}\label{exer:5.1.36}
Suppose  $p$ and $q$ are continuous  on  $(a,b)$ and
  $\{y_1,y_2\}$ is
a fundamental set of solutions of
\begin{equation}\label{eq:eqA5.1.36}
y''+p(x)y'+q(x)y=0
\end{equation}
on $(a,b)$. Show that if $y$ is a solution of \ref{eq:eqA5.1.36}
on  $(a,b)$, there's exactly one way  to choose $c_1$ and $c_2$
so that $y=c_1y_1+c_2y_2$ on  $(a,b)$.

\begin{solution}
    Theorem \ref{thmtype:5.1.6} implies that
that there are constants $c_1$ and $c_2$ such that
(B) $y=c_1y_1+c_2y_2$
on $(a,b)$. To see that $c_1$ and $c_2$ are unique, assume that (B)
holds, and let $x_0$ be a point in  $(a,b)$. Then (C)
$y'=c_1y_1'+c_2y_2'$. Setting $x=x_0$ in (B) and (C) yields
$$
\begin{array}{rcc}
c_1y_1(x_0)+c_2y_2(x_0)&=&y(x_0)\\
c_1y_1'(x_0)+c_2y_2'(x_0)&=&y'(x_0).
\end{array}
$$
Since  Theorem~5.1.6  implies that
$y_1(x_0)y_2'(x_0)-y_1'(x_0)y_2(x_0)\ne0$, the argument preceding
 Theorem~5.1.4 implies that $c_1$ and $c_2$ are given
uniquely by
$$
c_1=\frac{y_2'(x_0)y(x_0)-y_2(x_0)y'(x_0)}{
y_1(x_0)y_2'(x_0)-y_1'(x_0)y_2(x_0)}\,\
c_2=\frac{y_1(x_0)y'(x_0)-y_1'(x_0)y(x_0)}{
y_1(x_0)y_2'(x_0)-y_1'(x_0)y_2(x_0)}\,.
$$
\end{solution}
\end{problem}

\begin{problem}\label{exer:5.1.37}
Suppose  $p$ and $q$ are continuous  on  $(a,b)$ and
$x_0$ is in  $(a,b)$. Let $y_1$ and $y_2$ be the solutions of
\begin{equation}\label{eq:eqA5.1.37}
y''+p(x)y'+q(x)y=0
\end{equation}
 such that
$$
y_1(x_0)=1, \quad y'_1(x_0)=0 \quad
y_2(x_0)=0,\quad  y'_2(x_0)=1.
$$
(Theorem~\ref{thmtype:5.1.1} implies that each of these initial value
problems
has a unique solution on $(a,b)$.)
\begin{enumerate}
\item % (a)
Show that  $\{y_1,y_2\}$ is linearly independent on  $(a,b)$.
\item % (b)
Show that an arbitrary solution  $y$  of
\ref{eq:eqA5.1.37} on  $(a,b)$ can be written as
$y=y(x_0)y_1+y'(x_0)y_2$.
\item\label{exer:5.1.37C} % (c)
Express the solution of the initial value problem
$$
y''+p(x)y'+q(x)y=0,\quad y(x_0)=k_0,\quad y'(x_0)=k_1
$$
as a linear combination of $y_1$ and $y_2$.
\end{enumerate}
\end{problem}

\begin{problem}\label{exer:5.1.38}
Find solutions $y_1$ and $y_2$ of the equation $y''=0$ that satisfy the
initial conditions
$$
y_1(x_0)=1, \quad y'_1(x_0)=0 \mbox{\quad and \quad} y_2(x_0)=0,
\quad y'_2(x_0)=1.
$$
Then use  Exercise~\ref{exer:5.1.37C} to write
the solution of  the initial value problem
$$
y''=0,\quad y(0)=k_0,\quad y'(0)=k_1
$$
as a linear combination of $y_1$ and $y_2$.

\begin{solution}
    The general solution of $y''=0$ is $y=c_1+c_2x$, so $y'=c_2$. Imposing
the stated initial conditions on $y_1=c_1+c_2x$  yields $c_1+c_2x_0=1$
and $c_2=0$; therefore $c_1=1$, so $y_1=1$.
Imposing the stated initial conditions on $y_2=c_1+c_2x$  yields
$c_1+c_2x_0=0$ and $c_2=1$; therefore $c_1=-x_0$, so $y_2=x-x_0$.
The solution of the general initial value problem is
$y=k_0+k_1(x-x_0)$.

\end{solution}
\end{problem}

\begin{problem}\label{exer:5.1.39}
Let $x_0$ be an arbitrary real number. Given (Example~\ref{example:5.1.1})
that $e^x$ and $e^{-x}$ are solutions of $y''-y=0$, find
solutions
$y_1$ and $y_2$ of $y''-y=0$ such that
$$
y_1(x_0)=1, \quad y'_1(x_0)=0, \quad  
y_2(x_0)=0,\quad  y'_2(x_0)=1.
$$
Then use  Exercise~\ref{exer:5.1.37C} to write
the solution of the initial value problem
$$
y''-y=0,\quad y(x_0)=k_0,\quad y'(x_0)=k_1
$$
as a linear combination of $y_1$ and $y_2$.
\end{problem}

\begin{problem}\label{exer:5.1.40}
Let $x_0$ be an arbitrary real number. Given (Example~\ref{example:5.1.2})
that $\cos\omega x$ and $\sin\omega x$ are solutions of
$y''+\omega^2y=0$, find solutions of
 $y''+\omega^2y=0$ such that
$$
y_1(x_0)=1, \quad y'_1(x_0)=0\mbox{\quad  and  \quad}
y_2(x_0)=0,\;  y'_2(x_0)=1.
$$
Then use Exercise~\ref{exer:5.1.37C} to write
the solution of the initial value problem
$$
y''+\omega^2y=0,\quad y(x_0)=k_0,\quad y'(x_0)=k_1
$$
as a linear combination of $y_1$ and $y_2$.
Use the identities
\begin{eqnarray*}
\cos(A+B)&=&\cos A\cos B-\sin A\sin B\\
\sin(A+B)&=&\sin A\cos B+\cos A\sin B
\end{eqnarray*}
to  simplify your expressions for $y_1$, $y_2$, and $y$.

\begin{solution}
    Let $y_1=a_1\cos\omega x+a_2\sin\omega x$ and
 $y_2=b_1\cos\omega x+b_2\sin\omega x$. Then
\begin{eqnarray*}
\phantom{(-}a_1\cos\omega x_0+a_2\sin\omega x_0\phantom{)}&=&1\\
\omega(-a_1\sin\omega x_0+a_2\cos\omega x_0)&=&0
\end{eqnarray*}
and
\begin{eqnarray*}
\phantom{(-}b_1\cos\omega x_0+b_2\sin\omega x_0\phantom{)}&=&0\\
\omega(-b_1\sin\omega x_0+b_2\cos\omega x_0)&=&1.
\end{eqnarray*}
Solving these systems yields $a_1=\cos\omega x_0$,
$a_2=\sin\omega x_0$, $b_1=-\frac{\sin\omega x_0}{\omega}$, and
$b_2=\frac{\cos\omega x_0}{\omega}$. Therefore,
$y_1=\cos\omega x_0\cos\omega x+\sin\omega x_0\sin\omega
x=\cos\omega(x-x_0)$ and
 $y_2=\frac{1}{\omega}(-\sin\omega x_0\cos\omega x+\cos\omega x_0
\sin\omega x)=\frac{1}{\omega}\sin\omega(x-x_0)$.
 The solution of the general
initial value problem is
$y=k_0\cos\omega(x-x_0)+\frac{k_1}{\omega}\sin\omega(x-x_0)$.
\end{solution}
\end{problem}

\begin{problem}\label{exer:5.1.41}
Recall from Exercise~\ref{exer:5.1.4} that
 $1/(x-1)$ and $1/(x+1)$ are solutions of
\begin{equation}\label{eq:eqA5.1.41}
(x^2-1)y''+4xy'+2y=0
\end{equation}
on $(-1,1)$.
Find solutions of \ref{eq:eqA5.1.41}
 such that
$$
y_1(0)=1, \quad y'_1(0)=0\quad  
y_2(0)=0,\quad  y'_2(0)=1.
$$
Then use  Exercise~\ref{exer:5.1.37C} to write
the solution of  initial value problem
$$
(x^2-1)y''+4xy'+2y=0,\quad  y(0)=k_0,\quad y'(0)=k_1
$$
as a linear combination of $y_1$ and $y_2$.
\end{problem}

\begin{problem}\label{exer:5.1.42}
\begin{enumerate}
\item % (a)
Verify  that $y_1=x^2$ and $y_2=x^3$ satisfy
\begin{equation}\label{eq:eqA5.1.42}
x^2y''-4xy'+6y=0
\end{equation}
on $(-\infty,\infty)$ and that $\{y_1,y_2\}$ is a fundamental set of
solutions of \ref{eq:eqA5.1.42} on $(-\infty,0)$ and
$(0,\infty)$.

\begin{solution}
    If $y_1=x^2$, then $y_1'=2x$ and $y_1''=2$, so
$x^2y_1''-4xy_1'+6y_1=x^2(2)-4x(2x)+6x^2=0$
for $x$ in $(-\infty,\infty)$.
 If $y_2=x^3$, then $y_2'=3x^2$ and $y_2''=6x$, so
$x^2y_2''-4xy_2'+6y_2=x^2(6x)-4x(3x^2)+6x^3=0$
for $x$ in $(-\infty,\infty)$. If $x\ne0$, then $y_2(x)/y_1(x)=x$,
which is nonconstant on $(-\infty,0)$ and $(0,\infty)$, so
Theorem \ref{thmtype:5.1.6} implies that  $\{y_1,y_2\}$ is a fundamental
set of solutions of (A) on each of these intervals.
\end{solution}
\item % (b)
Let $a_1$, $a_2$, $b_1$, and $b_2$ be constants. Show that
$$
y=\left\{\begin{array}{rr}
a_1x^2+a_2x^3,&x\ge 0,\\
b_1x^2+b_2x^3,&x<0\phantom{,}
\end{array}\right.
$$
is a solution of  \ref{eq:eqA5.1.42} on $(-\infty,\infty)$
if and only if $a_1=b_1$. From this, justify the statement that
$y$ is a  solution of \ref{eq:eqA5.1.42} on $(-\infty,\infty)$
if and only if
$$
y=\left\{\begin{array}{rr}
c_1x^2+c_2x^3,&x\ge 0,\\
c_1x^2+c_3x^3,&x<0,
\end{array}\right.
$$
where $c_1$, $c_2$, and $c_3$ are arbitrary constants.

\begin{solution}
    Theorem \ref{thmtype:5.1.6} and part (a) imply that $y$
satisfies (A) on $(-\infty,0)$ and on $(0,\infty)$ if and only if
$y=\left\{\begin{array}{rr}
a_1x^2+a_2x^3,&x> 0,\\
b_1x^2+b_2x^3,&x<0.
\end{array}\right.$
 Since $y(0)=0$ we can  complete the proof that $y$ is
a solution of
(A) on $(-\infty,\infty)$ by showing
that $y'(0)$ and $y''(0)$ both exist if and only if $a_1=b_1$. Since
$$
\frac{y(x)-y(0)}{
x-0}=\left\{\begin{array}{cl}a_1x+a_2x^2,&\mbox{
if }x>0,\\ b_1x+b_2x^2,&\mbox{ if }x<0,\end{array}\right.
$$
 it follows
that $y'(0)=\lim_{x\to0}\frac{y(x)-y(0)}{ x-0}=0$. Therefore,
$y'=\left\{\begin{array}{rr}
2a_1x+3a_2x^2,&x\ge 0,\\
2b_1x+3b_2x^2,&x<0.
\end{array}\right.$
Since
$\frac{y'(x)-y'(0)}{
x-0}=\left\{\begin{array}{cl}2a_1+3a_2x,&\mbox{
if }x>0,\\ 2b_1+3b_2x,&\mbox{ if }x<0,\end{array}\right.$ it
follows
that $y''(0)=\lim_{x\to0}\frac{y'(x)-y'(0)}{ x-0}$ exists
if and only if $a_1=b_1$.  By renaming $a_1=b_1=c_1$, $a_2=c_2$,
and $b_2=c_3$  we see that $y$ is a solution of
(A) on $(-\infty,\infty)$ if and only if
$y=\left\{\begin{array}{rr}
c_1x^2+c_2x^3,&x\ge 0,\\
c_1x^2+c_3x^3,&x<0.
\end{array}\right.$
\end{solution}
\item % (c)
 For what values of $k_0$ and  $k_1$
does the initial value problem
$$
x^2y''-4xy'+6y=0,\quad y(0)=k_0,\quad y'(0)=k_1
$$
have a solution? What are the solutions?
\begin{solution}
    We have shown that $y(0)=y'(0)=0$ for any choice
of $c_1$ and $c_2$ in (C). Therefore,the given initial value problem
has a solution if and only if $k_0=k_1=0$, in which case every
function of the form (C) is a solution.
\end{solution}

\item % (d)
 Show that if $x_0\ne0$ and $k_0,k_1$  are
arbitrary constants,  the initial value problem
\begin{equation}\label{eq:eqB5.1.42}
x^2y''-4xy'+6y=0,\quad y(x_0)=k_0,\quad y'(x_0)=k_1
\end{equation}
has infinitely many solutions on $(-\infty,\infty)$.  On what interval does
\ref{eq:eqB5.1.42} have a unique solution?

\begin{solution}
    If $x_0>0$, then $c_1$ and $c_2$ in (C) are uniquely
determined by $k_0$ and $k_1$, but $c_3$ can be chosen arbitrarily.
Therefore,(B) has a unique solution on
$(0,\infty)$, but infinitely many solutions on $(-\infty,\infty)$.
  If $x_0<0$, then $c_1$ and $c_3$ in (C) are uniquely
determined by $k_0$ and $k_1$, but $c_2$ can be chosen arbitrarily.
Therefore,(B) has a unique solution on
$(-\infty,0)$, but infinitely many solutions on $(-\infty,\infty)$.
\end{solution}
\end{enumerate}
\end{problem}

\begin{problem}\label{exer:5.1.43} %\exercisec
\begin{enumerate}
\item % (a)
Verify  that $y_1=x$ and $y_2=x^2$ satisfy
\begin{equation}\label{eq:eqA5.1.43}
x^2y''-2xy'+2y=0
\end{equation}
on $(-\infty,\infty)$ and that $\{y_1,y_2\}$ is a fundamental set of
solutions of \ref{eq:eqA5.1.43} on $(-\infty,0)$ and
$(0,\infty)$.
\item % (b)
Let $a_1$, $a_2$, $b_1$, and $b_2$ be constants. Show that
$$
y=\left\{\begin{array}{rr}
a_1x+a_2x^2,&x\ge 0,\\
b_1x+b_2x^2,&x<0\phantom{,}
\end{array}\right.
$$
is a solution of  \ref{eq:eqA5.1.43} on $(-\infty,\infty)$
if and only if $a_1=b_1$ and $a_2=b_2$. From this, justify the
statement that the
 general solution of \ref{eq:eqA5.1.43} on $(-\infty,\infty)$
is $y=c_1x+c_2x^2$, where $c_1$ and $c_2$ are arbitrary
constants.

\item % (c)
 For what values of $k_0$ and  $k_1$
does the initial value problem
$$
x^2y''-2xy'+2y=0,\quad  y(0)=k_0,\quad y'(0)=k_1
$$
have a solution? What are the solutions?

\item % (d)
 Show that if $x_0\ne0$ and $k_0,k_1$  are
arbitrary constants then the initial value problem
$$
x^2y''-2xy'+2y=0,\quad  y(x_0)=k_0,\quad y'(x_0)=k_1
$$
has a unique  solution on $(-\infty,\infty)$.
\end{enumerate}
\end{problem}

\begin{problem}\label{exer:5.1.44}
\begin{enumerate}
\item % (a)
Verify  that $y_1=x^3$ and $y_2=x^4$ satisfy
\begin{equation}\label{eq:eqA5.1.44}
x^2y''-6xy'+12y=0
\end{equation}
on $(-\infty,\infty)$, and that $\{y_1,y_2\}$ is a fundamental set of
solutions of \ref{eq:eqA5.1.44} on $(-\infty,0)$ and
$(0,\infty)$.

\begin{solution}
     If $y_1=x^3$, then $y_1'=3x^2$ and $y_1''=6x$, so
$x^2y_1''-6xy_1'+12y_1=x^2(6x)-6x(3x^2)+12x^3=0$
for $x$ in $(-\infty,\infty)$.
 If $y_2=x^4$, then $y_2'=4x^3$ and $y_2''=12x^2$, so
$x^2y_2''-6xy_2'+12y_2=x^2(12x^2)-6x(4x^3)+12x^4=0$
for $x$ in $(-\infty,\infty)$.
If $x\ne0$, then $y_2(x)/y_1(x)=x$, which
is nonconstant on $(-\infty,0)$ and $(0,\infty)$, so
Theorem \ref{thmtype:5.1.6} implies that  $\{y_1,y_2\}$ is a fundamental
set of solutions of (A) on each of these intervals.
\end{solution}
\item % (b)
Show that
$y$ is a solution of  \ref{eq:eqA5.1.44} on $(-\infty,\infty)$
if and only if
$$
y=\left\{\begin{array}{rr}
a_1x^3+a_2x^4,&x\ge 0,\\
b_1x^3+b_2x^4,&x<0,
\end{array}\right.
$$
where $a_1$, $a_2$, $b_1$, and $b_2$  are arbitrary constants.

\begin{solution}
    Theorem \ref{thmtype:5.1.2} and part (a) imply that $y$ satisfies (A) on
$(-\infty,0)$ and on $(0,\infty)$ if and only if
(C) $y=\left\{\begin{array}{rr}
a_1x^3+a_2x^4,&x> 0,\\
b_1x^3+b_2x^4,&x<0.
\end{array}\right.$
Since $y(0)=0$ we can  complete the proof that $y$ is
a solution of
(A) on $(-\infty,\infty)$ by showing
that $y'(0)$ and $y''(0)$ both exist for any choice of $a_1$, $a_2$,
$b_1$, and $b_2$. Since
$\frac{y(x)-y(0)}{
x-0}=\left\{\begin{array}{cl}a_1x^2+a_2x^3,&\mbox{
if }x>0,\\ b_1x^2+b_2x^3,&\mbox{ if }x<0,\end{array}\right.$ it
follows
that $y'(0)=\lim_{x\to0}\frac{y(x)-y(0)}{ x-0}=0$. Therefore,
$y'=\left\{\begin{array}{rr}
3a_1x^2+4a_2x^3,&x\ge 0,\\
3b_1x^2+4b_2x^3,&x<0.
\end{array}\right.$
Since
$\frac{y'(x)-y'(0)}{
x-0}=\left\{\begin{array}{cl}3a_1x+4a_2x^2,&\mbox{
if }x>0,\\ 3b_1x+4b_2x^2,&\mbox{ if }x<0,\end{array}\right.$ it
follows that $y''(0)=\lim_{x\to0}\frac{y'(x)-y'(0)}{ x-0}=0$.
Therefore,(B) is a solution of (A) on $(-\infty,\infty)$.
\end{solution}
\item % (c)
 For what values of $k_0$ and  $k_1$
does the initial value problem
$$
x^2y''-6xy'+12y=0, \quad  y(0)=k_0,\quad y'(0)=k_1
$$
have a solution? What are the solutions?

\begin{solution}
    We have shown that $y(0)=y'(0)=0$ for any choice
of $a_1$, $a_2$, $b_1$, and $b_2$ in (B). Therefore,the given initial
value
problem has a solution if and only if $k_0=k_1=0$, in which case every
function of the form (B) is a solution.
\end{solution}
\item % (d)
 Show that if $x_0\ne0$ and $k_0,k_1$  are
arbitrary constants then the initial value problem
\begin{equation}\label{eq:eqB5.1.44}
x^2y''-6xy'+12y=0, \quad  y(x_0)=k_0,\quad y'(x_0)=k_1
\end{equation}
has infinitely many solutions on $(-\infty,\infty)$.  On what interval does
\ref{eq:eqB5.1.44} have a unique solution?

\begin{solution}
    If $x_0>0$, then $a_1$ and $a_2$ in (B) are uniquely
determined by $k_0$ and $k_1$, but $b_1$ and $b_2$ can be chosen
arbitrarily. Therefore,(C) has a unique solution on
$(0,\infty)$, but infinitely many solutions on $(-\infty,\infty)$.
  If $x_0<0$, then $b_1$ and $b_2$ in (B) are uniquely
determined by $k_0$ and $k_1$, but $a_1$ and $a_2$ can be chosen
arbitrarily. Therefore,(C) has a unique solution on
$(-\infty,0)$, but infinitely many solutions on $(-\infty,\infty)$.

\end{solution}
\end{enumerate}
\end{problem}

\end{document}