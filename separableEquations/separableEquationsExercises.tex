\documentclass{ximera}
\input{../preamble.tex}

\title{Exercises} \license{CC BY-NC-SA 4.0}

\begin{document}

\begin{abstract}
\end{abstract}
\maketitle

\begin{onlineOnly}
\section*{Exercises}
\end{onlineOnly}


\begin{problem}\label{exer:2.2.1} Find all solutions.
$$y'=\frac{3x^2+2x+1}{y-2}$$
\end{problem}

\begin{problem}\label{exer:2.2.2} Find all solutions.
$$(\sin x)(\sin y)+(\cos y)y'=0$$

 

\begin{solution}
    By inspection, $y\equiv k\pi$ ($k=$integer) is a constant solution.
Separate variables to find others: $\left(\frac{\cos y}{\sin
y}\right)y'=-\sin x$;\ $\ln (|\sin y|)=\cos x+c$.
\end{solution}

\end{problem}

 \begin{problem}\label{exer:2.2.3} Find all solutions.
 $$xy'+y^2+y=0$$
 \end{problem}
 
\begin{problem}\label{exer:2.2.4} Find all solutions.
$$y' \ln |y|+x^2y= 0$$

 

\begin{solution}
     $y\equiv0$ is a  constant solution.
Separate variables to find others: $\left(\frac{\ln y}{y}\right)y'=-x^2$;\ $\frac{(\ln y)^2}{2}=-\frac{x^3}{3}+c$.
\end{solution}
\end{problem}

\begin{problem}\label{exer:2.2.5} Find all solutions.
$$(3y^3+3y \cos y+1)y'+\frac{(2x+1)y}{1+x^2}=0$$
\end{problem}

\begin{problem}\label{exer:2.2.6} Find all solutions.
$$x^2yy'=(y^2-1)^{3/2}$$

 

\begin{solution}
    $y\equiv1$ and $y\equiv-1$ are constant solutions.
For others, separate variables:
$(y^2-1)^{-3/2}yy'=\frac{1}{ x^2}$;\;
$-(y^2-1)^{-1/2}=-\frac{1}{ x}-c=-\left(\frac{1+cx}{ x}\right)$;\;
$(y^2-1)^{1/2}=\left(\frac{x}{1+cx}\right)$;\;
$(y^2-1)=\left(\frac{x}{1+cx}\right)^2$;\;
$y^2=1+\left(\frac{x}{1+cx}\right)^2$;\;
$y=\pm\left(1+\left(\frac{x}{1+cx}\right)^2\right)^{1/2}$.
\end{solution}
\end{problem}

\begin{problem}\label{exer:2.2.7} Find all solutions. Also, plot a direction field and some integral curves on the indicated rectangular region.
$$y'=x^2(1+y^2);   \quad \{-1\leq x\leq 1,\,-1\leq y\leq 1\}$$
\end{problem}

\begin{problem}\label{exer:2.2.8} Find all solutions. Also, plot a direction field and some integral curves on the indicated rectangular region.
 $$ y'(1+x^2)+xy=0 ;  \quad \{-2\leq x\leq 2,\, -1\leq
y\leq 1\}$$

 

\begin{solution}
    By inspection, $y\equiv0$ is a constant solution. Separate variables
to find others: $\frac{y'}{ y}=-\frac{x}{1+x^2}$;\;
$\ln|y|=-\frac{1}{2}\ln(1+x^2)+k$;\;
 $y=\frac{c}{\sqrt{1+x^2}}$,
which includes the constant solution $y\equiv0$.
\end{solution}
\end{problem}

\begin{problem}\label{exer:2.2.9} Find all solutions. Also, plot a direction field and some integral curves on the indicated rectangular region.
$$y'=(x-1)(y-1)(y-2);   \quad \{-2\leq x\leq 2,\,
-3\leq y\leq 3\}$$
\end{problem}

 \begin{problem}\label{exer:2.2.10} Find all solutions. Also, plot a direction field and some integral curves on the indicated rectangular region.
 $$(y-1)^2y'=2x+3;   \quad \{-2\leq x\leq 2,\, -2\leq
y\leq 5\}$$

Click below to see the answers.

\begin{solution}
    $(y-1)^2y'=2x+3$;\ $\frac{(y-1)^3}{3}=x^2+3x+c$;\;
$(y-1)^3=3x^2+9x+c$;\;
$y=1+\big(3x^2+9x+c)^{1/3}$.
\end{solution}
\end{problem}

\begin{problem}\label{exer:2.2.11} Solve the initial value problem.
$$y'=\frac{x^2+3x+2}{y-2}, \quad y(1)=4$$
\end{problem}

\begin{problem}\label{exer:2.2.12} Solve the initial value problem.
$$y'+x(y^2+y)=0, \quad y(2)=1$$

 

\begin{solution}
    $\frac{y'}{ y(y+1)}=-x$;\ $\left[\frac{1}{ y}-\frac{1}{
y+1}\right]y'=-x$;\ $\ln\left|\frac{y}{
y+1}\right|=-\frac{x^2}{2}+k$;\ $\frac{y}{
y+1}=ce^{-x^2/2}$;\ $y(2)=1\Rightarrow c=\frac{e^2}{2}$;\;
$y=(y+1)ce^{-x^2/2}$;\ $y(1-ce^{-x^2/2})=ce^{-x^2/2}$;\;
$y=\frac{ce^{-x^2/2}}{1-ce^{-x^2}/2}$;\ setting
$c=\frac{e^2}{2}$ yields
 $y=\frac{e^{-(x^2-4)/2}
}{2-e^{-(x^2-4)/2}}$.
\end{solution}
\end{problem}

\begin{problem}\label{exer:2.2.13} Solve the initial value problem and graph the solution.
$$(3y^2+4y)y'+2x+\cos x=0, \quad y(0)=1$$
\end{problem}

\begin{problem}\label{exer:2.2.14} Solve the initial value problem and graph the solution.
$$y'+\frac{(y+1)(y-1)(y-2)}{x+1}=0, \quad y(1)=0$$

 

\begin{solution}
    $\frac{y'}{(y+1)(y-1)(y-2)}=-\frac{1}{ x+1}$;\;
$\left[\frac{1}{6}\frac{1}{ y+1}-\frac{1}{2}\frac{1}{
y-1}+\frac{1}{3}\frac{1}{ y-2}\right]y'=-\frac{1}{ x+1}$;\;
$\left[\frac{1}{ y+1}-\frac{3}{ y-1}+\frac{2}{
y-2}\right]y'=-\frac{6}{ x+1}$;\;
$\ln|y+1|-3\ln|y-1|+2\ln|y-2|=-6\ln|x+1|+k$;\;
$\frac{(y+1)(y-2)^2}{(y-1)^3}=\frac{c}{(x+1)^6}$;\;
$y(1)=0\Rightarrow c=-256$;\;
$\frac{(y+1)(y-2)^2}{(y-1)^3}=-\frac{256}{(x+1)^6}$.
\end{solution}
\end{problem}

\begin{problem}\label{exer:2.2.15} Solve the initial value problem and graph the solution.
$$y'+2x(y+1)=0, \quad y(0)=2$$
\end{problem}

\begin{problem}\label{exer:2.2.16} Solve the initial value problem and graph the solution.
$$y'=2xy(1+y^2),\quad y(0)=1$$

 

\begin{solution}
    $\frac{y'}{ y(1+y^2)}=2x$;\ $\left[\frac{1}{ y}-\frac{y}{
y^2+1}\right]y'=2x$;\ $\ln\left(\frac{|y|}{
\sqrt{y^2+1}}\right)=x^2+k$;\ $\frac{y}{
\sqrt{y^2+1}}=ce^{x^2}$;\ $y(0)=1\Rightarrow c=\frac{1}{\sqrt2}$;\;
$\frac{y}{\sqrt{y^2+1}}=\frac{e^{x^2}}{\sqrt2}$;\;
$2y^2=(y^2+1)e^{x^2}$;\ $y^2(2-e^{x^2})=e^{2x^2}$;
$y=\frac{1}{\sqrt{2e^{-2x^2}-1}}$.
\end{solution}
\end{problem}

\begin{problem}\label{exer:2.2.17} Solve the initial value problem and find the interval of validity of the solution.
$$y'(x^2+2)+ 4x(y^2+2y+1)=0, \quad y(1)=-1$$
\end{problem}

\begin{problem}\label{exer:2.2.18} Solve the initial value problem and find the interval of validity of the solution.
$$y'=-2x(y^2-3y+2), \quad y(0)=3$$

 

\begin{solution}
    $\frac{y'}{(y-1)(y-2)}=-2x$;\ $\left[\frac{1}{
y-2}-\frac{1}{ y-1}\right]y'=-2x$;\ $\ln\left|\frac{y-2}{
y-1}\right|=-x^2+k$;\ $\frac{y-2}{ y-1}=ce^{-x^2}$;\;
$y(0)=3\Rightarrow c=\frac{1}{2}$;\ $\frac{y-2}{
y-1}=\frac{e^{-x^2}}{2}$;\ $y-2=\frac{e^{-x^2}}{2}(y-1)$;\;
$y\left(1-\frac{e^{-x^2}}{2}\right)=2-\frac{e^{-x^2}}{2}$;\;
$y=\frac{4-e^{-x^2}}{2-e^{-x^2}}$. \
The interval of validity is $(-\infty,\infty)$.
\end{solution}
\end{problem}

\begin{problem}\label{exer:2.2.19} Solve the initial value problem and find the interval of validity of the solution.
$$y'=\frac{2x}{1+2y},\quad y(2)=0$$ 
\end{problem}

\begin{problem}\label{exer:2.2.20} Solve the initial value problem and find the interval of validity of the solution.
$$y'=2y-y^2,\quad y(0)=1$$ 

 

\begin{solution}
    $\frac{y'}{ y(y-2)}=-1$;\ $\frac{1}{2}\left[\frac{1}{
y-2}-\frac{1}{ y}\right]y'=-1$;\;
 $\left[\frac{1}{
y-2}-\frac{1}{ y}\right]y'=-2$;\ $\ln\left|\frac{y-2}{
y}\right|=-2x+k$;\ $\frac{y-2}{ y}=ce^{-2x}$;\;
$y(0)=1\Rightarrow c=-1$;\ $\frac{y-2}{ y}=-e^{-2x}$;\;
$y-2=-ye^{-2x}$;\ $y(1+e^{-2x})=2$;\;
   $y=\frac{2}{
1+e^{-2x}}$. The interval of validity is $(-\infty,\infty)$.
\end{solution}
\end{problem}

\begin{problem}\label{exer:2.2.21} Solve the initial value problem and find the interval of validity of the solution.
$$x+yy'=0, \quad y(3) =-4$$
\end{problem}

\begin{problem}\label{exer:2.2.22} Solve the initial value problem and find the interval of validity of the solution.
$$y'+x^2(y+1)(y-2)^2=0, \quad y(4)=2$$

 

\begin{solution}
    $y\equiv2$ is a constant solution of the differential equation, and
it satisfies the initial condition. Therefore, $y\equiv2$ is a
solution of the initial value problem. The interval of validity is
$(-\infty,\infty)$.
\end{solution}
\end{problem}

\begin{problem}\label{exer:2.2.23} Solve the initial value problem and find the interval of validity of the solution.
$$(x+1)(x-2)y'+y=0, \quad y(1)=-3$$
\end{problem}

\begin{problem}\label{exer:2.2.24} Solve $y'=\frac{(1+y^2)}{(1+x^2)}$ explicitly.
\begin{hint}Use the identity
$\tan(A+B)=\frac{\tan A+\tan B}{1-\tan A\tan B}$.
\end{hint}

 

\begin{solution}
    $\frac{y'}{1+y^2}=\frac{1}{1+x^2}$;\ $\tan^{-1}y=
\tan^{-1}x+k$;\ $y=\tan(\tan^{-1}x+k)$. Now use
the identity $\tan(A+B)=\frac{\tan A+\tan B}{1-\tan A\tan B}$ with
$A=\tan^{-1}x$ and $B=\tan^{-1}c$ to rewrite $y$ as
$y=\frac{x+c}{1-cx}$, where $c=\tan k$.
\end{solution}
\end{problem}

\begin{problem}\label{exer:2.2.25} Solve $y'\sqrt{1-x^2}+\sqrt{1-y^2}=0$ explicitly.
\begin{hint}Use the identity
$\sin(A-B)=\sin A\cos B-\cos A\sin B$.
\end{hint}
\end{problem}

\begin{problem}\label{exer:2.2.26}
Solve $y'=\frac{\cos x}{\sin y},\quad y (\pi)=\frac{\pi}{2}$
explicitly. 
\begin{hint}Use the identity $\cos(x+\pi/2)=-\sin x$ and the
periodicity of the cosine.
\end{hint}

 

\begin{solution}
    $(\sin y)y'=\cos x$;\ $-\cos
y=\sin x+c$;\  $y(\pi)=\frac{\pi}{2}\Rightarrow
c=0$, so $$\text{(A)}\quad  \cos y=-\sin
x$$ To obtain $y$ explicity we  note that $-\sin x=\cos(x+\pi/2)$, so
(A) can be rewritten as  $\cos y=\cos(x+\pi/2)$. This equation
holds if an only if one of the following conditions holds for some
integer $k$:
$$
\text{(B)}\quad y=x+\frac{\pi}{2}+2k\pi;\quad
\text{(C)}\quad
y=-x-\frac{\pi}{2}+2k\pi.
$$
Among these choices the only way to satisfy the initial condition
is to let $k=1$ in (C), so $y=-x+\frac{3\pi}{2}.$
\end{solution}
\end{problem}

\begin{problem}\label{exer:2.2.27}
Solve the initial value problem
$$
y'=ay-by^2,\quad y(0)=y_0.
$$
Discuss the behavior  of the solution if
\begin{enumerate}
    \item $y_0\geq 0$;
    \item $y_0<0$.
\end{enumerate}
\end{problem}

\begin{problem}\label{exer:2.2.28}
 The population $P=P(t)$ of a  species satisfies
the logistic equation
$$
P'=aP(1-\alpha P)
$$
 and  $P(0)=P_0>0$.  Find $P$  for $t>0$, and  find
$\lim_{t\to\infty}P(t)$.

 

\begin{solution}
    Rewrite the equation as $P'=-a\alpha P(P-1/\alpha)$.
By inspection, $P\equiv0$ and $P\equiv1/\alpha$ are constant
solutions.
Separate variables to find others: $\frac{P'}{
P(P-1/\alpha)}=-a\alpha$;\;
 $\left[\frac{1}{ P-1/\alpha}-\frac{1}{ P}\right]P'=-a
$;\ $\ln\left|\frac{P-1/\alpha}{ P}\right|=-at+k$;\;
(A) $\frac{P-1/\alpha}{
P}=ce^{-\alpha t}$; $P(1-ce^{-\alpha t})=1/\alpha$;\;
(B) $P=\frac{1}{\alpha(1-ce^{-\alpha t})}$. From (A),
 $P(0)=P_0\Rightarrow c=\frac{P_0-1/\alpha}{ P_0}$.
Substituting this into (B) yields
$P=\frac{P_0}{\alpha
P_0+(1-\alpha P_0)e^{-at}}$. From this
$\lim_{t\to\infty}P(t)=1/\alpha$.
\end{solution}
\end{problem}

\begin{problem}\label{exer:2.2.29}
An epidemic spreads through a population at a rate
proportional to the product of the number of people already infected and the number of people susceptible, but not yet infected. Therefore, if $S$ denotes the total population of susceptible people and $I=I(t)$ denotes the number of infected people at time $t$, then
$$
I'=rI(S-I),
$$
where $r$ is a positive constant. Assuming that $I(0)=I_0$,
find $I(t)$ for $t>0$, and show that
$\lim_{t\to\infty}I(t)=S$.
\end{problem}

\begin{problem}\label{exer:2.2.30}
The result of Exercise~\ref{exer:2.2.29} is discouraging: if any susceptible member of the group is initially infected, then in the long run all susceptible members are infected!
 On a more hopeful note, suppose   the disease spreads
according to the model of Exercise~\ref{exer:2.2.29}, but there's a
medication that cures the infected population at a rate proportional
to the number of infected individuals. Now the equation for the number
of infected individuals becomes
\begin{equation}\label{eqA:2.2.30}
    I'=rI(S-I)-qI
\end{equation}

where $q$ is a positive constant.
\begin{enumerate}
\item\label{partA:2.2.30} % (a)
Choose  $r$ and $S$ positive.
By plotting  direction fields and solutions of (\ref{eqA:2.2.30}) on
suitable rectangular grids
$$
R=\{0\leq t \leq T,\ 0\leq I \leq d\}
$$
in the $(t,I)$-plane, verify that if $I$ is any solution of
(\ref{eqA:2.2.30}) such that $I(0)>0$, then $\lim_{t\to\infty}I(t)=S-q/r$
if $q<rS$ and $\lim_{t\to\infty}I(t)=0$ if $q\geq rS$.

\item % (b)
To verify the experimental results of \ref{partA:2.2.30}, use separation of
variables to solve (\ref{eqA:2.2.30}) with initial condition $I(0)=I_0>0$, and find
$\lim_{t\to\infty}I(t)$. 
\begin{hint}There are three cases to consider:
\begin{enumerate}
    \item $q<rS$;
    \item $q>rS$; 
    \item $q=rS$.
\end{enumerate}     
\end{hint}

 

\begin{solution}
    If $q=rS$ the equation for $I$ reduces to $I'=-rI^2$, so
$\frac{I'}{ I^2}=-r$; $-\frac{1}{ I}=-rt-\frac{1}{ I_0}$;
so $I=\frac{I_0}{1+rI_0t}$ and $\lim_{t\to\infty}I(t)=0$. If $q\ne
rS$, then rewrite the equation for $I$ as $I'=-rI(I-\alpha)$ with
$\alpha=S-\frac{q}{ r}$. Separating variables yields
$\frac{I'}{ I(I-\alpha)}=-r$;\;
 $\left[\frac{1}{ I-\alpha}-\frac{1}{ I}\right]I'=-r\alpha
$;\ $\ln\left|\frac{I-\alpha}{ I}\right|=-r\alpha t+k$;\;
(A) $\frac{I-\alpha}{
I}=ce^{-r\alpha t}$; $I(1-ce^{-r\alpha t})=\alpha$;\;
(B) $I=\frac{\alpha}{ 1-ce^{-r\alpha t}}$. From (A),
 $I(0)=I_0\Rightarrow c=\frac{I_0-\alpha}{ I_0}$.
Substituting this into (B) yields
$I=\frac{\alpha I_0}{ I_0+(\alpha-I_0)e^{-r\alpha t}}$.
If $q<rS$, then $\alpha>0$ and
$\lim_{t\to\infty}I(t)=\alpha=S-\frac{q}{ r}$.
If $q>rS$, then $\alpha<0$ and $\lim_{t\to\infty}I(t)=0$.

\end{solution}
\end{enumerate}
\end{problem}

\begin{problem}\label{exer:2.2.31}
Consider the differential equation
\begin{equation}\label{eqA:2.2.31}
    y'=ay-by^2-q
\end{equation}

where $a$, $b$ are positive constants, and $q$ is an arbitrary
constant. Suppose  $y$ denotes a solution of this equation
that satisfies the initial condition $y(0)=y_0$.

\begin{enumerate}
\item % (a)
Choose  $a$ and $b$ positive and  $q<a^2/4b$.
By plotting  direction fields and solutions of (\ref{eqA:2.2.31}) on
suitable rectangular grids
\begin{equation}\label{eqB:2.2.31}
    R=\{0\leq t \leq T,\ c\leq y \leq d\}
\end{equation}
in the $(t,y)$-plane,  discover that there are numbers $y_1$ and
$y_2$ with $y_1<y_2$ such that if $y_0>y_1$ then
$\lim_{t\to\infty}y(t)=y_2$, and if $y_0<y_1$ then $y(t)=-\infty$
for some finite value of~$t$. (What happens if $y_0=y_1$?)

\item % (b)
Choose $a$ and $b$ positive and $q=a^2/4b$. By plotting direction
fields
and solutions of (\ref{eqA:2.2.31}) on suitable rectangular grids of the form (\ref{eqB:2.2.31}),
discover that there's a number $y_1$ such that if $y_0\geq y_1$ then
$\lim_{t\to\infty}y(t)=y_1$, while if $y_0<y_1$ then  $y(t)=-\infty$
for some finite value of $t$.

\item % (c)
Choose positive $a$, $b$ and $q>a^2/4b$. By plotting direction fields
and solutions of (\ref{eqA:2.2.31}) on suitable rectangular grids of the form (\ref{eqB:2.2.31}),
discover that
no matter what $y_0$ is, $y(t)=-\infty$ for some finite value of $t$.

\item % (d)
Verify your results experiments analytically.
Start by separating variables in (\ref{eqA:2.2.31})  to obtain
$$
\frac{y'}{ay-by^2-q}=1.
$$
To decide what to do next you'll have to use the
quadratic formula. This should lead you to see why there are
three cases.  Take it from there!

Because of its role in the transition between these
three cases,  $q_0=a^2/4b$ is called a \emph{bifurcation value}
of $q$. In general, if $q$ is a parameter in any differential
equation,  $q_0$ is said to be a bifurcation value of
$q$ if the nature of the solutions of the equation with $q<q_0$
is qualitatively different from the nature of the solutions
with $q>q_0$.
\end{enumerate}
\end{problem}

\begin{problem}\label{exer:2.2.32}
By plotting direction fields and solutions of
$$
y'=qy-y^3,
$$
convince yourself that $q_0=0$ is a bifurcation value of $q$
for this equation. Explain what makes you draw this conclusion.

\end{problem}

\begin{problem}\label{exer:2.2.33}
Suppose a disease spreads according to the model of
Exercise~\ref{exer:2.2.29}, but
there's a medication that cures the infected population at a constant
rate of $q$ individuals per unit  time, where $q>0$.
Then
the equation for the number of infected individuals becomes
$$
I'=rI(S-I)-q.
$$
Assuming that $I(0)=I_0>0$, use the results of
Exercise~\ref{exer:2.2.31} to describe what happens as $t\to\infty$.
\end{problem}

\begin{problem}\label{exer:2.2.34}
Assuming that $p \not\equiv 0$, state conditions under which the linear
equation
$$
y'+p(x)y=f(x)
$$
is separable.  If the equation satisfies these conditions, solve it by
separation of variables and by the method developed in
the previous section.

Click below to see the solution.

\begin{solution}
    The given equation is separable if $f=ap$, where $a$ is a constant.
In this case the equation is
$$
\text{(A)}\quad y'+p(x)y=ap(x).
$$
Let $P$ be an antiderivative  of $p$; that is, $P'=p$.

\textit{Solution by Separation of Variables.} $y'=-p(x)(y-a)$;\;
$\frac{y'}{ y-a}=-p(x)$;\ $\ln|y-a|=-P(x)+k$;\;
$y-a=ce^{-P(x)}$;\ $y=a+ce^{-P(x)}$.

\textit{Solution by Variation of Parameters.} $y_1=e^{-P(x)}$
is a solution of the complementary equation, so solutions of
(A) are of the form $y=ue^{-P(x)}$ where
$u'e^{-P(x)}=ap(x)$. Hence, $u'=ap(x)e^{P(x)}$;\;
$u=ae^{P(x)}+c$;\ $y=a+ce^{-P(x)}$.
\end{solution}
\end{problem}

 \begin{problem}\label{exer:2.2.35} Solve the equation using variation of parameters followed by separation of variables.
 $$y'+y=\frac{2xe^{-x}}{1+ye^x}$$
 \end{problem}

\begin{problem}\label{exer:2.2.36} Solve the equation using variation of parameters followed by separation of variables.
$$xy'-2y=\frac{x^6}{y+x^2}$$

 

\begin{solution}
    Rewrite the given equation as
(A) $y'-\frac{2}{ x}y=\frac{x^5}{ y+x^2}$.
 $y_1=x^2$ is a solution of $y'-\frac{2}{ x}y=0$.
Look for solutions of (A)  of the form $y=ux^2$.
Then $u'x^2=\frac{x^5}{(u+1)x^2}=\frac{x^3}{ u+1}$;\;
$u'=\frac{x}{ u+1}$;\;
$(u+1)u'=x$;\;
$\frac{(1+u)^2}{2}=\frac{x^2}{2}+\frac{c}{2}$;\;
$u=-1\pm\sqrt{x^2+c}$;\ $y=x^2\left(-1\pm\sqrt{x^2+c}\right)$.
\end{solution}
\end{problem}

 \begin{problem}\label{exer:2.2.37} Solve the equation using variation of parameters followed by separation of variables.
 $$y'-y=\frac{(x+1)e^{4x}}{(y+e^x)^2}$$
\end{problem}
 
\begin{problem}\label{exer:2.2.38} Solve the equation using variation of parameters followed by separation of variables.
$$y'-2y=\frac{xe^{2x}}{1-ye^{-2x}}$$

 

\begin{solution}
    $y_1=e^{2x}$ is a solution of $y'-2y=0$.
Look for solutions of the nonlinear equation of the form $y=ue^{2x}$.
Then $u'e^{2x}=\frac{xe^{2x}}{1-u}$;\ $u'=\frac{x}{1-u}$;\;
$(1-u)u'=x$;\;
$-\frac{(1-u)^2}{2}=\frac{1}{2}(x^2-c)$;\;
$u=1\pm\sqrt{c-x^2}$;\;
$y=e^{2x}\left(1\pm\sqrt{c-x^2}\right)$.
\end{solution}
\end{problem}


\begin{problem}\label{exer:2.2.39}
Use variation of parameters to show that the solutions of the
following equations are of the form $y=uy_1$, where $u$ satisfies a
separable equation $u'=g(x)p(u)$. Find $y_1$ and $g$ for each
equation.

\begin{enumerate}
    \item 
$xy'+y=h(x)p(xy)$
    \item 
    $xy'-y=h(x) p\left(\frac{y}{x}\right)$
    \item 
    $y'+y=h(x) p(e^xy)$
    \item 
    $xy'+ry=h(x) p(x^ry)$
    \item 
    $y'+\frac{v'(x)}{v(x)}y= h(x)p\left(v(x)y\right)$
\end{enumerate}
\end{problem}

\end{document}