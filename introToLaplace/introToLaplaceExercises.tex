\documentclass{ximera}
\input{../preamble.tex}

\title{Exercises} \license{CC BY-NC-SA 4.0}

\begin{document}

\begin{abstract}
\end{abstract}
\maketitle

\begin{onlineOnly}
\section*{Exercises}
\end{onlineOnly}


\begin{problem}\label{exer:8.1.1}
Find the Laplace transforms of the following functions by evaluating
the integral $F(s)=\int_0^\infty e^{-st} f(t)\,dt$.

\begin{enumerate}
    \item $t$
    
    \begin{solution}
        
    \end{solution}
    
    \item $te^{-t}$ 
    \item $\sinh bt$
    \item $e^{2t}-3e^t$
    \item $t^2$
\end{enumerate}
\end{problem}

\begin{problem}\label{exer:8.1.2}
Use  the table of Laplace transforms to
find the Laplace transforms of the following functions.

\begin{enumerate}
    \item $\cosh t\sin t$

    \begin{solution}
$\cosh t\sin t=\frac{1}{2}\left(e^t\sin t+e^{-t}\sin t\right)
\leftrightarrow\frac{1}{2}\left[\frac{1}{(s-1)^2+1}+\frac{1}{(s+1)^2+1}\right]
=\frac{s^2+2}{[(s-1)^2+1][(s+1)^2+1]}$.
    \end{solution}
    
    \item $\sin^2t$
    
    \begin{solution}
$
\sin^2t=\frac{1-\cos2t}{2}\leftrightarrow\frac{1}{2}\left[\frac{1}{s}
-\frac{s}{(s^2+4)}\right]=\frac{2}{s(s^2+4)}$.
    \end{solution}
    
    \item $\cos^2 2t$
        
    \begin{solution}
$\cos^22t=\frac{1}{2}\left[\frac{1}{s}+\frac{s}{s^2+16}\right]=
\frac{s^2+8}{s(s^2+16)}$.
    \end{solution}
    
    \item $\cosh^2 t$
        
    \begin{solution}
$\cosh^2
t=\frac{(e^t+e^{-t})^2)}{4}=\frac{(e^{2t}+2+e^{-2t})}{4}
\leftrightarrow\frac{1}{4}\left(\frac{1}{s-2}+\frac{2}{s}+\frac{1}{s+2}\right) =\frac{s^2-2}{s(s^2-4)}$.
\end{solution}
    
    \item $t\sinh 2t$
        
    \begin{solution}
$t\sinh2t=\frac{te^{2t}-te^{-2t}}{2}
\leftrightarrow\frac{1}{2}\left(\frac{1}{(s-2)^2}-\frac{1}{(s+2)^2}\right)
=\frac{4s}{(s^2-4)^2}$.
    \end{solution}
    
    \item $\sin t\cos t$
        
    \begin{solution}
$\sin t\cos t=\frac{\sin2t}{2}\leftrightarrow\frac{1}{s^2+4}$.
    \end{solution}
    
    \item $\sin\left(t+\frac{\pi}{4}\right)$
        
    \begin{solution}
$\sin(t+\pi/4)=\sin
t\cos(\pi/4)+\cos
t\cos(\pi/4)\leftrightarrow\frac{1}{\sqrt{2}}\,\frac{s+1}{s^2+1}$.
    \end{solution}
    
    \item $\cos 2t -\cos 3t$
        
    \begin{solution}
$\cos 2t -\cos 3t\leftrightarrow\frac{s}{s^2+4}-\frac{s}{s^2+9}=\frac{5s}{(s^2+4)(s^2+9)}$.
    \end{solution}
    
    \item $\sin 2t +\cos 4t$
        
    \begin{solution}
$\sin 2t +\cos4t\leftrightarrow\frac{2}{s^2+4}+\frac{s}{s^2+16} =\frac{s^3+2s^2+4s+32}{(s^2+4)(s^2+16)}$.
    \end{solution}
    
\end{enumerate}
\end{problem}

\begin{problem}\label{exer:8.1.3}
    Show that

    $$
\int_0^\infty e^{-st}e^{t^2} dt=\infty
$$
for every real number $s$.
\end{problem}

\begin{problem}\label{exer:8.1.4}
 Graph the following piecewise continuous functions and evaluate
$f(t+)$, $f(t-)$, and $f(t)$ at each point of discontinuity.

\begin{enumerate}
    \item $f(t)=\left\{\begin{array}{cl} -t,
& 0\le t<2,\\ t-4, & 2\le t<3,\\ 1, & t\ge
3.\end{array}\right.$
    \item $f(t)=\left\{\begin{array}{cl} t^2+2, & 0
\le t<1,\\4, & t=1,\\ t, & t>
1.\end{array}\right.$
    \item $f(t)=\left\{\begin{array}{rl}
\sin t, & 0\le t<\pi/ 2,\\ 2\sin t, &\pi/ 2
\le t<\pi,\\ \cos t, & t\ge\pi.\end{array}\right.$
    \item $f(t)=\left\{\begin{array}{cl}t,
 & 0\le t<1,\\ 2, & t=1,\\ 2-t, & 1
\le t<2,\\ 3, & t=2,\\ 6, & t>
2.\end{array}\right.$
\end{enumerate}
\end{problem}

\begin{problem}\label{exer:8.1.5}
 Find the Laplace transform:

\begin{enumerate}
    \item $f(t)=\left\{\begin{array}{rl} e^{-t}, &
0\le t<1,\\ e^{-2t}, & t\ge 1.\end{array}\right.$ 
    \item $f(t)=\left\{\begin{array}{rl} 1, & 0\le t<
4,\\ t, & t\ge 4.\end{array}\right.$
    \item $f(t)=\left\{\begin{array}{rl} t, & 0\le
t<1,\\ 1, & t\ge 1.\end{array}\right.$
    \item $f(t)=\left\{\begin{array}{rl} te^t, & 0\le
t<1,\\t e^t, & t\ge 1.\end{array}\right.$
\end{enumerate}
\end{problem}

\begin{problem}\label{exer:8.1.6}
Prove that if $f(t)\leftrightarrow F(s)$ then $t^kf(t)\leftrightarrow
(-1)^kF^{(k)}(s)$.

\begin{hint}
    Assume that it's permissible to differentiate
the integral $\int_0^\infty e^{-st}f(t)\,dt$ with respect to $s$ under
the integral sign.
\end{hint}

\begin{solution}
If $F(s)=\int_0^\infty e^{-st}f(t)\,dt$, then
$F'(s)=\int_0^\infty(-te^{-st})f(t)\,dt=-\int_0^\infty
e^{-st}(tf(t))\,dt$. Applying this argument repeatedly yields
the assertion.
\end{solution}
\end{problem}

\begin{problem}\label{exer:8.1.7}
Use the known Laplace transforms
$$
{\cal L}(e^{\lambda t}\sin\omega t)=\frac{\omega}{(s-\lambda)^2+\omega^2}
\quad\text{and }\quad
{\cal L}(e^{\lambda t}\cos\omega t)=\frac{s-\lambda}{(s-\lambda)^2+\omega^2}
$$
  and the result of Exercise~\ref{exer:8.1.6} to find
${\cal L}(te^{\lambda t}\cos\omega t)$ and
${\cal L}(te^{\lambda t}\sin\omega t)$.
\end{problem}

\begin{problem}\label{exer:8.1.8}
 Use the known Laplace transform ${\cal L}(1)=1/s$ and the result of
Exercise~\ref{exer:8.1.6} to show that
$$
{\cal L}(t^n)=\frac{n!}{s^{n+1}},\quad n=\text{ integer}.
$$

\begin{solution}
Let $f(t)=1$ and $F(s)=1/s$. From
Exercise~\ref{exer:8.1.6},
$t^n\leftrightarrow(-1)^nF^{(n)}(s)=n!/s^{n+1}$.
\end{solution}
\end{problem}

\begin{problem}\label{exer:8.1.9}
\begin{enumerate}
\item % (a)
 Show that if $\lim_{t\to\infty} e^{-s_0t} f(t)$ exists and
is finite then  $f$ is of exponential order $s_0$.
\item % (b)
 Show that if $f$ is of exponential order $s_0$ then $\lim_{t
\to\infty} e^{-st} f(t)=0$ for all $s>s_0$.
\item % (c)
Show that if $f$ is of exponential order $s_0$  and $g(t)=f(t+\tau)$
where $\tau>0$, then $g$ is also of exponential order $s_0$.
\end{enumerate}
\end{problem}

\begin{problem}\label{exer:8.1.10}
Recall the following theorem from calculus.

\begin{theorem}
    Let $g$ be integrable on $[0,T]$ for
every $T>0.$  Suppose there is a function $w$ defined on some
interval $[\tau,\infty)$ (with $\tau\ge 0$) such that $|g(t)|\le
w(t)$ for $t\ge\tau$ and $\int^\infty_\tau w(t)\,dt$
converges.  Then $\int_0^\infty g(t)\,dt$ converges.
\end{theorem} 

Use this theorem to show that if $f$ is piecewise continuous on
$[0,\infty)$ and of exponential order $s_0$, then $f$ has a Laplace
transform $F(s)$ defined for $s>s_0$.

\begin{solution}
If $|f(t)|\le Me^{s_0t}$ for  $t\ge t_0$, then
$|f(t)e^{-st}|\le Me^{-(s-s_0)t}$ for  $t\ge t_0$. Let
 $g(t)=e^{-st}f(t)$, $w(t)=Me^{-(s-s_0)t}$, and $\tau=t_0$.
Since $\int_{t_0}^\infty w(t)\,dt$ converges if $s>s_0$, $F(s)$
is defined for $s>s_0$.
\end{solution}
\end{problem}

\begin{problem}\label{exer:8.1.11}
 Prove: If $f$ is piecewise continuous and of exponential order
then $\lim_{s\to\infty}F(s)~=~0$.
\end{problem}

\begin{problem}\label{exer:8.1.12}
Prove: If $f$ is continuous on $[0,\infty)$ and of exponential order
$s_0>0$, then
$$
{\cal L}\left(\int^t_0 f(\tau)\,d\tau\right)=\frac{1}{s} {\cal L} (f),
\quad s>s_0.
$$
\begin{hint}
Use integration by parts to evaluate the transform on
the left. 
\end{hint}

\begin{solution}
$\int_0^Te^{-st}\left(\int_0^tf(\tau)\,d\tau\right)\,dt
=-\frac{e^{-st}}{s}\int_0^t f(\tau)\,d\tau\bigg|_0^T+\frac{1}{s}\int_0^Te^{-st}f(t)\,dt =-\frac{e^{-sT}}{s}\int_0^T
f(\tau)\,d\tau+\frac{1}{s}\int_0^Te^{-st}f(t)\,dt$. Since $f$ is of
exponential order $s_0$, the second integral on the right converges to

$\frac{1}{s}{\cal L}(f)$ as $T\to\infty$ (Exercise~\ref{exer:8.1.10}).
Now, it suffices to show

(A) $\lim_{T\to\infty}e^{-sT}\int_0^Tf(\tau)\,d\tau=0$ if $s>s_0$.

Suppose that $|f(t)|\le Me^{s_0t}$ if $t\ge t_0$ and $|f(t)|\le K$ if
$0\le t\le t_0$, and let $T> t_0$. Then
$\left|\int_0^Tf(\tau)\,d\tau\right|
\le\left|\int_0^{t_0}f(\tau)\,d\tau\right|+
\left|\int_{t_0}^Tf(\tau)\,d\tau \right|<Kt_0+M\int_{t_0}^T
e^{s_0\tau}\,d\tau<Kt_0+\frac{Me^{s_0T}}{s_0}$, which proves (A).

\end{solution}
\end{problem}

\begin{problem}\label{exer:8.1.13}
Suppose $f$ is piecewise continuous and of exponential
order, and that $\lim_{t\to 0+} f(t)/t$ exists.  Show that
$$
{\cal L}\left(\frac{f(t)}{t}\right)=\int^\infty_s F(r)\,dr.
$$
\begin{hint}
    Use  the results of Exercises~$6$ and $11$.
\end{hint}
\end{problem}

\begin{problem}\label{exer:8.1.14}
Suppose $f$ is piecewise continuous on $[0,\infty)$.
\begin{enumerate}
\item % (a)
Prove: If the integral $g(t)=\int^t_0 e^{-s_0\tau} f(\tau)\,d\tau$
satisfies the inequality $|g(t)|\le M\; (t\ge 0)$, then $f$ has a
Laplace transform $F(s)$ defined for $s>s_0$.

\begin{hint}
Use integration by
parts to show that
$$
\int_0^T e^{-st}f(t)\,dt = e^{-(s-s_0)T}g(T)
+(s-s_0)\int_0^Te^{-(s-s_0)t}g(t)\,dt.
$$
\end{hint}

\begin{solution}
If $T>0$, then
$\int_0^Te^{-st}f(t)\,dt=\int_0^Te^{-(s-s_0)t}(e^{-s_0t}f(t))\,dt$.
Use integration by parts with $u=e^{-(s-s_0)t}$,
$dv=e^{-s_0t}f(t)\,dt$, $du=-(s-s_0)e^{-(s-s_0)t}$, and $v=g$ to obtain
$\int_0^T e^{-st}f(t)\,dt = e^{-(s-s_0)t}g(t)\left. \right]_0^T
+(s-s_0)\int_0^Te^{-(s-s_0)t}g(t)\,dt$. Since $g(0)=0$ this reduces to
$\int_0^T e^{-st}f(t)\,dt = e^{-(s-s_0)T}g(T)
+(s-s_0)\int_0^Te^{-(s-s_0)t}g(t)\,dt$. Since $|g(t)|\le M$ for all
$t\ge0$, we can let $t\to\infty$ to conclude that $\int_0^\infty
e^{-st}f(t)\,dt = (s-s_0)\int_0^\infty e^{-(s-s_0)t}g(t)\,dt$ if
$s>s_0$.
\end{solution}

\item % (b)
 Show that if ${\cal L}(f)$ exists for
$s=s_0$ then it exists for $s>s_0$.
Show that the function
$$
f(t)=te^{t^2}\cos(e^{t^2})
$$
has a Laplace transform defined for $s>0$, even though $f$ is not of exponential order.

\begin{solution}
If $F(s_0)$ exists, then $g(t)$ is bounded on $[0,\infty)$.
Now apply the previous part.
\end{solution}

\item % (c)
Show that the function
$$
f(t)=te^{t^2}\cos(e^{t^2})
$$
has a Laplace transform defined for $s>0$, even though $f$ is not of exponential order.

\begin{solution}
Since  $f(t)=\frac{1}{2}\frac{d}{dt}\sin(e^{t^2})$,
$\left|\int_0^tf(\tau)\,d\tau\right|=\frac{|\sin(e^{t^2})-\sin(1)|}{2}\le1$
for all $t\ge0$. Now apply the first part of this exercise with $s_0=0$.
\end{solution}
\end{enumerate}
\end{problem}

\begin{problem}\label{exer:8.1.15}
Use the table of Laplace transforms and the result of
Exercise~\ref{exer:8.1.13} to find the Laplace transforms of the following
functions.

\begin{enumerate}
    \item $\frac{\sin\omega t}{t}\quad(\omega>0)$
    \item $\frac{\cos\omega t-1}{t}\quad (\omega>0)$
    \item $\frac{e^{at}-e^{bt}}{t}$
    \item $\frac{\cosh t-1}{t}$
    \item $\frac{\sinh^2 t}{t}$
\end{enumerate}
\end{problem}

\begin{problem}\label{exer:8.1.16}
  The \emph{gamma function} is defined by
$$
\Gamma (\alpha)=\int_0^\infty x^{\alpha-1}e^{-x}\,dx,
$$
which can be shown to converge if $\alpha>0$.

\begin{enumerate}
\item % (a)
 Use integration by parts to show that
$$
\Gamma (\alpha+1)=\alpha\Gamma (\alpha),\quad\alpha>0.
$$
\begin{solution}
$\Gamma (\alpha)=\int_0^\infty x^{\alpha-1}e^{-x}\,dx=
{x^\alpha e^{-x}\over\alpha}\bigg|_0^\infty+\frac{1}{\alpha}
\int_0^\infty x^\alpha e^{-x}\,dx=\frac{\Gamma(\alpha+1)}{\alpha}$.
\end{solution}

\item % (b)
 Show that $\Gamma(n+1)=n!$ if $n=1$, $2$, $3$,\dots.

\begin{solution}
Use induction. $\Gamma(1)=\int_0^\infty e^{-x}\,dx=1$.
If (A) $\Gamma(n+1)=n!$, then $\Gamma(n+2)=(n+1)\Gamma(n+1)$ (from the previous part) $=(n+1)n!$ (from the inductive hypothesis) $=(n+1)!$.
\end{solution}
 
\item % (c)
 Use the above facts and the table of Laplace transforms to show
$$
{\cal L}(t^\alpha)=\frac{\Gamma (\alpha+1)}{s^{\alpha+1}},\quad s>0,
$$
if $\alpha$ is a nonnegative integer.  Show that this formula is valid for
any $\alpha>-1$.
\begin{hint}
Change the variable of integration in the
integral for $\Gamma (\alpha+1)$.
\end{hint}

\begin{solution}
$\Gamma(\alpha+1)=\int_0^\infty x^\alpha e^{-x}\,dt$. Let
$x=st$. Then $\Gamma(\alpha+1)=\int_0^\infty (st)^\alpha
e^{-st}s\,dt$, so $\int_0^\infty
e^{-st}t^\alpha\,dt=\frac{\Gamma(\alpha+1)}{\alpha}$.
\end{solution}

\end{enumerate}
\end{problem}

\begin{problem}\label{exer:8.1.17}
  Suppose $f$ is continuous on $[0, T]$ and $f(t+T)=f(t)$
for all $t\ge 0$.  (We say in this case that $f$ is \emph{periodic with
period\/} $T$.)
\begin{enumerate}
\item % (a)
 Conclude from Theorem~\ref{thmtype:8.1.6} that the Laplace transform of
$f$ is defined for $s>0$. 
\begin{hint}
   Since $f$ is
continuous on $[0,T]$ and periodic with period $T$, it's bounded
on $[0,\infty)$. 
\end{hint}
\item\label{exer:8.1.17b} % (b) 
Show that
$$
F(s)=\frac{1}{1-e^{-sT}}\int_0^T e^{-st}f(t)\,dt,\quad s>0.
$$
\begin{hint}
    Write
$$
F(s)=\sum^\infty_{n=0}\int^{(n+1)T}_{nT}e^{-st} f(t)\,dt.
$$
Then show that
$$
\int^{(n+1)T}_{nT} e^{-st}f(t)\,dt=e^{-nsT}\int_0^T e^{-st}f(t)\,dt,
$$
and recall the formula for the sum of a geometric series.
\end{hint}
\end{enumerate}
\end{problem}

\begin{problem}\label{exer:8.1.18}
  Use the formula given in Exercise~\ref{exer:8.1.17b} to find
the
Laplace transforms of the given periodic functions:
\begin{enumerate}
\item % (a)
 $f(t)=\left\{\begin{array}{cl} t, & 0\le
t<1,\\ 2-t, & 1\le t<2,\end{array}\right.\hskip30pt f(t+2)=f(t),
\quad t\ge 0$

\begin{solution}
$\int_0^2e^{-st}f(t)\,dt=\int_0^1e^{-st}t\,dt+\int_1^2e^{-st}(2-t)\,dt=
\left(\frac{1}{s^2}-\frac{e^{-s}(s+1)}{s^2}\right)
+\left(\frac{e^{-s}(s-1)}{s^2}+\frac{e^{-2s}}{s^2}\right)=
-\frac{2e^{-s}}{s^2}+\frac{e^{-2s}}{s^2}+\frac{1}{s^2}
=\frac{(1-e^{-s})^2}{s^2}$. Therefore,$F(s)=\frac{(1-e^{-s})^2}{s^2(1-e^{-2s})}=\frac{1-e^{-s}}{s^2(1+e^{-s})}=
\frac{1}{s^2}\tanh\frac{s}{2}$.
\end{solution}

\item % (b)
 $f(t)=\left\{\begin{array}{rl}1, & 0\le
t<\frac{1}{2},\\ -1, & \frac{1}{2}\le t<1,\end{array}\right.
\hskip30pt f(t+1)=f(t),\quad t\ge 0$

\begin{solution}
$\\int_0^1e^{-st}f(t)\,dt=\int_0^{1/2}e^{-st}\,dt-\int_{1/2}^1e^{-st}\,dt=
\frac{1}{s}-\frac{e^{-s/2}}{s}+
\frac{e^{-s}}{s}-\frac{e^{-s/2}}{s}=
-\frac{2e^{-s/2}}{s}+\frac{e^{-s}}{s}+\frac{1}{s}=
\frac{(1-e^{-s/2})^2}{s}$. Therefore,$F(s)=\frac{(1-e^{-s/2})^2}{s(1-e^{-s})}=\frac{1-e^{-s/2}}{s(1+e^{-s/2})}=
\frac{1}{s}\tanh\frac{s}{4}$.
\end{solution}

\item % (c)
 $f(t)=|\sin t|$

\begin{solution}
$\int_0^\pi e^{-st}f(t)\,dt=\int_0^\pi e^{-st}\sin t\,dt=
\frac{1+e^{-\pi s}}{(s^2+1)}$. Therefore,$F(s)=\frac{1+e^{-\pi
s}}{(s^2+1)(1-e^{-\pi s})}
\frac{1}{s^2+1}\coth \frac{\pi s}{2}$.
\end{solution}
 
\item % (d)
 $f(t)=\left\{\begin{array}{cl}\sin t, & 0\le t<
\pi,
\\ 0, &\pi\le t<2\pi,\end{array}\right.\hskip30pt
f(t+2\pi)=f(t)$

\begin{solution}
$\int_0^{2\pi}e^{-st}f(t)\,dt=\int_0^\pi
e^{-st}\sin t\,dt=
\frac{1+e^{-\pi s}}{(s^2+1)}$. Therefore,$F(s)=\frac{1+e^{-\pi s}}{(s^2+1)
1+e^{-2\pi s}}=
\frac{1}{(s^2+1)(1-e^{-\pi s})}$.
\end{solution}

\end{enumerate}
\end{problem}


\end{document}