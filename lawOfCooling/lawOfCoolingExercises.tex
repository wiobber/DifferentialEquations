\documentclass{ximera}
\input{../preamble.tex}

\title{Exercises} \license{CC BY-NC-SA 4.0}

\begin{document}

\begin{abstract}
\end{abstract}
\maketitle

\begin{onlineOnly}
\section*{Exercises}
\end{onlineOnly}


\begin{problem}\label{exer:4.2.1}
A thermometer is moved from a room where the temperature is
$70^\circ$F to a freezer where the temperature is $12^\circ
F$.  After 30 seconds the thermometer reads
$40^\circ$F.  What does it read after 2 minutes?
\end{problem}

\begin{problem}\label{exer:4.2.2}
A fluid initially at $100^\circ$C is placed outside on a
day when the temperature is $-10^\circ$C, and the
temperature of the fluid drops $20^\circ$C in one minute.
Find the temperature $T(t)$ of the fluid for $t > 0$.

\begin{solution}
Since $T_0=100$ and $T_M=-10$, $T=-10+110e^{-kt}$. Now
$T(1)=80\Rightarrow 80=-10+110e^{-k}$, so $e^{-k}=\frac{9}{11}$
and $k=\ln\frac{11}{9}$. Therefore, $T=-10+110e^{-t\ln\frac{11}{9}}$.
\end{solution}
\end{problem}

\begin{problem}\label{exer:4.2.3}
At 12:00 {\sc pm} a thermometer reading $10^\circ$F is placed in
a room where the temperature is $70^\circ$F.  It reads
$56^\circ$ when it's placed outside, where the temperature
is $5^\circ$F, at 12:03.  What does it read at 12:05 {\sc pm}?
\end{problem}

\begin{problem}\label{exer:4.2.4}
A thermometer initially reading $212^\circ$F is placed in a
room where the temperature is $70^\circ$F.  After 2
minutes the thermometer reads $125^\circ$F.

\begin{enumerate}
\item %(a)
What does the thermometer read after 4 minutes?

\begin{solution}
Let $T$ be the thermometer reading.
Since $T_0=212$ and $T_M=70$, $T=70+142e^{-kt}$. Now
$T(2)=125\Rightarrow 125=70+142e^{-2k}$, so
$e^{-2k}=\frac{55}{142}$
and $k=\frac{1}{2}\ln\frac{142}{55}$. Therefore,
(A) $T=70+142e^{-\frac{t}{2}\ln\frac{142}{55}}$.

$T(2)=70+142e^{-2\ln\frac{142}{55}}=70+142\left(\frac{55}{142}\right)^2
\approx91.30^\circ$F.
\end{solution}

\item %(b)
When will the thermometer read $72^\circ$F?

\begin{solution}
Let $\tau$ be the time when
 $T(\tau)=72$, so $72=70+142e^{-\frac{\tau}{2}\ln\frac{142}{55}}$, or
$e^{-\frac{\tau}{2}\ln\frac{142}{55}}=\frac{1}{71}$. Therefore,
$\tau=2\frac{\ln71}{\ln\frac{142}{55}}\approx8.99$ min.
\end{solution}

\item %(c)
When will the thermometer read $69^\circ$F?

\begin{solution}
Since (A) implies that $T>70$ for all $t>0$, the
thermometer will never read $69^\circ$F.
\end{solution}
\end{enumerate}
\end{problem}

\begin{problem}\label{exer:4.2.5}
An object with initial temperature $150^\circ$C is placed
outside, where the temperature is $35^\circ$C.  Its
temperatures at 12:15 and 12:20 are $120^\circ$C and
$90^\circ$C, respectively.

\begin{enumerate}
\item %(a)
At what time was the object placed outside?

\item %(b)
When will its temperature be $40^\circ$C?
\end{enumerate}
\end{problem}

\begin{problem}\label{exer:4.2.6}
An object is placed in a room where the temperature is
$20^\circ$C.  The temperature of the object drops by
$5^\circ$C in 4 minutes and by $7^\circ$C in 8 minutes.
What was the temperature of the object when it was initially
placed in the room?

\begin{solution}
Since $T_M=20$, $T=20+(T_0-20)e^{-kt}$. Now
$T_0-5=20+(T_0-20)e^{-4k}$  and $T_0-7=20+(T_0-20)e^{-8k}$. Therefore,
$\frac{T_0-25}{T_0-20}=e^{-4k}$ and $\left(\frac{T_0-27}{T_0-20}\right)=e^{-8k}$, so  $\frac{T_0-27}{T_0-20}=\left(\frac{T_0-25}{T_0-20}\right)^2$,
which implies that $(T_0-20)(T_0-27)=(T_0-25)^2$, or
$T_0^2-47T_0+540=T_0^2-50T_0+625$; hence $3T_0=85$
and  $T_0={(85/3)^\circ C}$.
\end{solution}
\end{problem}

\begin{problem}\label{exer:4.2.7}
A cup of boiling water is placed outside at 1:00 {\sc pm}.  One
minute later the temperature of the water is $152^\circ$F.
After another minute its temperature is $112^\circ$F.  What
is the outside temperature?
\end{problem}



\begin{problem}\label{exer:4.2.16}
Suppose an object  with initial temperature $T_0$
is placed in a sealed container, which is in turn placed in a medium with
temperature $T_m$. Let the initial
temperature of the container be $S_0$. Assume that the temperature of the
object does not affect the temperature of the container, which in turn does
not affect the temperature of the medium. (These assumptions  are
reasonable, for example, if the object is a cup of coffee, the container is
a house, and the medium is the atmosphere.)

\begin{enumerate}
\item % (a)
Assuming that the container and the medium have distinct temperature
decay constants $k$ and $k_m$ respectively, use Newton's law of
cooling to find the temperatures $S(t)$ and $T(t)$ of the container
and object at time $t$.

\begin{solution}
$S'=-k_m(S-T_m),\ S(0)=0$, so (A)
$S=T_m+(S_0-T_m)e^{-k_mt}$.
$T'=-k(T-S)=-k\left(T-T_m-(S_0-T_m)e^{-k_mt}\right)$,
from
(A). Therefore,$T'+kT=kT_m+k(S_0-T_m)e^{-k_mt}$; $T=ue^{-kt}$;
(B) $u'=kT_me^{kt}+k(S_0-T_m)e^{(k-k_m)t}$;
 $u=T_me^{kt}+\frac{k}{k-k_m}(S_0-T_m)e^{(k-k_m)t}+c$;
 $T(0)=T_0\Rightarrow
 c=T_0-T_m-\frac{k}{k-k_m}(S_0-T_m)$;
 $u=T_me^{kt}+\frac{k}{k-k_m}(S_0-T_m)e^{(k-k_m)t}+T_0-T_m-\frac{k}{k-k_m}(S_0-T_m)$;
$T=T_m+(T_0-T_m)e^{-kt}+\frac{k(S_0-T_m)}{(k-k_m)}\left(e^{-k_mt}-e^{-kt}\right)$.
\end{solution}

\item % (b)
Assuming that the container and the medium have the same temperature
decay constant $k$, use Newton's law of cooling to find the
temperatures $S(t)$ and $T(t)$ of the container and object at time
$t$.

\begin{solution}
If $k=k_m$ (B) becomes
(B) $u'=kT_me^{kt}+k(S_0-T_m)$; $u=T_me^{kt}+k(S_0-T_m)t+c$;
$T(0)=T_0\Rightarrow c=T_0-T_m$;
$u=T_me^{kt}+k(S_0-T_m)t+(T_0-T_m)$;
$T=T_m+k(S_0-T_m)te^{-kt}+(T_0-T_m)e^{-kt}$.
\end{solution}

\item % (c)
Find $\lim._{t\to\infty}S(t)$  and $\lim_{t\to\infty}T(t)$ .

\begin{solution}
$\lim_{t\to\infty}T(t)=\lim_{t\to\infty}S(t)=T_m$
in either case.
\end{solution}
\end{enumerate}
\end{problem}

\begin{problem}\label{exer:4.2.17}  %\exercisemolten
In  our previous examples and exercises concerning Newton's law of cooling
we assumed that
the temperature of the medium remains constant.  This model is adequate
if the heat lost or gained by the object is
insignificant compared to the heat required to cause an appreciable change
in the temperature of the medium. If this isn't  so,  we
must use a model that accounts for the heat exchanged between the object
and the medium. Let $T=T(t)$ and $T_m=T_m(t)$  be the temperatures of the
object and the medium, respectively, and let $T_0$  and $T_{m0}$ be
their
initial values.  Again, we assume that $T$  and $T_m$ are related by
Newton's law of cooling,
$$
T'=-k(T-T_m).
\text{(A)}
$$
We also assume that the change in heat of the object as its
temperature changes from $T_0$ to $T$  is $a(T-T_0)$
and that the change in heat of the medium  as its
temperature changes from $T_{m0}$ to $T_m$  is $a_m(T_m-T_{m0})$,
where $a$
and $a_m$ are positive constants  depending  upon the masses and
thermal properties of the
object and medium, respectively.  If we assume that the total heat of
the system consisting of the object and the medium remains constant
(that is, energy is conserved), then
$$
a(T-T_0)+a_m(T_m-T_{m0})=0.
\text{(B)}
$$

\begin{enumerate}
\item % (a)
Equation (A) involves  two unknown functions $T$ and $T_m$.
 Use (A) and (B) to derive a
differential equation involving only $T$.
\item % (b)
Find $T(t)$ and $T_m(t)$ for $t>0$.
\item % (c)
Find $\lim_{t\to\infty}T(t)$ and
 $\lim_{t\to\infty}T_m(t)$.
\end{enumerate}
\end{problem}

\end{document}