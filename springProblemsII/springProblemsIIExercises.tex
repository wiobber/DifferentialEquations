\documentclass{ximera}
\input{../preamble.tex}

\title{Exercises} \license{CC BY-NC-SA 4.0}

\begin{document}

\begin{abstract}
\end{abstract}
\maketitle

\begin{onlineOnly}
\section*{Exercises}
\end{onlineOnly}

\begin{problem}\label{exer:6.2.1}
A 64 lb object stretches a spring 4 ft in equilibrium. It is attached
to a dashpot with damping constant $c=8$ lb-sec/ft. The object is
initially displaced 18 inches above equilibrium and given a downward
velocity of 4 ft/sec. Find its displacement and time--varying
amplitude for $t>0$.
\end{problem}

\begin{problem}\label{exer:6.2.2}  
A 16 lb weight is attached to a spring with natural length 5 ft. With
the weight attached, the spring measures 8.2 ft. The weight is
initially displaced 3 ft below equilibrium and given an upward
velocity of 2 ft/sec. Find and graph its displacement
 for $t>0$ if the medium resists the motion with a force of
one lb for each ft/sec of velocity. Also, find its time--varying amplitude.
\end{problem}

\begin{problem}\label{exer:6.2.3}  
An 8 lb weight stretches a spring 1.5 inches. It is attached to a
dashpot with damping constant $c$=8 lb-sec/ft. The weight is
initially displaced 3 inches above equilibrium and given an upward
velocity of 6 ft/sec. Find and graph its displacement for $t>0$.
\end{problem}

\begin{problem}\label{exer:6.2.4}
A 96 lb weight stretches a spring 3.2 ft in equilibrium. It is
attached to a dashpot with damping constant $c$=18 lb-sec/ft. The
weight is initially displaced 15 inches below equilibrium and given a
downward velocity of 12 ft/sec. Find its displacement for $t>0$.
\end{problem}

\begin{problem}\label{exer:6.2.5}
  A 16 lb weight  stretches a spring 6 inches in equilibrium.
It is attached to a damping mechanism with constant $c$.  Find all values
of $c$ such that the free vibration of the weight has infinitely many
oscillations.
\end{problem}

\begin{problem}\label{exer:6.2.6}
An 8 lb weight stretches a spring .32 ft. The weight is initially
displaced 6 inches above equilibrium and given an upward velocity of 4
ft/sec. Find its displacement for $t>0$ if the medium exerts a
damping force of 1.5 lb for each ft/sec of velocity.
\end{problem}

\begin{problem}\label{exer:6.2.7}
A 32 lb weight stretches a spring 2 ft in equilibrium. It is attached
to a dashpot with constant $c=8$ lb-sec/ft. The weight is initially
displaced 8 inches below equilibrium and released from rest. Find its
displacement for $t>0$.
\end{problem}

\begin{problem}\label{exer:6.2.8}
  A mass of 20 gm stretches a  spring 5 cm.  The spring
is attached to a dashpot with damping constant 400 dyne sec/cm.  Determine
the displacement for $t>0$ if the mass is  initially displaced
 9 cm  above equilibrium  and released from rest.
\end{problem}

\begin{problem}\label{exer:6.2.9}
A 64 lb weight is suspended from a spring with constant $k=25$ lb/ft.
It is initially displaced 18 inches above equilibrium and released
from rest. Find its displacement for $t>0$ if the medium resists the
motion with 6 lb of force for each ft/sec of velocity.
\end{problem}

\begin{problem}\label{exer:6.2.10}
A 32 lb weight stretches a spring 1 ft in equilibrium. The weight is
initially displaced 6 inches above equilibrium and given a downward
velocity of 3 ft/sec. Find its displacement for $t>0$ if the medium
resists the motion with a force  equal to 3 times the speed
in ft/sec.

\begin{solution}
Since $k=\frac{mg}{\Delta l}=\frac{32}{1}=32$ the equation of
motion is (A) $y''+3y'+32y=0$. The characteristic polynomial of (A) is
$p(r)=r^2+3r+32=\left(r+\frac{3}{2}\right)^2+\frac{119}{4}$.
Therefore,the general solution of (A) is
$y=e^{-3t/2}\left(c_1\cos\frac{\sqrt{119}}{2}t
+c_2\sin\frac{\sqrt{119}}{2}t\right)$, so $y'=-\frac{3}{2}y+
\frac{\sqrt{119}}{2}e^{-3t/2}
\left(-c_1\sin\frac{\sqrt{119}}{2}t+c_2\cos\frac{\sqrt{119}}{2}t\right)$.
Now $y(0)=\frac{1}{2}$ and $y'(0)=-3\Rightarrow c_1=\frac{1}{2}$
and $-3=-\frac{3}{4}+\frac{\sqrt{119}}{2}c_2$, or
$c_2=-\frac{9}{2\sqrt{119}}$. Therefore,
$y=e^{-\frac{3}{2}t}\left(\frac{1}{2}\cos \frac{\sqrt{119}}{2} t
-\frac{9}{2\sqrt{119}}\sin\frac{\sqrt{119}}{2}t\right)$ ft.
\end{solution}
\end{problem}

\begin{problem}\label{exer:6.2.11}
An 8 lb weight stretches a spring 2 inches. It is attached to a
dashpot with damping constant $c$=4 lb-sec/ft. The weight is
initially displaced 3 inches above equilibrium and given a downward
velocity of 4 ft/sec. Find its displacement for $t>0$.
\end{problem}

\begin{problem}\label{exer:6.2.12}  
A 2 lb weight stretches a spring .32 ft. The weight is initially
displaced 4 inches below equilibrium and given an upward velocity of 5
ft/sec. The medium provides damping with constant $c=1/8$ lb-sec/ft.
Find and graph the displacement for $t>0$.
\end{problem}

\begin{problem}\label{exer:6.2.13}
  An 8 lb weight stretches a spring 8 inches in equilibrium.
It is attached to a dashpot with damping constant $c=.5$ lb-sec/ft
and subjected to an external force  $F(t)=4\cos2t$ lb.
Determine the steady state component of the displacement for $t>0$.
\end{problem}

\begin{problem}\label{exer:6.2.14}
A 32 lb weight stretches a spring 1 ft in equilibrium. It is attached
to a dashpot with constant $c=12$ lb-sec/ft. The weight is initially
displaced 8 inches above equilibrium and released from rest. Find its
displacement for $t>0$.
\end{problem}

\begin{problem}\label{exer:6.2.15}
A mass of one kg stretches a spring 49 cm in equilibrium. A dashpot
attached to the spring supplies a damping force of 4 N for each m/sec
of speed. The mass is initially displaced 10 cm above equilibrium and
given a downward velocity of 1 m/sec. Find its displacement for $t>0$.
\end{problem}

\begin{problem}\label{exer:6.2.16}
A mass of 100 grams stretches a spring 98 cm in equilibrium. A dashpot
attached to the spring supplies a damping force of 600 dynes for each
cm/sec of speed. The mass is initially displaced 10 cm above
equilibrium and given a downward velocity of 1 m/sec. Find its
displacement for $t>0$.
\end{problem}

\begin{problem}\label{exer:6.2.17}
A 192 lb weight is suspended from a spring with constant $k=6$ lb/ft
and subjected to an external force $F(t)=8\cos3t$ lb. Find the steady
state component of the displacement for $t>0$ if the medium resists
the motion with a force  equal to 8 times the speed in
ft/sec.
\end{problem}

\begin{problem}\label{exer:6.2.18}
A 2 gm mass is attached to a spring with constant 20 dyne/cm. Find the
steady state component of the displacement if the mass is subjected to
an external force $F(t)=3\cos4t-5\sin4t$ dynes and a dashpot supplies
4 dynes of damping for each cm/sec of velocity.
\end{problem}

\begin{problem}\label{exer:6.2.19}  
A 96 lb weight is attached to a spring with constant 12 lb/ft. Find
and graph the steady state component of the displacement if the mass is
subjected to an external force $F(t)=18\cos t-9\sin t$ lb and a
dashpot supplies 24 lb of damping for each ft/sec of velocity.
\end{problem}

\begin{problem}\label{exer:6.2.20}
A mass of one kg stretches a spring 49 cm in equilibrium. It is
attached to a dashpot that supplies a damping force of 4 N for each
m/sec of speed. Find the steady state component of its displacement if
it's subjected to an external force $F(t)=8\sin2t-6\cos2t$ N.

\begin{solution}
Since $k=\frac{mg}{\Delta l}=\frac{9.8}{.49}=20$ the
equation of motion is
(A) $y''+4y'+20y=8\sin2t-6\cos2t$.
The steady state component of the solution of (A)
is of the form
$y_p=A\cos2t+B\sin2t$; therefore $y_p'=-2A\sin2t+2B\cos2t$
and $y_p''=-4A\cos2t-4B\sin2t$, so
$y_p''+4y_p'+20y_p=(16A+8B)\cos2t-(8A-16B)\sin2t=8\sin2t-6\cos2t$
if $16A+8B=-6$, $-8A+16B=8$; therefore $A=-\frac{1}{2}$,
$B=\frac{1}{4}$, and
$y=-\frac{1}{2}\cos2t+\frac{1}{4}\sin2t$ m.
\end{solution}
\end{problem}

\begin{problem}\label{exer:6.2.21}
A mass $m$ is suspended from a spring with constant $k$ and subjected to an
external force $F(t)=\alpha\cos\omega_0t+\beta\sin\omega_0t$, where
$\omega_0$ is the natural frequency of the spring--mass system without
damping. Find the steady state component of the displacement if a dashpot
with constant $c$ supplies damping.
\end{problem}

\begin{problem}\label{exer:6.2.22}
Show that if $c_1$ and $c_2$ are not both zero then
$$
y=e^{r_1t}(c_1+c_2t)
$$
can't equal zero for more than one value of
$t$.
\end{problem}

\begin{problem}\label{exer:6.2.23}
Show that if $c_1$ and $c_2$ are not both zero then
$$
y=c_1e^{r_1t}+c_2e^{r_2t}
$$
 can't equal zero for more than one
value of $t$.
\end{problem}

\begin{problem}\label{exer:6.2.24}
Find the solution of the initial value problem
$$
my''+cy'+ky=0,\quad y(0)=y_0,\;y'(0)=v_0,
$$
given that the motion is underdamped, so  the general solution of
the equation is
$$
y=e^{-ct/2m}(c_1\cos\omega_1t+c_2\sin\omega_1t).
$$
\end{problem}

\begin{problem}\label{exer:6.2.25}
Find the solution of the initial value problem
$$
my''+cy'+ky=0,\quad y(0)=y_0,\;y'(0)=v_0,
$$
given that the motion is overdamped, so  the general solution of
the equation is
$$
y=c_1e^{r_1t}+c_2e^{r_2t}\;(r_1,r_2<0).
$$
\end{problem}

\begin{problem}\label{exer:6.2.26}
Find the solution of the initial value problem
$$
my''+cy'+ky=0,\quad y(0)=y_0,\;y'(0)=v_0,
$$
given that the motion is critically damped, so that the general
solution of the equation is of the form
$$
y=e^{r_1t}(c_1+c_2t)\,(r_1<0).
$$
\end{problem}

\end{document}