\documentclass{ximera}
\input{../preamble.tex}

\title{Exercises} \license{CC BY-NC-SA 4.0}

\begin{document}

\begin{abstract}
\end{abstract}
\maketitle

\begin{onlineOnly}
\section*{Exercises}
\end{onlineOnly}

\begin{problem}
In Exercises~\ref{exer:4.4.1}--\ref{exer:4.4.4} find the equations of the
trajectories of the given undamped equation. Identify the equilibrium
solutions, determine whether they are stable or unstable, and
plot some trajectories. 
\begin{hint}
  Use Eqn.~$\eqref{eq:4.4.8}$ to
obtain the equations of the trajectories.  
\end{hint}

\begin{enumerate}
\item\label{exer:4.4.1}   $y''+y^3=0$ 

\begin{solution}
$\overline y=0$ is a stable equilibrium.
The phase plane equivalent is $v\frac{dv}{dy}+y^3=0$, so the
 trajectories are $v^2+\frac{y^4}{4}=c$.
\end{solution}

\item\label{exer:4.4.2}   $y''+y^2=0$

\begin{solution}
$\overline y=0$ is an unstable equilibrium.
The phase plane equivalent is $v\frac{dv}{dy}+y^2=0$, so the
 trajectories are $v^2+\frac{2y^3}{3}=c$.
\end{solution}

\item\label{exer:4.4.3}  
$y''+y|y|=0$ 

\item\label{exer:4.4.4}   $y''+ye^{-y}=0$

\begin{solution}
$\overline y=0$ is a stable equilibrium.
The phase plane equivalent is $v\frac{dv}{dy}+ye^{-y}=0$, so the
 trajectories are $v^2-e^{-y}(y+1)=c$.
\end{solution}
\end{enumerate}
\end{problem}

\begin{problem}
In Exercises~\ref{exer:4.4.5}--\ref{exer:4.4.8} find the equations of the
trajectories of the given undamped equation. Identify the equilibrium
solutions,
determine whether they are stable or unstable, and find the equations
of the separatrices (that is, the curves through the unstable
equilibria). Plot the separatrices and some
trajectories in each of the regions of Poincar\'e plane determined by
them. 
\begin{hint}
  Use Eqn.~$\eqref{eq:4.4.17}$ to determine the separatrices.  
\end{hint}

\begin{enumerate}
\item\label{exer:4.4.5}   $y''-y^3+4y=0$

\item\label{exer:4.4.6}   $y''+y^3-4y=0$

\begin{solution}
$p(y)=y^3-4y=(y+2)y(y-2)$, so the
 equilibria are $-2,0,2$.
Since
$$
\begin{array}{rcl}
y(y-2)(y+2)&<0& \mbox{ if } y<-2\mbox{ or }0<y<2,\\
&>0&\mbox{ if }-2<y<0\mbox{ or }y>2,
\end{array}
$$
$0$ is unstable and $-2,2$ are stable. The phase plane equivalent is
$v\frac{dv}{dy}+y^3-4y=0$, so the trajectories are
$2v^2+y^4-8y^2=c$. Setting $(y,v)=(0,0)$ yields $c=0$, so the equation
of the separatrix is $2v^2-y^4+8y^2=0$.
\end{solution}

\item\label{exer:4.4.7}   $y''+y(y^2-1)(y^2-4)=0$

\item\label{exer:4.4.8}   $y''+y(y-2)(y-1)(y+2)=0$

\begin{solution}
$p(y)=y(y-2)(y-1)(y+2)$, so the
 equilibria are $-2,0,1,2$.
Since
$$
\begin{array}{rcl}
y(y-2)(y-1)(y+2)&>0& \mbox{ if } y<-2\mbox{ or }0<y<1\mbox{ or }
y>2,\\
&<0&\mbox{ if }-2<y<0\mbox{ or }1<y<2,
\end{array}
$$
$0,2$ are stable and $-2,1$ are unstable. The phase plane
equivalent is
$v\frac{dv}{dy}+y(y-2)(y-1)(y+2)=0$, so the trajectories are
$30v^2+y^2(12y^3-15y^2-80y+120)=c$. Setting $(y,v)=(-2,0)$
and $(y,v)=(1,0)$
yields
$c=496$ and $c=37$ respectively, so the equations of the separatrices
are $30v^2+y^2(12y^3-15y^2-80y+120)=496$
and $30v^2+y^2(12y^3-15y^2-80y+120)=37$.
\end{solution}
\end{enumerate}
\end{problem}

\begin{problem}
In Exercises~\ref{exer:4.4.9}--\ref{exer:4.4.12}
plot some trajectories of the given equation for various values
(positive, negative, zero) of the parameter $a$. Find the equilibria
of the equation and classify them as stable or unstable. Explain why
the phase plane plots corresponding to positive and negative values of
$a$ differ so markedly. Can you think of a reason why zero deserves to
be called the \emph{critical value} of $a$?

\begin{enumerate}
\item\label{exer:4.4.9} $y''+y^2-a=0$

\item\label{exer:4.4.10} $y''+y^3-ay=0$

\begin{solution}
$p(y)=y^3-ay$.
If $a\le0$, then $p(0)=0$, $p(y)>0$ if $y>0$, and $p(y)<0$ if $y<0$, so
$0$ is stable.
 If $a>0$, then
\begin{eqnarray*}
y^3-ay=y(y-\sqrt a)(y+\sqrt a)&>&0\mbox{ if }-\sqrt a<y<0\mbox{ or }
y>\sqrt a,\\
&<&0\mbox{ if }y<-\sqrt a\mbox{ or }0<y<\sqrt a,
\end{eqnarray*}
so $-\sqrt a$ and $\sqrt a$ are  stable  and $0$
is unstable. We say that $a=0$ is a critical value
because it separates the two cases.
\end{solution}

\item\label{exer:4.4.11} $y''-y^3+ay=0$

\item\label{exer:4.4.12} $y''+y-ay^3=0$
\begin{solution}
$p(y)=y-ay^3$. If $a\le0$, then $p(0)=0$, $p(y)>0$ if $y>0$, and
$p(y)<0$ if $y<0$, so $0$ is stable. If $a>0$, then
\begin{eqnarray*}
y-ay^3=-ay(y-1/\sqrt a)(y+1/\sqrt a)&>&0\mbox{ if }y<-1/\sqrt
a<y<0\mbox{ or } 0<y<1/\sqrt a\\
&<&0\mbox{ if }-1/\sqrt a<y<0\mbox{ or }y>1/\sqrt a,
\end{eqnarray*}
so $-\sqrt a$ and $\sqrt a$ are unstable and $0$ is stable. We say
that $a=0$ is a critical value because it separates the two cases.
\end{solution}
\end{enumerate}
\end{problem}

\begin{problem}
In Exercises~\ref{exer:4.4.13}-\ref{exer:4.4.18}
plot trajectories of the given equation for $c=0$ and small nonzero
(positive and negative) values of $c$ to observe the effects of
damping.

\begin{enumerate}
\item\label{exer:4.4.13} $y''+cy'+y^3=0$
\item\label{exer:4.4.14} $y''+cy'-y=0$
\item\label{exer:4.4.15} $y''+cy'+y^3=0$
\item\label{exer:4.4.16} $y''+cy'+y^2=0$
\item\label{exer:4.4.17} $y''+cy'+y|y|=0$
\item\label{exer:4.4.18} $y''+y(y-1)+cy=0$
\end{enumerate}
\end{problem}

\begin{problem}\label{exer:4.4.19} 
The \href{http://www-history.mcs.st-and.ac.uk/Mathematicians/Van_der_Pol.html}
{\emph{van der Pol equation}}
\begin{equation}\label{eqA:4.4.19}
y''-\mu(1-y^2)y'+y=0,
\end{equation}
where $\mu$ is a positive constant and $y$ is electrical current
\ref{Section~6.3}, arises in the study of an electrical circuit
whose resistive properties depend upon the current. The damping term
$-\mu(1-y^2)y'$ works to reduce $|y|$ if $|y|<1$ or to increase $|y|$
if
$|y|>1$. It can be shown that van der
Pol's equation has exactly one closed trajectory, which is called a
\emph{limit cycle}. Trajectories inside the limit cycle spiral
outward to it, while trajectories outside the limit cycle spiral
inward to it. 
%(See the figure below). 
Use your favorite differential equations software  to verify this for
$\mu=.5,1.1.5,2$. Use a grid with $-4<y<4$ and $-4<v<4$.

%\begin{image}
 % \includegraphics[height=1.5in]{fig040416.jpg}
%\end{image}
\end{problem}


\begin{problem}\label{exer:4.4.20}  
\href{http://www-history.mcs.st-and.ac.uk/Mathematicians/Rayleigh.html}
{\emph{Rayleigh's equation}},
$$
y''-\mu(1-(y')^2/3)y'+y=0
$$
also has a limit cycle. Follow the directions of
Exercise~\ref{exer:4.4.19} for this equation.
\end{problem}

\begin{problem}\label{exer:4.4.21}
In connection with Eqn~\eqref{eq:4.4.15},
suppose   $y(0)=0$ and $y'(0)=v_0$, where $0<v_0<v_c$.
\begin{enumerate}
\item % (a)
Let  $T_1$ be the time
required for  $y$ to increase from zero to
$y_{\max}=2\sin^{-1}(v_0/v_c)$. Show that
\begin{equation}\label{eqA:4.4.21}
\frac{dy}{dt}=\sqrt{v_0^2-v_c^2\sin^2y/2},\quad 0\le t<T_1.
\end{equation}
\item % (b)
Separate variables in (\ref{eqA:4.4.21}) and show that
\begin{equation}\label{eqB:4.4.21}
T_1=\int_0^{y_{\max}}\frac{du}{\sqrt{v_0^2-v_c^2\sin^2u/2}}
\end{equation}
\item % (c)
Substitute $\sin u/2=(v_0/v_c)\sin\theta$ in (\ref{eqB:4.4.21}) to obtain
\begin{equation}\label{eqC:4.4.21}
T_1=2\int_0^{\pi/2}\frac{d\theta}{\sqrt{v_c^2-v_0^2\sin^2\theta}}.
\end{equation}
\item % (d)
Conclude from  symmetry
that  the time required for  $(y(t),v(t))$ to traverse the
trajectory
$$
v^2=v_0^2-v_c^2\sin^2y/2
$$
is $T=4T_1$, and that consequently $y(t+T)=y(t)$ and $v(t+T)=v(t)$;
that is, the oscillation is periodic with period $T$.
\item % (e)
Show that if $v_0=v_c$,  the integral in (\ref{eqC:4.4.21}) is improper and
diverges to $\infty$. Conclude from this  that $y(t)<\pi$
for all $t$ and $\lim_{t\to\infty}y(t)=\pi$.
\end{enumerate}
\end{problem}

\begin{problem}\label{exer:4.4.22}
Give a direct definition of an unstable equilibrium of $y''+p(y)=0$.
\end{problem}

\begin{problem}\label{exer:4.4.23}
Let $p$ be continuous for all $y$ and $p(0)=0$. Suppose there's
a positive number $\rho$ such that $p(y)>0$ if $0<y\le \rho$ and
$p(y)<0$ if $-\rho\le y<0$. For $0<r\le\rho$ let
$$
\alpha(r)=\min\left\{\int_0^r p(x)\,dx,\ \int_{-r}^0
|p(x)|\,dx\right\}
\mbox{\quad and \quad}
\beta(r)=\max\left\{\int_0^r p(x)\,dx,\ \int_{-r}^0
|p(x)|\,dx\right\}.
$$
Let $y$ be the solution of the initial value problem
$$
y''+p(y)=0,\quad y(0)=v_0,\quad y'(0)=v_0,
$$
and define
$c(y_0,v_0)=v_0^2+2\int_0^{y_0}p(x)\,dx$.
\begin{enumerate}
\item % (a)
Show that
$$
0<c(y_0,v_0) <v_0^2+2\beta(|y_0|)\mbox{\quad if \quad} 0<|y_0|\le\rho.
$$
\item\label{partB:4.4.23} % (b)
Show that
$$
v^2+2\int_0^y p(x)\,dx=c(y_0,v_0),\quad t>0.
$$
\item % (c)
Conclude from \ref{partB:4.4.23} that if $c(y_0,v_0)<2\alpha(r)$ then $|y|<r,\,t>0$.
\item % (d)
Given $\epsilon>0$, let  $\delta>0$ be chosen so that
$$
\delta^2+2\beta(\delta)<\max\left\{\epsilon^2/2,2\alpha(\epsilon/\sqrt2)
\right\}.
$$
Show that if $\sqrt{y_0^2+v_0^2}<\delta$ then
$\sqrt{y^2+v^2}<\epsilon$ for $t>0$, which implies that $\overline
  y=0$ is a stable equilibrium of $y''+p(y)=0$.
\item % (e)
Now let $p$
be continuous for all $y$ and $p(\overline y)=0$, where $\overline y$
is not necessarily zero. Suppose there's a positive number
$\rho$ such that $p(y)>0$ if $\overline y<y\le \overline y+\rho$ and
$p(y)<0$ if $\overline y-\rho\le y<\overline y$. Show that $\overline
y$ is a stable equilibrium of $y''+p(y)=0$.
\end{enumerate}
\end{problem}

\begin{problem}\label{exer:4.4.24}
Let $p$ be continuous for all $y$.
\begin{enumerate}
\item\label{partA:4.4.24} % (a)
Suppose $p(0)=0$ and there's a positive number $\rho$ such that
$p(y)<0$ if  $0<y\le \rho$. Let $\epsilon$ be any
number such that
$0<\epsilon<\rho$.
Show that if $y$ is the solution of the initial value problem
$$
y''+p(y)=0,\quad y(0)=y_0,\quad y'(0)=0
$$
with $0<y_0<\epsilon$, then
$y(t)\ge\epsilon$ for some
$t>0$. Conclude that $\overline y=0$ is an unstable equilibrium of
$y''+p(y)=0$.
\begin{hint}
   Let $k=\min_{y_0\le
x\le\epsilon}\left(-p(x)\right)$,
which is positive. Show that if $y(t)<\epsilon$ for $0\le t<T$ then
$kT^2<2(\epsilon-y_0)$. 
\end{hint}

\begin{solution}
Since $v'=-p(y)\ge k$ and $v(0)=0$, $v\ge kt$ and therefore
$y\ge y_0+kt^2/2$ for $0\le t<T$.
\end{solution}

\item\label{partB:4.4.24} % (b)
Now let $p(\overline y)=0$, where $\overline y$ isn't  necessarily
zero. Suppose there's a positive number $\rho$ such that
$p(y)<0$ if $\overline y<y\le \overline y+\rho$. Show that $\overline
y$ is an unstable equilibrium of $y''+p(y)=0$.

\begin{solution}
Let $0<\epsilon<\rho$.
Suppose that $y$ is the solution of the initial value problem
(A) $y''+p(y)=0,\quad y(0)=y_0,\quad y'(0)=0$, where $\overline
y<y_0<\overline y+\epsilon$. Now let $Y=y-\overline y$ and
$P(Y)=p(Y+\overline y)$. Then $P(0)=0$ and $P(Y)<0$ if
$0<Y\le\rho$ . Moreover, $Y$ is the solution of
 $Y''+p(Y)=0,\quad Y(0)=Y_0,\quad Y''(0)=0$, where $Y_0=y_0-\overline
y$, so $0<Y_0<\epsilon$. From \ref{partA:4.4.24}, $Y(t)\ge\epsilon$ for some
$t>0$. Therefore,$y(t)>\overline y+\epsilon$ for some $t>0$,
so $\overline y$ is an unstable equilibrium of $y''+p(y)=0$.
\end{solution}

\item % (c)
Modify your proofs above to show that if there is a
positive number $\rho$ such that $p(y)>0$ if $\overline y-\rho\le y<\overline
y$, then $\overline y$ is an unstable equilibrium of $y''+p(y)=0$.
\end{enumerate}
\end{problem}

\end{document}