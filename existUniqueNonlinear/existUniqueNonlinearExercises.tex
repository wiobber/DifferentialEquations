\documentclass{ximera}
\input{../preamble.tex}

\title{Exercises} \license{CC BY-NC-SA 4.0}

\begin{document}

\begin{abstract}
\end{abstract}
\maketitle

\begin{onlineOnly}
\section*{Exercises}
\end{onlineOnly}

\begin{problem}\label{exer:2.3.1} 
Find all $(x_0,y_0)$ for
which Theorem~\ref{thmtype:2.3.1}
implies that the initial value problem $y'=f(x,y),\,  y(x_0)=y_0$ has
(a) a solution, (b) a  unique solution on some open interval that contains $x_0$.

$$y'=\frac{x^2+y^2}{\sin x}$$
\end{problem}

\begin{problem}\label{exer:2.3.2} Find all $(x_0,y_0)$ for
which Theorem~\ref{thmtype:2.3.1}
implies that the initial value problem $y'=f(x,y),\,  y(x_0)=y_0$ has
(a) a solution, (b) a  unique solution on some open interval that contains $x_0$.

$$y'=\frac{e^x+y}{x^2+y^2}$$ 



\begin{solution}
    $f(x,y)=\frac{e^x+y}{ x^2+y^2}$ and $f_y(x,y)=\frac{1}{
x^2+y^2}-\frac{2y(e^x+y)}{(x^2+y^2)^2}$ are both continuous at all
$(x,y)\ne(0,0)$. Hence,  Theorem~2.3.1 implies that if
$(x_0,y_0)\ne(0,0)$, then  the initial
value problem has a a unique solution on some open interval containing
$x_0$.  Theorem~\ref{thmtype:2.3.1} does not apply
if $(x_0,y_0)=(0,0)$.
\end{solution}
\end{problem}

\begin{problem}\label{exer:2.3.3} Find all $(x_0,y_0)$ for
which Theorem~\ref{thmtype:2.3.1}
implies that the initial value problem $y'=f(x,y),\,  y(x_0)=y_0$ has
(a) a solution, (b) a  unique solution on some open interval that contains $x_0$.

$$y'= \tan xy$$
\end{problem}

\begin{problem}\label{exer:2.3.4} Find all $(x_0,y_0)$ for
which Theorem~\ref{thmtype:2.3.1}
implies that the initial value problem $y'=f(x,y),\,  y(x_0)=y_0$ has
(a) a solution, (b) a  unique solution on some open interval that contains $x_0$.

$$y'=\frac{x^2+y^2}{\ln xy}$$ 



\begin{solution}
    $f(x,y)=\frac{x^2+y^2}{\ln xy}$ and $f_y(x,y)=\frac{2y}{\ln
xy}-\frac{x^2+y^2}{ x(\ln xy)^2}$ are both continuous at all $(x,y)$
such that $xy>0$ and $xy\ne1$. Hence,  Theorem~2.3.1 implies
that if
$x_0y_0>0$ and $x_0y_0\ne1$, then the
initial value problem has unique solution on an open interval
containing $x_0$.   Theorem~\ref{thmtype:2.3.1} does not apply
if $x_0y_0\le0$ or $x_0y_0=1$.
\end{solution}
\end{problem}

\begin{problem}\label{exer:2.3.5} Find all $(x_0,y_0)$ for
which Theorem~\ref{thmtype:2.3.1}
implies that the initial value problem $y'=f(x,y),\,  y(x_0)=y_0$ has
(a) a solution, (b) a  unique solution on some open interval that contains $x_0$.

$$y'=(x^2+y^2)y^{1/3}$$ 
\end{problem}

\begin{problem}\label{exer:2.3.6} Find all $(x_0,y_0)$ for
which Theorem~\ref{thmtype:2.3.1}
implies that the initial value problem $y'=f(x,y),\,  y(x_0)=y_0$ has
(a) a solution, (b) a  unique solution on some open interval that contains $x_0$.

$$y'=2xy$$ 



\begin{solution}
    $f(x,y)=2xy$ and $f_y(x,y)=2x$ are both continuous at all $(x,y)$.
Hence,  Theorem~\ref{thmtype:2.3.1} implies that if $(x_0,y_0)$
is arbitrary, then the initial value problem
has a  unique solution on some open interval containing  $x_0$.
\end{solution}
\end{problem}

\begin{problem}\label{exer:2.3.7} Find all $(x_0,y_0)$ for
which Theorem~\ref{thmtype:2.3.1}
implies that the initial value problem $y'=f(x,y),\,  y(x_0)=y_0$ has
(a) a solution, (b) a  unique solution on some open interval that contains $x_0$.

$$y'=\ln(1+x^2+y^2)$$
\end{problem}

\begin{problem}\label{exer:2.3.8} Find all $(x_0,y_0)$ for
which Theorem~\ref{thmtype:2.3.1}
implies that the initial value problem $y'=f(x,y),\,  y(x_0)=y_0$ has
(a) a solution, (b) a  unique solution on some open interval that contains $x_0$.

$$y'=\frac{2x+3y}{x-4y}$$ 



\begin{solution}
    $f(x,y)=\frac{2x+3y}{ x-4y}$ and $f_y(x,y)=\frac{3}{
x-4y}+4\frac{2x+3y}{(x-4y)^2}$ are both continuous at all $(x,y)$
such that $x\ne4y$. Hence,  Theorem~\ref{thmtype:2.3.1} implies that if
$x_0\neq 4y_0$, then the
initial value problem has a  unique solution on some open interval
containing  $x_0$.
  Theorem~\ref{thmtype:2.3.1} does not apply if $x_0=4y_0$.
\end{solution}
\end{problem}

\begin{problem}\label{exer:2.3.9} Find all $(x_0,y_0)$ for
which Theorem~\ref{thmtype:2.3.1}
implies that the initial value problem $y'=f(x,y),\,  y(x_0)=y_0$ has
(a) a solution, (b) a  unique solution on some open interval that contains $x_0$.

$$y'=(x^2+y^2)^{1/2}$$ 
\end{problem}

\begin{problem}\label{exer:2.3.10} Find all $(x_0,y_0)$ for
which Theorem~\ref{thmtype:2.3.1}
implies that the initial value problem $y'=f(x,y),\,  y(x_0)=y_0$ has
(a) a solution, (b) a  unique solution on some open interval that contains $x_0$.  

$$y'=x(y^2-1)^{2/3}$$ 



\begin{solution}
    $f(x,y)=x(y^2-1)^{2/3}$ is continuous at all $(x,y)$, but
$f_y(x,y)=\frac{4}{3}xy(y^2-1)^{1/3}$ is continuous at $(x,y)$ if
and only if $y\neq\pm1$. Hence,  Theorem~\ref{thmtype:2.3.1} implies that
if $y_0\neq\pm1$, then the
initial value problem has a  unique solution on some open interval
containing  $x_0$, while  if $y_0=\pm1$, then the initial value problem
 has at least one solution (possibly not unique on any
open interval containing  $x_0$).
\end{solution}
\end{problem}

\begin{problem}\label{exer:2.3.11} Find all $(x_0,y_0)$ for
which Theorem~\ref{thmtype:2.3.1}
implies that the initial value problem $y'=f(x,y),\,  y(x_0)=y_0$ has
(a) a solution, (b) a  unique solution on some open interval that contains $x_0$.

$$y'=(x^2+y^2)^2$$
\end{problem}

\begin{problem}\label{exer:2.3.12} Find all $(x_0,y_0)$ for
which Theorem~\ref{thmtype:2.3.1}
implies that the initial value problem $y'=f(x,y),\,  y(x_0)=y_0$ has
(a) a solution, (b) a  unique solution on some open interval that contains $x_0$.

$$y'=(x+y)^{1/2}$$



\begin{solution}
    $f(x,y)=(x+y)^{1/2}$ and $f_y(x,y)=\frac{1}{2(x+y)^{1/2}}$ are both
continuous at all $(x,y)$ such that $x+y>0$ Hence,
 Theorem~\ref{thmtype:2.3.1} implies that if $x_0+y_0>0$, then the initial
value problem has a
 unique solution on some open interval containing  $x_0$.
 Theorem~\ref{thmtype:2.3.1} does not apply if $x_0+y_0\leq 0$.
\end{solution}
\end{problem}


\begin{problem}\label{exer:2.3.13}Find all $(x_0,y_0)$ for
which Theorem~\ref{thmtype:2.3.1}
implies that the initial value problem $y'=f(x,y),\,  y(x_0)=y_0$ has
(a) a solution, (b) a  unique solution on some open interval that contains $x_0$.

$$y'=\frac{\tan y}{x-1}$$
\end{problem}

\begin{problem}\label{exer:2.3.14}
Apply Theorem~\ref{thmtype:2.3.1} to the initial value problem
$$
y'+p(x)y = q(x), \quad y(x_0)=y_0
$$
for a linear equation, and compare the conclusions that can be drawn
from it to those that follow from Theorem~\ref{thmtype:2.1.2}.



\begin{solution}
    To apply  Theorem~\ref{thmtype:2.3.1}, rewrite the given initial value problem
as (A) $y'=f(x,y),\, y(x_0)=y_0$, where $f(x,y)=-p(x)y+q(x)$ and
$f_y(x,y)=-p(x)$. If $p$ and $f$ are continuous on some open interval
$(a,b)$ containing $x_0$, then $f$ and $f_y$ are continuous on some
open
rectangle containing $(x_0,y_0)$, so  Theorem~2.3.1 implies
that (A) has a unique solution  on some open
interval
containing $x_0$. The conclusion of Theorem~\ref{thmtype:2.1.2} is more
specific:
the solution of (A) exists and is unique on $(a,b)$. For example, in
the extreme case where $(a,b)=(-\infty,\infty)$,  Theorem~\ref{thmtype:2.3.1}
still implies only existence and uniqueness on some open
interval containing $x_0$, while  Theorem~\ref{thmtype:2.1.2} implies that the
solution exists and is unique on $(-\infty,\infty)$.
\end{solution}
\end{problem}

\begin{problem}\label{exer:2.3.15}
\begin{enumerate}
\item\label{partA:2.3.15} %(a)
Verify that the function
$$
y = \left\{ \begin{array}{cl}
(x^2-1)^{5/3}, & -1 < x < 1, \\
0, & |x| \geq 1, \end{array} \right.
$$
is a solution of the initial value problem
$$
y'=\frac{10}{3}xy^{2/5}, \quad y(0)=-1
$$
on $(-\infty,\infty)$. 
\begin{hint}
You'll need the definition $$
y'(\overline{x}) = \lim_{x \to \overline{x}} \frac{y(x)-y(\overline{x})}{x-\overline{x}} $$ to verify that $y$ satisfies the differential
equation at $\overline{x} = \pm 1$.
\end{hint}

\item %(b)
Verify that if $\epsilon_i=0$ or $1$ for $i=1$, $2$ and $a$, $b>1$, then
the function
$$
y = \left\{ \begin{array}{cl}
\epsilon_1(x^2-a^2)^{5/3}, & - \infty < x < -a, \\
0, & -a \leq x \leq -1, \\
(x^2-1)^{5/3}, &  -1 < x < 1, \\
0, & 1 \leq x \leq b, \\
\epsilon_2(x^2-b^2)^{5/3}, & b < x < \infty 
\end{array} \right.
$$
is a solution of the initial value problem in part \ref{partA:2.3.15} on
$(-\infty,\infty)$.
\end{enumerate}
\end{problem}

\begin{problem}\label{exer:2.3.16}
Use the ideas developed in Exercise~\ref{exer:2.3.15} to find
infinitely many solutions of the initial value problem
$$
y'=y^{2/5}, \quad y(0)=1
$$
on $(-\infty,\infty)$.



\begin{solution}
    First find solutions of (A) $y'=y^{2/5}$. Obviously $y\equiv0$ is a
solution. If $y\not\equiv0$, then we can separate variables on any
open interval where $y$ has no zeros: $y^{-2/5}y'=1$;\;
$\frac{5}{3}y^{3/5}=x+c$;\;
$y=\left(\frac{3}{5}(x+c)^{5/3}\right)$. (Note that this solution
is also defined  at $x=-c$, even though  $y(-c)=0$. To satisfy the
initial condition, let $c=1$. Thus,
$y=\left(\frac{3}{5}(x+1)^{5/3}\right)$ is a solution of the
initial value problem on $(-\infty,\infty)$; moreover,
since $f(x,y)=y^{2/5}$ and $f_y(x,y)=\frac{2}{5}y^{-3/5}$
are both continuous  at all $(x,y)$ such that $y\neq 0$,
this is the only
solution on $(-5/3,\infty)$, by an argument similar to that given in
Example~2.3.7, the function
$$
y =
\left\{
\begin{array}{cl}
0, &  -\infty< x \le -\frac{5}{ 3} \\
\left(\frac{3}{ 5}x + 1\right)^{5/3}, & -\frac{5}{3} < x < \infty
\end{array}\right.
$$
(To see that $y$ satisfies $y'=y^{2/5}$  at $x=-\frac{5}{3}$
use an argument similar to that of
Discussion~2.3.15-2)
For every $a \geq \frac{5}{ 3}$, the following function is also a
solution:
$$
y =
\left\{
\begin{array}{cl}
\left( \frac{3}{ 5} (x+a) \right)^{5/3}, & - \infty < x < -a, \\
0, &  -a \le x \le -\frac{5}{ 3} \\
\left(\frac{3}{ 5}x + 1\right)^{5/3}, & -\frac{5}{3} < x < \infty.
\end{array}\right.
$$
\end{solution}
\end{problem}

\begin{problem}\label{exer:2.3.17}
 Consider the initial value problem
\begin{equation}\label{eqA:2.3.17}
 y' = 3x(y-1)^{1/3}, \quad y(x_0) = y_0.
\end{equation}

\begin{enumerate}
\item %(a)
For what points $(x_0,y_0)$ does Theorem~\ref{thmtype:2.3.1} imply that
(\ref{eqA:2.3.17}) has a solution?
\item %(b)
 For what points $(x_0,y_0)$ does Theorem~\ref{thmtype:2.3.1} imply that
(\ref{eqA:2.3.17}) has a  unique solution on some open interval
that contains  $x_0$?
\end{enumerate}
\end{problem}

\begin{problem}\label{exer:2.3.18}
Find nine solutions of the initial value problem
$$
y'=3x(y-1)^{1/3}, \quad y(0)=1
$$
that are all defined on $(-\infty,\infty)$ and differ from each other
for values of  $x$ in every open interval that contains $x_0=0$.



\begin{solution}
    Obviously, $y_1\equiv1$ is a solution.
From Discussion~2.3.18 (taking $c=0$
in the two families of solutions) yields
$y_2=1+|x|^3$ and $y_3=1-|x|^3$. Other solutions are
$y_4=1+x^3$, $y_5=1-x^3$,
$$
y_6=\left\{\begin{array}{ccl}1+x^3,&x\geq 0,\\
1,&x<0\end{array}\right.;\quad
y_7=\left\{\begin{array}{ccl}1-x^3,&x\geq 0,\\
1,&x<0\end{array}\right.;
$$
$$
y_8=\left\{\begin{array}{ccl}1,&x\geq 0,\\
1+x^3,&x<0\end{array}\right.;\quad
y_9=\left\{\begin{array}{ccl}1,&x\geq 0,\\
1-x^3,&x<0\end{array}\right.
$$


It is straightforward to verify that all these functions satisfy
$y'=3x(y-1)^{1/3}$  for all $x\ne0$. Moreover,
 $y_i'(0)=\lim_{x\to0}\frac{y_i(x)-1}{x}=0$ for
$1\leq i\leq 9$, which implies that they also satisfy the equation at
$x=0$.
\end{solution}
\end{problem}

\begin{problem}\label{exer:2.3.19}  From Theorem~\ref{thmtype:2.3.1}, the initial value problem
 $$
y'=3x(y-1)^{1/3}, \quad y(0)=9
$$
has a unique solution on an open interval that contains  $x_0=0$. Find the
solution and determine the largest open interval on which it's unique.
\end{problem}

\begin{problem}\label{exer:2.3.20}
\begin{enumerate}
\item %(a)
 From Theorem~\ref{thmtype:2.3.1},
the initial value problem
\begin{equation}\label{eqA:2.3.20}
 y'=3x(y-1)^{1/3}, \quad y(3)=-7
\end{equation}
 has a unique solution on some open interval that contains $x_0=3$.
Determine
the largest such open interval, and find the solution on this interval.
\item %(b)
 Find infinitely many solutions of (\ref{eqA:2.3.20}), all defined
on
$(-\infty,\infty)$.
\end{enumerate}



\begin{solution}
    Let $y$ be any solution of (A) $y'=3x(y-1)^{1/3},\ y(3)=-7$.
By continuity, there is some open interval $I$ containing  $x_0=3$
on which $y(x)<1$. From
Discussion~2.3.18,
 $y=1+(x^2+c)^{3/2}$ on $I$;\ $y(3)=-7\Rightarrow c=-5$;\;
(B) $y=1-(x^2-5)^{3/2}$.  It now follows that every solution of (A)
satisfies $y(x)<1$
and is given by (B) on $(\sqrt5,\infty)$; that is,
(B) is the unique solution of (A) on
$(\sqrt5,\infty)$.
 This solution can be extended uniquely to
$(0,\infty)$ as
$$
y=
\left\{
\begin{array}{cl}
1, & 0< x\le\sqrt5,\\
1-(x^2-5)^{3/2}, & \sqrt5<x<\infty
\end{array}\right.
$$
It can be extended to $(-\infty,\infty)$ in infinitely many ways.
Thus,
$$
y= \left\{
\begin{array}{cl}
 1, &-\infty< x\le\sqrt5,\\
1-(x^2-5)^{3/2},& \sqrt5<x<\infty
\end{array} \right.$$
is a solution of the initial value problem on $(-\infty,\infty)$.
Moroever, if
$\alpha\geq 0$, then
$$
y= \left\{
\begin{array}{cl}
 1+(x^2-\alpha^2)^{3/2},& -\infty<x<-\alpha, \\
1, & -\alpha\leq x\le\sqrt5,\\
1-(x^2-5)^{3/2},& \sqrt5<x<\infty,
\end{array}\right. $$
and
$$
y= \left\{
\begin{array}{cl}
 1-(x^2-\alpha^2)^{3/2},& -\infty<x<-\alpha, \\
1, & -\alpha\leq x\leq\sqrt5,\\
1-(x^2-5)^{3/2},& \sqrt5<x<\infty,
\end{array} \right.
$$
are also solutions of the initial value problem  on
$(-\infty,\infty)$.
\end{solution}
\end{problem}

\begin{problem}\label{exer:2.3.21}
Prove:
\begin{enumerate}
\item % ()
If
\begin{equation}\label{eqA:2.3.21}
f(x,y_0) = 0,\quad a<x<b,
\end{equation}
and  $x_0$ is in  $(a,b)$,
then $y\equiv y_0$ is a solution of
$$
y'=f(x,y), \quad y(x_0)=y_0
$$
on $(a,b)$.
\item % (b)
If $f$ and $f_y$ are continuous on an open rectangle
that contains $(x_0,y_0)$ and (\ref{eqA:2.3.21}) holds, no
solution of $y'=f(x,y)$ other than $y\equiv y_0$ can equal $y_0$ at
any point in $(a,b)$.
\end{enumerate}
\end{problem}


\end{document}