\documentclass{ximera}
\input{../preamble.tex}

\title{Exercises} \license{CC BY-NC-SA 4.0}

\begin{document}

\begin{abstract}
\end{abstract}
\maketitle

\begin{onlineOnly}
\section*{Exercises}
\end{onlineOnly}


\begin{problem}\label{exer:9.2.1}  Find the general solution.

$$y'''-3y''+3y'-y=0$$
\end{problem}

\begin{problem}\label{exer:9.2.2} Find the general solution.

$$y^{(4)}+8y''-9y=0$$

\begin{solution}
$p(r)=r^4+8r^2-9=(r-1)(r+1)(r^2+9)$;
 $y=c_1e^x+c_2e^{-x}+c_3\cos3x+c_4\sin3x$.
\end{solution}

\end{problem}

\begin{problem}\label{exer:9.2.3} Find the general solution.

$$y'''-y''+16y'-16y=0$$

\end{problem}


\begin{problem}\label{exer:9.2.4} Find the general solution.

$$2y'''+3y''-2y'-3y=0$$

\begin{solution}
$p(r)=2r^3+3r^2-2r-3=(r-1)(r+1)(2r+3)$;
$y=c_1e^x+c_2e^{-x}+c_3e^{-3x/2}$.
\end{solution}
\end{problem}

\begin{problem}\label{exer:9.2.5} Find the general solution.

$$y'''+5y''+9y'+5y=0$$

\end{problem}


\begin{problem}\label{exer:9.2.6}  Find the general solution.

$$4y'''-8y''+5y'-y=0$$

\begin{solution}
$p(r)=4r^3-8r^2+5r-1=(r-1)(2r-1)^2$;

$y=c_1e^x+e^{x/2}(c_2+c_3x)$.
\end{solution}

\end{problem}

\begin{problem}\label{exer:9.2.7} Find the general solution.

$$27y'''+27y''+9y'+y=0$$

\end{problem}


\begin{problem}\label{exer:9.2.8} Find the general solution.

$$y^{(4)}+y''=0$$

\begin{solution}
$p(r)=r^4+r^2=r^2(r^2+1)$;
 $y=c_1+c_2x+c_3\cos x+c_4\sin x$.
\end{solution}

\end{problem}

 \begin{problem}\label{exer:9.2.9}  Find the general solution.

$$y^{(4)}-16y=0$$

\end{problem}


\begin{problem}\label{exer:9.2.10} Find the general solution.

$$y^{(4)}+12y''+36y=0$$

\begin{solution}
$p(r)=r^4+12r^2+36=(r^2+6)^2$;
 $y=(c_1+c_2x)\cos\sqrt{6} x+(c_3+c_4x)\sin\sqrt{6} x$.
\end{solution}
\end{problem}



\begin{problem}\label{exer:9.2.11} Find the general solution.

$$16y^{(4)}-72y''+81y=0$$

\end{problem}

\begin{problem}\label{exer:9.2.12} Find the general solution.

$$6y^{(4)}+5y'''+7y''+5y'+y=0$$

\begin{solution}
$p(r)=6r^4+5r^3+7r^2+5r+1=(2r+1)(3r+1)(r^2+1)$;
$y=c_1e^{-x/2}+c_2e^{-x/3}+c_3\cos x+c_4\sin x$.
\end{solution}
\end{problem}


\begin{problem}\label{exer:9.2.13}  Find the general solution.

$$4y^{(4)}+12y'''+3y''-13y'-6y=0$$
\end{problem}


\begin{problem}\label{exer:9.2.14}  Find the general solution.

$$y^{(4)}-4y'''+7y''-6y'+2y=0$$

\begin{solution}
$p(r)=r^4-4r^3+7r^2-6r+2=(r-1)^2(r^2-2r+2)$;
$y=e^x(c_1+c_2x+c_3\cos x+c_4\sin x)$.
\end{solution}
\end{problem}


\begin{problem}\label{exer:9.2.15}
Solve the
initial value problem.

$y'''-2y''+4y'-8y=0, \quad  y(0)=2,\quad y'(0)=-2,\;  y''(0)=0$
\end{problem}

\begin{problem}\label{exer:9.2.16}  Solve the
initial value problem.

$y'''+3y''-y'-3y=0, \quad  y(0)=0,\quad y'(0)=14,\quad y''(0)=-40$

\begin{solution}
$p(r)=r^3+3r^2-r-3=(r-1)(r+1)(r+3)$;
$$
\begin{array}{lcl}
y&=&c_1e^x+c_2e^{-x}+c_3e^{-3x}\\
y'&=&c_1e^x-c_2e^{-x}-3c_3e^{-3x}\\
y''&=&c_1e^x+c_2e^{-x}+9c_3e^{-3x}
\end{array};\quad
\begin{array}{rcr}
c_1+c_2+c_3&=&0\\
c_1-c_2-3c_3&=&14\\
c_1+c_2+9c_3&=&-40
\end{array};
$$
$c_1=2$, $c_2=3$, $c_3=-5$;

 $$y=2e^x+3e^{-x}-5e^{-3x}$$.
\end{solution}
\end{problem}

\begin{problem}\label{exer:9.2.17} Solve the
initial value problem, and graph the solution.
$y'''-y''-y'+y=0, \quad  y(0)=-2,\quad y'(0)=9,\quad y''(0)=4$
\end{problem}

\begin{problem}\label{exer:9.2.18} Solve the
initial value problem, and graph the solution.

$y'''-2y'-4y=0, \quad  y(0)=6,\quad y'(0)=3,\quad y''(0)=22$

\begin{solution}
$p(r)=r^3-2r-4=(r-2)(r^2+2r+2)$;
$$
\begin{array}{lcl}
y&=&e^{-x}(c_1\cos x+c_2\sin x)+c_3e^{2x}\\
y'&=&-e^{-x}((c_1-c_2)\cos x+(c_1+c_2)\sin x)+2c_3e^{2x}\\
y''&=&e^{-x}(2c_1\sin x-2c_2\cos x)+4c_3e^{2x}
\end{array};
\begin{array}{rcr}
c_1+c_3&=&6\\
-c_1+c_2+2c_3&=&3\\
-2c_2+4c_3&=&22
\end{array};
$$
$c_1=2$, $c_2=-3$, $c_3=4$;
 $y=2e^{-x}\cos x-3e^{-x}\sin x+4e^{2x}$.
\end{solution}
\end{problem}

\begin{problem}\label{exer:9.2.19} Solve the
initial value problem, and graph the solution.

$3y'''-y''-7y'+5y=0, \quad  y(0)=\frac{14}{5},\quad y'(0)=0,\quad y''(0)=10$
\end{problem}

\begin{problem}\label{exer:9.2.20}  Solve the
initial value problem.

$y'''-6y''+12y'-8y=0, \quad  y(0)=1,\quad y'(0)=-1,\quad y''(0)=-4$

\begin{solution}
$p(r)=r^3-6r^2+12r-8=(r-2)^3$;
$$
\begin{array}{lcl}
y&=&e^{2x}(c_1+c_2x+c_3x^2)\\
y'&=&e^{2x}(2c_1+c_2+(2c_2+2c_3)x+2c_3x^2)\\
y''&=&2e^{2x}(2c_1+2c_2+c_3+2(c_2+2c_3)x+2c_3x^2)
\end{array};
\begin{array}{rcr}
c_1&=&1\\
2c_1+c_2&=&-1\\
4c_1+4c_2+2c_3&=&-4
\end{array}
$$
$c_1=1$, $c_2=-3$, $c_3=2$;
 $y=e^{2x}(1-3x+2x^2)$.
\end{solution}
\end{problem}

\begin{problem}\label{exer:9.2.21}  Solve the
initial value problem.

$2y'''-11y''+12y'+9y=0, \quad  y(0)=6,\quad y'(0)=3,\quad y''(0)=13$
\end{problem}

\begin{problem}\label{exer:9.2.22}  Solve the
initial value problem.

$8y'''-4y''-2y'+y=0, \quad  y(0)=4,\quad y'(0)=-3,\quad y''(0)=-1$

\begin{solution}
$p(r)=8r^3-4r^2-2r+1=(2r+1)(2r-1)^2$;
$$
\begin{array}{lcl}
y&=&e^{x/2}(c_1+c_2x)+c_3e^{-x/2}\\
y'&=&\frac{1}{2}e^{x/2}(c_1+2c_2+c_2x)-\frac{1}{2}c_3e^{-x/2}\\
y''&=&\frac{1}{4}e^{x/2}(c_1+4c_2+c_2x)+\frac{1}{4}c_3e^{-x/2}
\end{array};
\begin{array}{rcr}
c_1+c_3&=&4\\
\frac{1}{2}c_1+c_2-\frac{1}{2}c_3&=&-3\\
\frac{1}{4}c_1+c_2+\frac{1}{4}c_3&=&-1
\end{array};
$$
$c_1=1$, $c_2=-2$, $c_3=3$;
 $y=e^{x/2}(1-2x)+3e^{-x/2}$.
\end{solution}
\end{problem}

\begin{problem}\label{exer:9.2.23}  Solve the
initial value problem.

$y^{(4)}-16y=0, \quad   y(0)=2,\;  y'(0)=2,\;  y''(0)=-2,\;  y'''(0)=0$
\end{problem}

\begin{problem}\label{exer:9.2.24}  Solve the
initial value problem.

$y^{(4)}-6y'''+7y''+6y'-8y=0, \quad  y(0)=-2,\quad y'(0)=-8,\quad y''(0)=-14$,
 $y'''(0)=-62$

\begin{solution}
$p(r)=r^4-6r^3+7r^2+6r-8=(r-1)(r-2)(r-4)(r+1)$;
$$
\begin{array}{lcl}
y&=&c_1e^x+c_2e^{2x}+c_3e^{4x}+c_4e^{-x}\\
y'&=&c_1e^x+2c_2e^{2x}+4c_3e^{4x}-c_4e^{-x}\\
y''&=&c_1e^x+4c_2e^{2x}+16c_3e^{4x}+c_4e^{-x}\\
y'''&=&c_1e^x+8c_2e^{2x}+64c_3e^{4x}-c_4e^{-x}
\end{array};
\begin{array}{rcr}
c_1+c_2+c_3+c_4&=&-2\\
c_1+2c_2+4c_3-c_4&=&-8\\
c_1+4c_2+16c_3+c_4&=&-14\\
c_1+8c_2+64c_3-c_4&=&-62
\end{array};
$$
$c_1=-4$, $c_2=1$, $c_3=-1$, $c_4=2$;
 $y=-4e^x+e^{2x}-e^{4x}+2e^{-x}$.
\end{solution}
\end{problem}

\begin{problem}\label{exer:9.2.25}  Solve the
initial value problem.

$4y^{(4)}-13y''+9y=0, \quad  y(0)=1,\quad y'(0)=3,\quad y''(0)=1,\quad y'''(0)=3$
\end{problem}

\begin{problem}\label{exer:9.2.26}  Solve the
initial value problem.

$y^{(4)}+2y'''-2y''-8y'-8y=0, \quad  y(0)=5,\quad y'(0)=-2,\quad y''(0)=6,\quad y'''(0)=8$

\begin{solution}
$p(r)=r^4+2r^3-2r^2-8r-8=(r-2)(r+2)(r^2+2r+2)$;
$$
\begin{array}{lcl}
y&=&c_1e^{2x}+c_2e^{-2x}+e^{-x}(c_3\cos x+c_4\sin x)\\
y'&=&2c_1e^{2x}-2c_2e^{-2x}-e^{-x}((c_3-c_4)\cos x+(c_3+c_4)\sin x)\\
y''&=&4c_1e^{2x}+4c_2e^{-2x}+e^{-x}(2c_3\sin x-2c_4\cos x)\\
y'''&=&8c_1e^{2x}-8c_2e^{-2x}+e^{-x}((2c_3+2c_4)\cos x+2(c_4-c_3)\sin
x)
\end{array};
$$
$$
\begin{array}{rcr}
c_1+c_2+c_3&=&5\\
2c_1-2c_2-c_3+c_4&=&-2\\
4c_1+4c_2-2c_4&=&6\\
8c_1-8c_2+2c_3+2c_4&=&8
\end{array};
$$
$c_1=1$, $c_2=1$, $c_3=3$, $c_4=1$;

$$y=e^{2x}+e^{-2x}+e^{-x}(3\cos x+\sin x)$$.

\end{solution}
\end{problem}

\begin{problem}\label{exer:9.2.27}  Solve the
initial value problem, and graph the solution.
$4y^{(4)}+8y'''+19y''+32y'+12y=0, \quad  y(0)=3,\quad y'(0)=-3,\quad y''(0)=
-\frac{7}{2}$,
 $y'''(0)=\frac{31}{4}$
\end{problem}

\begin{problem}\label{exer:9.2.28}
Find a fundamental set of solutions of the given equation, and verify that it is a fundamental set by evaluating its Wronskian at $x=0$.

\begin{enumerate}
    \item $(D-1)^2(D-2)y=0$

\begin{solution}
$
W(x)=\left|\begin{array}{ccc}
e^x&xe^x&e^{2x}\\
e^x&e^x(x+1)&2e^{2x}\\
e^x&e^x(x+2)&4e^{2x}
\end{array}\right|;
$
$
W(0)=\left|\begin{array}{cccc}
1&0&1\\1&1&2\\1&2&4
\end{array}\right|=
\left|\begin{array}{cccc}
1&0&0\\1&1&1\\1&2&3
\end{array}\right|
=\left|\begin{array}{cccc}
1&1\\2&3\end{array}\right|=1.
$
\end{solution}

    \item $(D^2+4)(D-3)y=0$

\begin{solution}
$
W(x)=\left|\begin{array}{ccc}
\cos 2x&\sin 2x&e^{3x}\\
-2\sin 2x&2\cos 2x&3e^{3x}\\
-4\cos 2x&-4\sin 2x&9e^{3x}
\end{array}\right|;
$
$
W(0)=\left|\begin{array}{rccc}
1&0&1\\0&2&3\\-4&0&9
\end{array}\right|=
\left|\begin{array}{rccc}
1&0&0\\0&2&3\\-4&0&13
\end{array}\right|=
\left|\begin{array}{rccc}
2&3\\0&13\end{array}\right|=26.
$
\end{solution}

    \item $(D^2+2D+2)(D-1)y=0$

\begin{solution}
$
W(x)=\left|\begin{array}{ccc}
e^{-x}\cos x&e^{-x}\sin x&e^x\\
-e^{-x}(\cos x+\sin x)&e^{-x}(\cos x-\sin x)&e^x\\
2e^{-x}\sin x&-2e^{-x}\cos x&
e^x\end{array}\right|;
$
$
W(0)=\left|\begin{array}{rrc}
1&0&1\\-1&1&1\\0&-2&1
\end{array}\right|=
\left|\begin{array}{rrc}
1&0&1\\0&1&2\\0&-2&1
\end{array}\right|=
\left|\begin{array}{rcc}
1&2\\-2&1\end{array}\right|=5.
$
\end{solution}

    \item $D^3(D-1)y=0$

\begin{solution}
$
W(x)=\left|\begin{array}{cccc}
1&x&x^2&e^x\\
0&1&2x&e^x\\0&0&2&e^x\\0&0&0&e^x\end{array}\right|;
$
$
W(0)=\left|\begin{array}{cccc}
1&0&0&1\\0&1&0&1\\0&0&2&1\\0&0&0&1
\end{array}\right|=1.
$
\end{solution}

    \item $(D^2-1)(D^2+1)y=0$

\begin{solution}
$
W(x)=\left|\begin{array}{rrrr}e^x&e^{-x}&\cos x&\sin x\\
e^x&-e^{-x}&-\sin x&\cos x\\e^x&e^{-x}&-\cos x&-\sin
x\\e^x&-e^{-x}&\sin x&-\cos x\end{array}\right|;
$
\begin{eqnarray*}
W(0)&=&\left|\begin{array}{rrrr}
1&1&1&0\\1&-1&0&1\\1&1&-1&0\\1&-1&0&-1
\end{array}\right|
=\left|\begin{array}{rrrr}
1&1&1&0\\1&-1&0&1\\1&1&-1&0\\0&0&0&-2
\end{array}\right|
=-2\left|\begin{array}{rrrr}
1&1&1\\1&-1&0\\1&1&-1\\
\end{array}\right|\\
&=&-2\left|\begin{array}{rrrr}
1&1&1\\1&-1&0\\0&0&-2\\
\end{array}\right|
=4\left|\begin{array}{rrrr}
1&1\\1&-1\\\end{array}\right|=-8.
\end{eqnarray*}
\end{solution}

    \item $(D^2-2D+2)(D^2+1)y=0$

\begin{solution}
$
W(x)=\left|\begin{array}{rrcc}\cos x&\sin x&e^x\cos x&e^x\sin x\\
-\sin x&\cos x&e^x(\cos x-\sin x)&e^x(\cos x+\sin x)\\
-\cos x&-\sin x&-2e^x\sin x&2e^x\cos x\\
\sin x&-\cos x&-e^x(2\cos x+2\sin x)&e^x(2\cos x-2\sin x)
\end{array}\right|;
$

\begin{eqnarray*}
W(0)&=&\left|\begin{array}{rrrr}
1&0&1&0\\0&1&1&1\\-1&0&0&2\\0&-1&-2&2
\end{array}\right|=
\left|\begin{array}{rrrr}
1&0&1&0\\0&1&1&1\\0&0&1&2\\0&-1&-2&2
\end{array}\right|=
\left|\begin{array}{rrrr}
1&1&1\\0&1&2\\-1&-2&2
\end{array}\right|\\
&=&\left|\begin{array}{rrrr}
1&1&1\\0&1&2\\0&-1&3
\end{array}\right|=
\left|\begin{array}{rrrr}
1&1&1\\0&1&2\\0&0&5
\end{array}\right|=5.
\end{eqnarray*}
\end{solution}

\end{enumerate}
\end{problem}

\begin{problem}\label{exer:9.2.29} Find a
fundamental set of solutions.

$(D^2+6D+13)(D-2)^2D^3y=0$
\end{problem}

\begin{problem}\label{exer:9.2.30} Find a
fundamental set of solutions.

$(D-1)^2(2D-1)^3(D^2+1)y=0$
\end{problem}

\begin{problem}\label{exer:9.2.31} Find a
fundamental set of solutions.

$(D^2+9)^3D^2y=0$
\end{problem}

\begin{problem}\label{exer:9.2.32}  Find a
fundamental set of solutions.

$(D-2)^3(D+1)^2Dy=0$
\end{problem}


\begin{problem}\label{exer:9.2.33}  Find a
fundamental set of solutions.

$(D^2+1)(D^2+9)^2(D-2)y=0$
\end{problem}

\begin{problem}\label{exer:9.2.34} Find a
fundamental set of solutions.

$(D^4-16)^2y=0$
\end{problem}

\begin{problem}\label{exer:9.2.35}  Find a
fundamental set of solutions.

$(4D^2+4D+9)^3y=0$
\end{problem}

\begin{problem}\label{exer:9.2.36} Find a
fundamental set of solutions.

$D^3(D-2)^2(D^2+4)^2y=0$
\end{problem}

\begin{problem}\label{exer:9.2.37}  Find a
fundamental set of solutions.

$(4D^2+1)^2(9D^2+4)^3y=0$
\end{problem}

\begin{problem}\label{exer:9.2.38} Find a
fundamental set of solutions.

$\left[(D-1)^4-16\right]y=0$
\end{problem}


\begin{problem}\label{exer:9.2.39}
It can be shown that
$$
\left|\begin{array}{cccc}
1&1&\cdots&1\\
a_1&a_2&\cdots&a_n\\
a^2_1&a^2_2&\cdots&a^2_n\\
\vdots&\vdots&\ddots&\vdots\\
a^{n-1}_1&a^{n-1}_2&\cdots&a^{n-1}_n\end{array}\right|=
\prod_{1\le i<j\le n}(a_j-a_i),
\text{(A)}
$$
where the left side is  the
\href{http://www-history.mcs.st-and.ac.uk/Mathematicians/Vandermonde.html}{Vandermonde
  determinant} and
the right side is the product of all factors of the form $(a_j-a_i)$
with $i$ and $j$ between $1$ and $n$ and $i<j$.

\begin{enumerate}
\item %(a)
Verify  (A) for $n=2$ and $n=3$.

\item %(b)
Find the Wronskian of $\{e^{{a_1}x}, \quad  e^{{a_2}x},\dots, e^{{a_n}x}\}$.
\end{enumerate}
\end{problem}

\begin{problem}\label{exer:9.2.40}
A theorem from algebra says that if $P_1$ and $P_2$ are polynomials
with no common factors then there are polynomials $Q_1$ and $Q_2$
such that
$$
Q_1P_1+Q_2P_2=1.
$$
This implies that
$$
Q_1(D)P_1(D)y+Q_2(D)P_2(D)y=y
$$
for every function $y$ with enough derivatives for the left side to be
defined.

\begin{enumerate}
\item %(a)
Use this to show that if $P_1$ and $P_2$ have no common factors and
$$
P_1(D)y=P_2(D)y=0
$$
then $y=0$.

\begin{solution}
Since $y=Q_1(D)P_1(D)y+Q_2(D)P_2(D)y$ and $P_1(D)y=P_2(D)y=0$, it
follows that $y=0$.
\end{solution}

\item %(b)
Suppose $P_1$ and $P_2$ are polynomials with no common factors.
Let $u_1$, \dots, $u_r$ be linearly independent solutions of $P_1(D)y=0$
and let $v_1$, \dots, $v_s$ be linearly independent solutions of
$P_2(D)y=0$. Use the previous part to show that $\{u_1,\dots,u_r,\,
v_1,\dots,v_s\}$ is a linearly independent set.

\begin{solution}
Suppose that
(A) $a_1u_1+\cdots+a_ru_r+b_1v_1+\cdots+b_sv_s=0$,
where $a_1,\dots,a_r$ and $b_1,\dots,b_s$ are constants. Denote
$u=a_1u_1+\cdots+a_ru_r$ and $v=b_1v_1+\cdots+b_sv_s$. Then
(B) $P_1(D)u=0$ and (C) $P_2(D)v=0$. Since $u+v=0$, $P_2(D)(u+v)=0$.
Therefore,$0=P_2(D)(u+v)=P_2(D)u+P_2(D)v$. Now (C) implies that
$P_2(D)u=0$. This, (B), and the previous part imply that
$u=a_1u_1+\cdots+a_ru_r=0$, so $a_1=\dots=a_r=0$, since
$u_1,\dots,u_r$ are linearly independent. Now (A) reduces to
$b_1v_1+\cdots+b_sv_s=0$,
 so $b_1=c\dots=b_s=0$, since
$v_1,\dots,v_s$ are linearly independent.
Therefore,$u_1,\dots,u_r,
v_1,\dots,v_r$ are linearly independent.
\end{solution}

\item %(c)
Suppose the characteristic polynomial of the constant coefficient
equation
$$
a_0y^{(n)}+a_1y^{(n-1)}+\cdots+a_ny=0
\text{(A)}
$$
has the factorization
$$
p(r)=a_0p_1(r)p_2(r)\cdots p_k(r),
$$
where each $p_j$ is of the form
$$
p_j(r)=(r-r_j)^{n_j} \mbox{ or }
p_j(r)=[(r-\lambda_j)^2+w^2_j]^{m_j}\quad  (\omega_j>0)
$$
and no two of the polynomials $p_1$, $p_2$, \dots, $p_k$ have a common
factor. Show that we can find a fundamental set of solutions
$\{y_1,y_2,\dots,y_n\}$ of (A) by finding a
fundamental set of solutions of each of the equations
$$
p_j(D)y=0,\quad 1\le j\le k,
$$
 and taking $\{y_1,y_2,\dots,y_n\}$ to be the set of all
functions in these separate fundamental sets.

\begin{solution}
It suffices to show that
 $\{y_1,y_2,\dots,y_n\}$ is linearly independent. Suppose that
$c_1y_1+\cdots+c_ny_n=0$. We may assume that $y_1,\dots,y_r$
are linearly independent solutions of $p_1(D)y=0$ and
$y_{r+1},\dots,y_n$ are solutions of
$P_2(D)=p_2(D)\cdots p_k(D)y=0$. Since $p_1(r)$ and $P_2(r)$ have no
common factors,
the previous part implies
that (A) $c_1y_1+\cdots+c_ry_r=0$ and
(B) $c_{r+1}y_{r+1}+\cdots+c_ny_n=0$.
Now (A) implies that $c_1=\cdots=c_r=0$, since $y_1,\dots,y_r$
are linearly independent. If $k=2$, then $y_{r+1},\dots,y_n$
are linearly independent, so $c_{r+1}=\cdots=c_n=0$, and the proof is
complete. If $k>2$ repeat this argument, starting from (B), with $p_1$
replaced by $p_2$, and $P_2$ replaced by $P_3=p_3\cdots p_n$.
\end{solution}
\end{enumerate}
\end{problem}

\begin{problem}\label{exer:9.2.41}
\begin{enumerate}
\item % (a)
Show that if
$$
z=p(x)\cos\omega x+q(x)\sin\omega x,
\text{(A)}
$$
where $p$ and $q$ are polynomials of degree $\le k$, then
$$
(D^2+\omega^2)z=p_1(x)\cos\omega x+q_1(x)\sin\omega x,
$$
where $p_1$ and $q_1$ are polynomials of degree $\le k-1$.

\item % (b)
Apply the previous part $m$ times to show that if $z$ is of
the form (A) where $p$ and $q$ are polynomials of
degree $\le m-1$, then
$$
(D^2+\omega^2)^mz=0.
\text{(B)}
$$

\item %(c)
Use Eqn.~\eqref{eq:9.2.17} to show that if $y=e^{\lambda x}z$ then
$$
[(D-\lambda)^2+\omega^2]^my=e^{\lambda
x}(D^2+\omega^2)^mz.
$$

\item % (d)
Conclude that if $p$ and $q$ are arbitrary polynomials of degree $\le m-1$ then
$$
y=e^{\lambda x}(p(x)\cos\omega x+q(x)\sin\omega x)
$$
is a solution of
$$
[(D-\lambda)^2+\omega^2]^my=0.
\text{(C)}
$$

\item % (e)
Use this conclusion to show that the functions
$$
\begin{array}{rl}
e^{\lambda x}\cos\omega x, xe^{\lambda x}\cos\omega x,
&\dots, x^{m-1}e^{\lambda x}\cos\omega x,\\
e^{\lambda x}\sin\omega x, xe^{\lambda x}\sin\omega x,&
\dots, x^{m-1}e^{\lambda x}\sin\omega x
\end{array}
\text{(D)}
$$
are all solutions of (C).

\item % (f)
Complete the proof of Theorem~\ref{thmtype:9.2.2} by showing that the
functions in (D) are linearly independent.
\end{enumerate}
\end{problem}

\begin{problem}\label{exer:9.2.42}
\begin{enumerate}
\item % (a)
Use the trigonometric identities
\begin{eqnarray*}
\cos(A+B)&=&\cos A\cos B-\sin A\sin B\\
\sin(A+B)&=&\cos A\sin B+\sin A\cos B
\end{eqnarray*}
to show that
$$
(\cos A+i\sin A)(\cos B+i\sin B)=\cos(A+B)+i\sin(A+B).
$$

\begin{solution}
\begin{eqnarray*}
(\cos A+i\sin A)(\cos B+i\sin B)&=&(\cos A\cos B-\sin A\sin B)
\\&&+(\cos A\sin B+\sin A\cos B)\\
&=&\cos(A+B)+i\sin(A+B).
\end{eqnarray*}
\end{solution}

\item % (b)
Apply the previous part repeatedly to show that if $n$ is a positive
integer then
$$
\prod_{k=1}^n(\cos A_k+i\sin A_k)=\cos(A_1+A_2+\cdots+A_n)
+i\sin(A_1+A_2+\cdots+A_n).
$$

\begin{solution}
Obvious for $n=0$. If $n=-1$ write
\begin{eqnarray*}
{1\over\cos\theta+i\sin\theta}&=&{1\over\cos\theta+i\sin\theta}
{\cos\theta-i\sin\theta\over\cos\theta-i\sin\theta}\\
&=&{\cos\theta-i\sin\theta\over\cos^2\theta+\sin^2\theta}=
\cos\theta-i\sin\theta=\cos(-\theta)+i\sin(-\theta).
\end{eqnarray*}
\end{solution}

\item % (c)
Infer that if $n$ is a positive integer then
$$
(\cos\theta+i\sin\theta)^n=\cos n\theta+i\sin n\theta.
\text{(A)}
$$

\item % (d)
Show that (A) also holds if $n=0$ or a negative integer.
\begin{hint}
Verify by direct calculation that
$$
(\cos\theta+i\sin\theta)^{-1}=(\cos\theta-i\sin\theta).
$$
Then replace $\theta$ by $-\theta$  in (A).
\end{hint}

\begin{solution}
If $n$ is a negative integer, then
(B)
$(\cos\theta+i\sin\theta)^n=\frac{1}{(\cos\theta+i\sin\theta)^{|n|}}$.
From the hint,
(C)
$\frac{1}{(\cos\theta+i\sin\theta)^{|n|}}=(\cos\theta-i\sin\theta)^{|n|}
=(\cos(-\theta)+i\sin(-\theta))^{|n|}$.
Replacing $\theta$ by $-\theta$ and $n$ by $|n|$ in (A) shows that
(D)
$(\cos(-\theta)+i\sin(-\theta))^{|n|}=\cos(-|n|\theta)+i\sin(-|n|\theta)$.
Since $|n|=-n$, (E)
$\cos(-|n|\theta)+i\sin(-|n|\theta)=\cos n\theta+i\sin n\theta$. Now
(B), (C), (D), and (E) imply (A).
\end{solution}

\item\label{exer:9.2.42e} % (e)
Now suppose $n$ is a positive integer. Infer from (A)
that if
$$
z_k=\cos\left(2k\pi\over n\right)+i\sin\left(2k\pi\over n\right)
,\quad k=0,1,\dots,n-1,
$$
and
$$
\zeta_k=\cos\left((2k+1)\pi\over
n\right)+i\sin\left((2k+1)\pi\over n\right)
,\quad k=0,1,\dots,n-1,
$$
then
$$
z_k^n=1\quad\mbox{ and }\quad\zeta_k^n=-1,\quad k=0,1,\dots,n-1.
$$
(Why don't we also consider other integer values for $k$?)

\begin{solution}
From (A), $z_k^n=\cos2k\pi+i\sin2k\pi=1$ and
$\zeta_k^n=\cos(2k+1)\pi+i\sin(2k+1)\pi=\cos(2k+1)\pi=\cos\pi=-1$.
\end{solution}

\item % (f)
Let $\rho$ be a positive number. Use the previous part to show that
$$
z^n-\rho=(z-\rho^{1/n} z_0)(z-\rho^{1/n}z_1)\cdots(z-\rho^{1/n} z_{n-1})
$$
and
$$
z^n+\rho=(z-\rho^{1/n} \zeta_0)(z-\rho^{1/n} \zeta_1)\cdots(z-\rho^{1/n}
\zeta_{n-1}).
$$
\end{enumerate}

\begin{solution}
From the previous part, $\rho^{1/n}z_0,\dots,\rho^{1/n}z_{n-1}$
are all zeros of $z^n-\rho$. Since they are distinct numbers,
$z^n-\rho$ has the stated factorization.

From the previous part, $\rho^{1/n}\zeta_0,\dots,\rho^{1/n}\zeta_{n-1}$
are all zeros of $z^n+\rho$. Since they are distinct numbers,
$z^n+\rho$ has the stated factorization.
\end{solution}
\end{problem}

\begin{problem}\label{exer:9.2.43}
Use \ref{exer:9.2.42e} to find a fundamental set of
solutions of the given equation.

\begin{enumerate}
\item $y'''-y=0$

\begin{solution}
$p(r)=r^3-1=(r-z_0)(r-z_1)(r-z_2)$ where
$z_k=\cos\frac{2k\pi}{3}+i\sin\frac{2k\pi}{3}$, $k=0,1,2$. Hence,
$z_0=1$, $z_1=-\frac{1}{2}+i\frac{\sqrt3}{2}$, and
$z_2=-\frac{1}{2}-i\frac{\sqrt3}{2}$. Therefore,
$p(r)=(r-1)\left(\left(r+\frac{1}{2}\right)^2+\frac{3}{4}\right)$,
so
$\left\{e^x,\,e^{-x/2}\cos\left(\frac{\sqrt3}{2}x\right),
\,e^{-x/2}\sin\left({\sqrt3\over2}x\right)\right\}$
is a fundamental set of solutions.
\end{solution}

\item $y'''+y=0$

\begin{solution}
$p(r)=r^3+1=(r-\zeta_0)(r-\zeta_1)(r-\zeta_2)$ where
$\zeta_k=\cos\frac{(2k+1)\pi}{3}+i\sin\frac{(2k+1)\pi}{3}$, $k=0,1,2$.
Hence,
 $\zeta_0=\frac{1}{2}+i\frac{\sqrt3}{2}$, $\zeta_1=-1$
$\zeta_2=\frac{1}{2}-i\frac{\sqrt3}{2}$. Therefore,
$p(r)=(r+1)\left(\left(r-\frac{1}{2}\right)^2+\frac{3}{4}\right)$,
so
 $\left\{e^{-x},\,e^{x/2}\cos\left(\frac{\sqrt3}{2}x\right),
\,e^{x/2}\sin\left(\frac{\sqrt3}{2}x\right)\right\}$
is a fundamental set of solutions.
\end{solution}

\item $y^{(4)}+64y=0$

\begin{solution}
$p(r)=r^4+64=(r-2\sqrt2\zeta_0)(r-2\sqrt2\zeta_1)
(r-2\sqrt2\zeta_2)(r-2\sqrt2\zeta_3)$,
where
$\zeta_k=\cos\frac{(2k+1)\pi}{4}+i\sin\frac{(2k+1)\pi}{4}$,
$k=0,1,2,3$. Therefore,
$\zeta_0=\frac{1+i}{\sqrt2}$,
$\zeta_1=\frac{-1+i}{\sqrt2}$,
$\zeta_2=\frac{-1-i}{\sqrt2}$,
and $\zeta_3=\frac{1-i}{\sqrt2}$, so
 $p(r)=((r-2)^2+4)((r+2)^2+4)$ and
$\{e^{2x}\cos2x,\,e^{2x}\sin2x,\,e^{-2x}\cos2x,\,e^{-2x}\sin2x\}$
is a fundamental set of solutions.
\end{solution}

\item $y^{(6)}-y=0$\label{exer:9.2.43d}

\begin{solution}
$p(r)=r^6-1=(r-z_0)(r-z_1)(r-z_2)(r-z_3)(r-z_4)(r-z_5)$ where
$z_k=\cos\frac{2k\pi}{6}+i\sin\frac{2k\pi}{6}$,
$k=0,1,2,3,4,5$. Therefore,
$z_0=1$,
$z_1=\frac{1}{2}+i\frac{\sqrt3}{2}$,
$z_2=-\frac{1}{2}+i\frac{\sqrt3}{2}$,
$z_3=-1$,
$z_4=-\frac{1}{2}-i\frac{\sqrt3}{2}$,
and $z_5=\frac{1}{2}-i\frac{\sqrt3}{2}$, so
$p(r)=(r-1)(r+1)\left(\left(r-\frac{1}{2}\right)^2+\frac{3}{4}\right)
\left(\left(r+\frac{1}{2}\right)^2+\frac{3}{4}\right)$ and

 $\left\{e^x,\,e^{-x}
,\,e^{x/2}\cos\left(\frac{\sqrt3}{2}x\right),
\,e^{x/2}\sin\left(\frac{\sqrt3}{2}x\right),
\,e^{-x/2}\cos\left(\frac{\sqrt3}{2}x\right),
\,e^{-x/2}\sin\left(\frac{\sqrt3}{2}x\right)\right\}$
is a fundamental set of solutions.
\end{solution}

\item $y^{(6)}+64y=0$

\begin{solution}
$p(r)=r^6+64=(r-2\zeta_0)(r-2\zeta_1)(r-2\zeta_2)(r-2\zeta_3)
(r-2\zeta_4)(r-2\zeta_5)$ where
$\zeta_k=\cos\frac{(2k+1)\pi}{6}+i\sin\frac{(2k+1)\pi}{6}$,
$k=0,1,2,3,4,5$. Therefore,
$\zeta_0=\frac{\sqrt3}{2}+\frac{i}{2}$,
$\zeta_1=i$,
$\zeta_2=-\frac{\sqrt3}{2}+\frac{i}{2}$,
$\zeta_3=-\frac{\sqrt3}{2}-\frac{i}{2}$,
$\zeta_4=-i$, and
$\zeta_5=\frac{\sqrt3}{2}-\frac{i}{2}$,  so
$p(r)=(r^2+4)((r-\sqrt3)^2+1)((r+\sqrt3)^2+1)$ and
 $\{\cos2x,\,\sin2x,\,e^{-\sqrt3x}\cos x,\,e^{-\sqrt3x}\sin x
,\,e^{\sqrt3x}\cos x,\,e^{\sqrt3x}\sin x\}$
is a fundamental set of solutions.
\end{solution}

\item $\left[(D-1)^6-1\right]y=0$

\begin{solution}
$p(r)=(r-1)^6-1=(r-1-z_0)(r-1-z_1)(r-1-z_2)(r-1-z_3)(r-1-z_4)(r-1-z_5)$
where
$z_k=\cos\frac{2k\pi}{6}+i\sin\frac{2k\pi}{6}$,
$k=0,1,2,3,4,5$. Therefore,
$z_0=1$,
$z_1=\frac{1}{2}+i\frac{\sqrt3}{2}$,
$z_2=-\frac{1}{2}+i\frac{\sqrt3}{2}$,
$z_3=-1$,
$z_4=-\frac{1}{2}-i\frac{\sqrt3}{2}$,
and $z_5=\frac{1}{2}-i\frac{\sqrt3}{2}$, so
$p(r)=r(r-2)\left(\left(r-\frac{3}{2}\right)^2+\frac{3}{4}\right)
\left(\left(r-\frac{1}{2}\right)^2+\frac{3}{4}\right)$ and
$\left\{1,\,e^{2x}
,\,e^{3x/2}\cos\left(\frac{\sqrt3}{2}x\right),
\,e^{3x/2}\sin\left(\frac{\sqrt3}{2}x\right),
\,e^{x/2}\cos\left(\frac{\sqrt3}{2}x\right),
\,e^{x/2}\sin\left(\frac{\sqrt3}{2}x\right)\right\}$
is a fundamental set of solutions.
\end{solution}

\item $y^{(5)}+y^{(4)}+y'''+y''+y'+y=0$

\begin{solution}
$p(r)=r^5+r^4+r^3+r^2+r+1=\frac{r^6-1}{r-1}$. Therefore,
from the solution of \ref{exer:9.2.43d}
$p(r)=(r+1)\left(\left(r-\frac{1}{2}\right)^2+\frac{3}{4}\right)
\left(\left(r+\frac{1}{2}\right)^2+\frac{3}{4}\right)$ and

$\left\{e^{-x}
,\,e^{x/2}\cos\left(\frac{\sqrt3}{2}x\right),
\,e^{x/2}\sin\left(\frac{\sqrt3}{2}x\right),
\,e^{-x/2}\cos\left(\frac{\sqrt3}{2}x\right),
\,e^{-x/2}\sin\left(\frac{\sqrt3}{2}x\right)\right\}$
is a fundamental set of solutions.
\end{solution}
\end{enumerate}
\end{problem}

\begin{problem}\label{exer:9.2.44}
An equation of the form
$$
 a_0x^ny^{(n)}+a_1x^{n-1}y^{(n-1)}+\cdots
+a_{n-1}xy'+a_ny=0,\quad x>0,
\text{(A)}
$$
where $a_0$, $a_1$, \dots, $a_n$ are constants, is an \emph{Euler} or
\emph{equidimensional} equation.

Show that if
$$
 x=e^t \quad \mbox{ and } \quad Y(t)=y(x(t)),
\text{(B)}
$$
then
\begin{eqnarray*}
x \frac{dy}{dx}&=&\frac{dY}{dt}\\
x^2\frac{d^2y}{dx^2}&=&\frac{d^2Y}{dt^2}-\frac{dY}{dt}\\
x^3\frac{d^3y}{dx^3}&=&\frac{d^3Y}{dt^3}-3\frac{d^2Y}{dt^2}+2\frac{dY}{dt}.
\end{eqnarray*}
In general, it can be shown that if $r$ is any integer $\ge2$ then
$$
x^r \frac{d^ry}{dx^r}=\frac{d^rY}{dt^r}+
 A_{1r}\frac{d^{r-1}Y}{dt^{r-1}}+\cdots+A_{r-1,r}
 \frac{dY}{dt}
$$
where $A_{1r}$, \dots, $A_{r-1,r}$ are integers.  Use these results to
show that the substitution (B) transforms (A) into a
constant coefficient equation for $Y$ as a function of $t$.
\end{problem}

\begin{problem}\label{exer:9.2.45}
Use Exercise~\ref{exer:9.2.44}  to show that a function $y=y(x)$ satisfies
the equation
$$
 a_0x^3y'''+a_1x^2y''+a_2xy'+a_3y=0,
\text{(A)}
$$
on $(0,\infty)$
if and only if the function $Y(t)=y(e^t)$ satisfies
$$
a_0\frac{d^3Y}{dt^3}+(a_1-3a_0) \frac{d^2Y}{dt^2}+(a_2-a_1+2a_0)
\frac{dY}{dt}+a_3Y=0.
$$
Assuming that $a_0$, $a_1$, $a_2$, $a_3$ are real and $a_0 \ne0$, find
the possible forms for the general solution of (A).
\end{problem}

\end{document}