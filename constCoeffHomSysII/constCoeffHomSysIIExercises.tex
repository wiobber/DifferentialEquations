\documentclass{ximera}
\input{../preamble.tex}

\title{Exercises} \license{CC BY-NC-SA 4.0}

\begin{document}

\begin{abstract}
\end{abstract}
\maketitle

\begin{onlineOnly}
\section*{Exercises}
\end{onlineOnly}


 \begin{problem}\label{exer:10.5.1}
 Find the general solution.
 
 $ {\bf y}'= \begin{bmatrix}3&4\\-1&7\end{bmatrix}{\bf y}$

  \end{problem}

 \begin{problem}\label{exer:10.5.2}
 Find the general solution.
 
 $ {\bf y}'= \begin{bmatrix}0&-1\\1&-2\end{bmatrix}{\bf y}$

 \begin{solution}
     $ \begin{vmatrix}-\lambda&-1\\1&-2-\lambda\end{vmatrix}=(\lambda+1)^2$.
Hence $\lambda_1=-1$.
Eigenvectors   satisfy
$  \begin{bmatrix}1&-1\\1&-1\end{bmatrix} \begin{bmatrix}x_1\\x_2\end{bmatrix}= \begin{bmatrix}0\\0\end{bmatrix}$,
so $x_1=x_2$.  Taking $x_2=1$ yields
${\bf y}_1=  \begin{bmatrix}1\\1\end{bmatrix}e^{-t}$.
For a second solution we need a vector ${\bf u}$
such that $  \begin{bmatrix}1&-1\\1&-1\end{bmatrix} \begin{bmatrix}u_1\\u_2\end{bmatrix}
= \begin{bmatrix}1\\1\end{bmatrix}$. Let $u_1=1$ and $u_2=0$. Then
${\bf y}_2=  \begin{bmatrix}1\\0\end{bmatrix}e^{-t}+  \begin{bmatrix}1\\1\end{bmatrix}te^{-t}$.
The general solution is
${\bf y}= 
c_1 \begin{bmatrix}1\\1\end{bmatrix}e^{-t}+c_2\left( \begin{bmatrix}1\\0\end{bmatrix}e^{-t}+ \begin{bmatrix}1\\1\end{bmatrix}te^{-t}\right)$.
 \end{solution}
 \end{problem}


 \begin{problem}\label{exer:10.5.3}
 Find the general solution.
 
 $ {\bf y}'= \begin{bmatrix}-7&4\\-1&-11\end{bmatrix}{\bf y}$ 
 \end{problem}

 \begin{problem}\label{exer:10.5.4} 
 Find the general solution.
 
 $ {\bf y}'= \begin{bmatrix}3&1\\-1&1\end{bmatrix}{\bf y}$

 \begin{solution}
      $ {\bf y}'=\begin{bmatrix}3-\lambda&1\\-1&1-\lambda\end{bmatrix}=(\lambda-2)^2$.
Hence $\lambda_1=2$.
Eigenvectors   satisfy
$  \begin{bmatrix}1&1\\-1&-1\end{bmatrix} \begin{bmatrix}x_1\\x_2\end{bmatrix}= \begin{bmatrix}0\\0\end{bmatrix}$,
so $x_1=-x_2$.  Taking $x_2=1$ yields
${\bf y}_1=  \begin{bmatrix}-1\\1\end{bmatrix}e^{2t}$.
For a second solution we need a vector ${\bf u}$
such that $  \begin{bmatrix}1&1\\-1&-1\end{bmatrix} \begin{bmatrix}u_1\\u_2\end{bmatrix}
= \begin{bmatrix}-1\\1\end{bmatrix}$. Let $u_1=-1$ and $u_2=0$. Then
${\bf y}_2=  \begin{bmatrix}-1\\0\end{bmatrix}e^{2t}+  \begin{bmatrix}-1\\1\end{bmatrix}te^{2t}$.
The general solution is
 ${\bf y}= 
c_1 \begin{bmatrix}-1\\1\end{bmatrix}e^{2t}+c_2\left( \begin{bmatrix}-1\\0\end{bmatrix}
e^{2t}+ \begin{bmatrix}-1\\1\end{bmatrix}te^{2t}\right)$.
 \end{solution}
 \end{problem}


 \begin{problem}\label{exer:10.5.5}
 Find the general solution.
 
 $ {\bf y}'= \begin{bmatrix}4&12\\-3&-8\end{bmatrix}{\bf y}$
  \end{problem}

 \begin{problem}\label{exer:10.5.6}
 Find the general solution.
 
 $ {\bf
y'}= \begin{bmatrix}-10&9\\-4&2\end{bmatrix}{\bf y}$

\begin{solution}
    $ \begin{vmatrix}-10-\lambda&9\\-4&2-\lambda\end{vmatrix}=(\lambda+4)^2$.
Hence $\lambda_1=-4$.
Eigenvectors   satisfy
$  \begin{bmatrix}-6&9\\-4&6\end{bmatrix} \begin{bmatrix}x_1\\x_2\end{bmatrix}= \begin{bmatrix}0\\0\end{bmatrix}$,
so $x_1=\frac{3}{2}x_2$.  Taking $x_2=2$ yields
${\bf y}_1=  \begin{bmatrix}3\\2\end{bmatrix}e^{-4t}$.
For a second solution we need a vector ${\bf u}$
such that $  \begin{bmatrix}-6&9\\-4&6\end{bmatrix} \begin{bmatrix}u_1\\u_2\end{bmatrix}
= \begin{bmatrix}3\\2\end{bmatrix}$. Let $u_1=-\frac{1}{2}$ and $u_2=0$. Then
${\bf y}_2=  \begin{bmatrix}-1\\0\end{bmatrix}\frac{e^{-4t}}{2}+  \begin{bmatrix}3\\2\end{bmatrix}te^{-4t}$.
The general solution is
${\bf y}= c_1 \begin{bmatrix}3\\2\end{bmatrix}e^{-4t}+
c_2\left( \begin{bmatrix}-1\\0\end{bmatrix}\frac{e^{-4t}}{2}+ \begin{bmatrix}3\\2\end{bmatrix}te^{-4t}\right)$.
\end{solution}
 \end{problem}


 \begin{problem}\label{exer:10.5.7}
 Find the general solution.
 
 $ {\bf
y'}= \begin{bmatrix}-13&16\\-9&11\end{bmatrix}{\bf y}$
 \end{problem}

 \begin{problem}\label{exer:10.5.8} 
 Find the general solution.
 
 $ {\bf
y'}= \begin{bmatrix}0&2&1\\-4&6&1\\0&4&2\end{bmatrix}{\bf y}$

\begin{solution}
    $ \begin{vmatrix}-\lambda&2&1\\-4&6-\lambda&1\\0&4&2-\lambda\end{vmatrix}
=-\lambda(\lambda-4)^2$.
Hence $\lambda_1=0$ and  $\lambda_2=\lambda_3=4$.
The eigenvectors associated
 with $\lambda_1=0$ satisfy the system with  augmented matrix
$  \begin{bmatrix}0&2&1&\vdots&0\\{-4}&6&1&
\vdots&0\\0&4&2&\vdots&0 \end{bmatrix}$,
which is row equivalent to
$  \begin{bmatrix}1&0&1/2&\vdots&0\\0&1&1/2&
\vdots&0\\0&0&0&\vdots&0 \end{bmatrix}$.
Hence  $x_1=-\frac{1 }{2}x_3$ and $x_2=-\frac{1 }{2}x_3$.  Taking $x_3=2$
yields
${\bf y}_1=  \begin{bmatrix}-1\\-1\\2\end{bmatrix}$.
The eigenvectors associated
 with $\lambda_2=4$ satisfy the system with  augmented matrix
$  \begin{bmatrix}-4&2&1&\vdots&0\\-4&2&1&
\vdots&0\\0&4&-2&\vdots&0 \end{bmatrix}$,
which is row equivalent to
$  \begin{bmatrix}1&0&-1/2&\vdots&0\\0&1&-1/2&
\vdots&0\\0&0&0&\vdots&0 \end{bmatrix}$.
Hence  $x_1=\frac{1 }{2}x_3$ and $x_2=\frac{1 }{2}x_3$.  Taking
$x_3=2$ yields
${\bf y}_2=  \begin{bmatrix}1\\1\\2\end{bmatrix}e^{4t}$.
For a third solution we need a vector ${\bf u}$ such that
$  \begin{bmatrix}-4&2&1\\-4&2&1\\0&4&-2\end{bmatrix} \begin{bmatrix}u_1\\u_2\\u_3\end{bmatrix}=  \begin{bmatrix}1\\1\\2\end{bmatrix}$.
The augmented matrix of this system is row equivalent to
$  \begin{bmatrix}1&0&-1/2&\vdots&0\\0&1&-1/2&
\vdots&1/2\\0&0&0&\vdots&0 \end{bmatrix}$.
Let  $u_3=0$, $u_1=0$, and $u_2=\frac{1 }{2}$. Then
${\bf
y}_3=  \begin{bmatrix}0\\1\\0\end{bmatrix}\frac{e^{4t} }{2}+  \begin{bmatrix}1\\1\\2\end{bmatrix}te^{4t}$.
The general solution is
$$
{\bf y}= 
c_1 \begin{bmatrix}-1\\-1\\2\end{bmatrix}+c_2 \begin{bmatrix}1\\1\\2\end{bmatrix}e^{4t}+
c_3\left( \begin{bmatrix}0\\1\\0\end{bmatrix}\frac{e^{4t} }{2}+ \begin{bmatrix}1\\1\\2\end{bmatrix}te^{4t}\right).
$$

\end{solution}
 \end{problem}


 \begin{problem}\label{exer:10.5.9}
 Find the general solution.
 
$ {\bf y}' =\frac{1 }{3} \begin{bmatrix}1&1&-3\\-4&-4&3\\-2&1&0\end{bmatrix}{\bf y}$
 \end{problem}

 \begin{problem}\label{exer:10.5.10} 
 Find the general solution.
 
 $ {\bf y}'
= \begin{bmatrix}-1&1&-1\\-2&0&2\\-1&3&-1\end{bmatrix}{\bf y}$

\begin{solution}
    $ \begin{vmatrix}-1-\lambda&1&-1\\-2&-\lambda&2\\-1&3&-1-\lambda\end{vmatrix}
=-(\lambda-2)(\lambda+2)^2$.
Hence $\lambda_1=2$ and  $\lambda_2=\lambda_3=-2$.
The eigenvectors associated
 with $\lambda_1=2$ satisfy the system with  augmented matrix
$  \begin{bmatrix}-3&1&-1&\vdots&0\\-2&-2&2&
\vdots&0\\-1&3&-3&\vdots&0 \end{bmatrix}$,
which is row equivalent to
$  \begin{bmatrix}1&0&0&\vdots&0\\0&1&-1&
\vdots&0\\0&0&0&\vdots&0 \end{bmatrix}$.
Hence  $x_1=0$ and $x_2=x_3$.  Taking $x_3=1$
yields
${\bf y}_1=  \begin{bmatrix}0\\1\\1\end{bmatrix}e^{2t}$.
The eigenvectors associated
 with $\lambda_2=-2$ satisfy the system with  augmented matrix
$  \begin{bmatrix}1&1&-1&\vdots&0\\-2&2&2&
\vdots&0\\-1&3&1&\vdots&0 \end{bmatrix}$,
which is row equivalent to
$  \begin{bmatrix}1&0&-1&\vdots&0\\0&1&0&
\vdots&0\\0&0&0&\vdots&0 \end{bmatrix}$.
Hence  $x_1=x_3$ and $x_2=0$.  Taking $x_3=1$
yields
${\bf y}_2=  \begin{bmatrix}1\\0\\1\end{bmatrix}e^{-2t}$.
For a third solution we need a vector ${\bf u}$ such that
$  \begin{bmatrix}1&1&-1\\-2&2&2\\-1&3&1\end{bmatrix} \begin{bmatrix}u_1\\u_2\\u_3\end{bmatrix}=  \begin{bmatrix}1\\0\\1\end{bmatrix}$.
The augmented matrix of this system is row equivalent to
$  \begin{bmatrix}1&0&-1&\vdots&1/2\\0&1&0&
\vdots&1/2\\0&0&0&\vdots&0 \end{bmatrix}$.
Let $u_3=0$, $u_1=\frac{1 }{2}$, and $u_2=\frac{1 }{2}$. Then ${\bf
y}_3=  \begin{bmatrix}1\\1\\0\end{bmatrix}\frac{e^{-2t} }{2}+  \begin{bmatrix}1\\0\\1\end{bmatrix}te^{-2t}$.
The general solution is
$$
{\bf y}= 
c_1 \begin{bmatrix}0\\1\\1\end{bmatrix}e^{2t}+c_2 \begin{bmatrix}1\\0\\1\end{bmatrix}e^{-2t}+c_3\left
( \begin{bmatrix}1\\1\\0\end{bmatrix}\frac{e^{-2t} }{2}+ \begin{bmatrix}1\\0\\1\end{bmatrix}te^{-2t}\right).
$$

\end{solution}
 \end{problem}


 \begin{problem}\label{exer:10.5.11}
 Find the general solution.
 
 $ {\bf y}'
= \begin{bmatrix}4&-2&-2\\-2&3&-1\\2&-1&3\end{bmatrix}{\bf y}$
 \end{problem}

 \begin{problem}\label{exer:10.5.12}  
 Find the general solution.
 
 $ {\bf y}'=
 \begin{bmatrix}6&-5&3\\2&-1&3\\2&1&1\end{bmatrix}{\bf y}$

 \begin{solution}
     $ \begin{vmatrix}6-\lambda&-5&3\\2&-1-\lambda&3\\2&1&1-\lambda\end{vmatrix}
=-(\lambda+2)(\lambda-4)^2$.
Hence $\lambda_1=-2$ and  $\lambda_2=\lambda_3=4$.
The eigenvectors associated
 with $\lambda_1=-2$ satisfy the system with  augmented matrix
$ \begin{bmatrix}8&-5&3&\vdots&0\\2&1&3&
\vdots&0\\2&1&3&\vdots&0\end{bmatrix}$,
which is row equivalent to
$ \begin{bmatrix}1&0&1&\vdots&0\\0&1&1&
\vdots&0\\0&0&0&\vdots&0\end{bmatrix}$.
Hence  $x_1=-x_3$ and $x_2=-x_3$.  Taking $x_3=1$
yields
${\bf y}_1= \begin{bmatrix}-1\\-1\\1\end{bmatrix}e^{-2t}$.
The eigenvectors associated
 with $\lambda_2=4$ satisfy the system with  augmented matrix
$ \begin{bmatrix}2&-5&3&\vdots&0\\2&-5&3&
\vdots&0\\2&1&-3&\vdots&0\end{bmatrix}$,
which is row equivalent to
$ \begin{bmatrix}1&0&-1&\vdots&0\\0&1&-1&
\vdots&0\\0&0&0&\vdots&0\end{bmatrix}$.
Hence  $x_1=x_3$ and $x_2=x_3$.  Taking $x_3=1$
yields
${\bf y}_2= \begin{bmatrix}1\\1\\1\end{bmatrix}e^{4t}$.
For a third solution we need a vector ${\bf u}$ such that
$ \begin{bmatrix}2&-5&3\\2&-5&3\\2&1&-3\end{bmatrix}\begin{bmatrix}u_1\\u_2\\u_3\end{bmatrix}=
 \begin{bmatrix}1\\1\\1\end{bmatrix}$.
The augmented matrix of this system is row equivalent to
$ \begin{bmatrix}1&0&-1&\vdots&1/2\\0&1&-1&
\vdots&0\\0&0&0&\vdots&0\end{bmatrix}$.
Let $u_3=0$, $u_1=\frac{1 }{2}$, and $u_2=0$. Then ${\bf
y}_3= \begin{bmatrix}1\\0\\0\end{bmatrix}\frac{e^{4t} }{2}+ \begin{bmatrix}1\\1\\1\end{bmatrix}te^{4t}$. The
general solution is
 ${\bf y}= 
c_1\begin{bmatrix}-1\\-1\\1\end{bmatrix}e^{-2t}+c_2\begin{bmatrix}1\\1\\1\end{bmatrix}e^{4t}+c_3\left
(\begin{bmatrix}1\\0\\0\end{bmatrix}\frac{e^{4t} }{2}+\begin{bmatrix}1\\1\\1\end{bmatrix}te^{4t}\right)$.

 \end{solution}
 \end{problem}


 \begin{problem}\label{exer:10.5.13}
 Solve the initial value problem.
 
 $ {\bf
y}'= \begin{bmatrix}-11&\\-2&-3\end{bmatrix}{\bf y} ,\quad{\bf y}(0)= \begin{bmatrix}6\\2\end{bmatrix}$
 \end{problem}

 \begin{problem}\label{exer:10.5.14}
 Solve the initial value problem.
 
 $ {\bf
y}'= \begin{bmatrix}15&-9\\16&-9\end{bmatrix}{\bf y} ,\quad{\bf y}(0)= \begin{bmatrix}5\\8\end{bmatrix}$

\begin{solution}
    $ \begin{vmatrix}15-\lambda&-9\\16&-9-\lambda\end{vmatrix}=(\lambda-3)^2$.
Hence $\lambda_1=3$.
Eigenvectors   satisfy
$ \begin{bmatrix}12&-9\\16&-12\end{bmatrix}\begin{bmatrix}x_1\\x_2\end{bmatrix}=\begin{bmatrix}0\\0\end{bmatrix}$,
so $x_1=\frac{3 }{4}x_2$.  Taking $x_2=4$ yields
${\bf y}_1= \begin{bmatrix}3\\4\end{bmatrix}e^{3t}$.
For a second solution we need a vector ${\bf u}$
such that $ \begin{bmatrix}12&-9\\16&-12\end{bmatrix}\begin{bmatrix}u_1\\u_2\end{bmatrix}
=\begin{bmatrix}3\\4\end{bmatrix}$. Let $u_1=\frac{1 }{4}$ and $u_2=0$. Then
${\bf y}_2= \begin{bmatrix}1\\0\end{bmatrix}\frac{e^{3t} }{4}+ \begin{bmatrix}3\\4\end{bmatrix}te^{3t}$.
The general solution is
${\bf y}=c_1 \begin{bmatrix}3\\4\end{bmatrix}e^{3t}+
c_2\left( \begin{bmatrix}1\\0\end{bmatrix}\frac{e^{3t} }{4}+ \begin{bmatrix}3\\4\end{bmatrix}te^{3t}\right)$.
Now ${\bf y}(0)= \begin{bmatrix}5\\8\end{bmatrix}\Rightarrow
c_1 \begin{bmatrix}3\\4\end{bmatrix}+
c_2 \begin{bmatrix}1/4\\0\end{bmatrix}= \begin{bmatrix}5\\8\end{bmatrix}$,  so $c_1=2$
and $c_2=-4$. Therefore,
${\bf y}= \begin{bmatrix}5\\8\end{bmatrix}e^{3t}-\begin{bmatrix}12\\16\end{bmatrix}te^{3t}$.
\end{solution}
 \end{problem}

 \begin{problem}\label{exer:10.5.15} 
 Solve the initial value problem.
 
 $ {\bf y}'= \begin{bmatrix}-3&-4\\1&-7\end{bmatrix}{\bf y},\quad{\bf y}(0)= \begin{bmatrix}2\\3\end{bmatrix}$
 \end{problem}

 \begin{problem}\label{exer:10.5.16} 
 Solve the initial value problem.
 
 $ {\bf
y}'= \begin{bmatrix}-7&24\\-6&17\end{bmatrix}{\bf y} ,\quad{\bf y}(0)= \begin{bmatrix}3\\1\end{bmatrix}$

\begin{solution}
    $ \begin{vmatrix}-7-\lambda&24\\-6&17-\lambda\end{vmatrix}=(\lambda-5)^2$.
Hence $\lambda_1=5$.
Eigenvectors   satisfy
$  \begin{bmatrix}-12&24\\-6&12\end{bmatrix} \begin{bmatrix}x_1\\x_2\end{bmatrix}= \begin{bmatrix}0\\0\end{bmatrix}$,
so $x_1=2x_2$.  Taking $x_2=1$ yields
${\bf y}_1=  \begin{bmatrix}2\\1\end{bmatrix}e^{5t}$.
For a second solution we need a vector ${\bf u}$
such that $  \begin{bmatrix}-12&24\\-6&12\end{bmatrix} \begin{bmatrix}u_1\\u_2\end{bmatrix}
=  \begin{bmatrix}2\\1\end{bmatrix}$. Let $u_1=\frac{1 }{6}$ and $u_2=0$. Then
${\bf y}_2=  \begin{bmatrix}1\\0\end{bmatrix}\frac{e^{5t} }{6}+  \begin{bmatrix}2\\1\end{bmatrix}te^{5t}$.
The general solution is
${\bf y}=c_1  \begin{bmatrix}2\\1\end{bmatrix}e^{5t}
+c_2\left(  \begin{bmatrix}1\\0\end{bmatrix}\frac{e^{5t} }{6}+  \begin{bmatrix}2\\1\end{bmatrix}te^{5t}\right)$.
Now
${\bf y}(0)=  \begin{bmatrix}3\\1\end{bmatrix}\Rightarrow c_1  \begin{bmatrix}2\\1\end{bmatrix}
+c_2 \begin{bmatrix}1/6\\0\end{bmatrix}=  \begin{bmatrix}3\\1\end{bmatrix}$, so $c_1=1$
and $c_2=6$. Therefore,
${\bf y}=  \begin{bmatrix}3\\1\end{bmatrix}e^{5t}- \begin{bmatrix}12\\6\end{bmatrix}te^{5t}$.
\end{solution}
 \end{problem}

 \begin{problem}\label{exer:10.5.17} 
 Solve the initial value problem.
 
 $ {\bf y}'= \begin{bmatrix}-7&3\\-3&-1\end{bmatrix}{\bf
y} ,\quad{\bf y}(0)= \begin{bmatrix}0\\2\end{bmatrix}$
 \end{problem}

 \begin{problem}\label{exer:10.5.18}  
 Solve the initial value problem.
 
 $ {\bf y}'
= \begin{bmatrix}-1&1&0\\1&-1&-2\\-1&-1&-1\end{bmatrix}{\bf y},\quad
{\bf y}(0)= \begin{bmatrix}6\\5\\-7\end{bmatrix}$

\begin{solution}
    $ \begin{vmatrix}-1-\lambda&1&0\\1&-1-\lambda&-2\\-1&-1&-1\end{vmatrix}
=-(\lambda-1)(\lambda+2)^2$.
Hence $\lambda_1=1$ and  $\lambda_2=\lambda_3=-2$.
The eigenvectors associated
 with $\lambda_1=1$ satisfy the system with  augmented matrix
$  \begin{bmatrix}{-2}&1&0&\vdots&0\\1&-2&-2&
\vdots&0\\-1&-1&-2&\vdots&0 \end{bmatrix}$,
which is row equivalent to
$  \begin{bmatrix}1&0&2/3&\vdots&0\\0&1&4/3&
\vdots&0\\0&0&0&\vdots&0 \end{bmatrix}$.
Hence  $x_1=-\frac{2 }{3}x_3$ and $x_2=-\frac{4 }{3}x_3$.  Taking
$x_3=3$ yields
The eigenvectors associated
 with $\lambda_2=-2$ satisfy the system with  augmented matrix
$  \begin{bmatrix}1&1&0&\vdots&0\\1&1&-2&
\vdots&0\\-1&-1&1&\vdots&0 \end{bmatrix}$,
which is row equivalent to
$  \begin{bmatrix}1&1&0&\vdots&0\\0&0&1&
\vdots&0\\0&0&0&\vdots&0 \end{bmatrix}$.
Hence  $x_1=-x_2$ and $x_3=0$.  Taking $x_2=1$
yields
${\bf y}_2=  \begin{bmatrix}-1\\1\\0\end{bmatrix}e^{-2t}$.
For a third solution we need a vector ${\bf u}$ such that
$  \begin{bmatrix}1&1&0\\1&1&-2\\-1&-1&1\end{bmatrix} \begin{bmatrix}u_1\\u_2\\u_3\end{bmatrix}=
  \begin{bmatrix}-1\\1\\0\end{bmatrix}$.
The augmented matrix of this system is row equivalent to
$  \begin{bmatrix}1&1&0&\vdots&-1\\0&0&1&
\vdots&-1\\0&0&0&\vdots&0 \end{bmatrix}$.
Let $u_2=0$, $u_1=-1$, and $u_3=-1$. Then ${\bf
y}_3=  \begin{bmatrix}-1\\0\\-1\end{bmatrix}e^{-2t}+  \begin{bmatrix}-1\\1\\0\end{bmatrix}te^{-2t}$.
The general solution is
${\bf y}=c_1  \begin{bmatrix}-2\\-4\\3\end{bmatrix}e^t+
c_2  \begin{bmatrix}-1\\1\\0\end{bmatrix}e^{-2t}
+c_3\left(  \begin{bmatrix}-1\\0\\-1\end{bmatrix}e^{-2t}+
  \begin{bmatrix}-1\\1\\0\end{bmatrix}te^{-2t}\right)$.
Now ${\bf y}(0)=  \begin{bmatrix}6\\5\\-7\end{bmatrix}\Rightarrow
c_1  \begin{bmatrix}-2\\-4\\3\end{bmatrix}+ c_2  \begin{bmatrix}-1\\1\\0\end{bmatrix}
+c_3  \begin{bmatrix}-1\\0\\-1\end{bmatrix}=  \begin{bmatrix}6\\5\\-7\end{bmatrix}$,
so $c_1=-2$, $c_2=-3$, and $c_3=1$.  Therefore,
${\bf y}=  \begin{bmatrix}4\\8\\-6\end{bmatrix}e^t+ \begin{bmatrix}2\\-3\\-1\end{bmatrix}e^{-2t}+
 \begin{bmatrix}-1\\1\\0\end{bmatrix}te^{-2t}$.

\end{solution}
 \end{problem}

 \begin{problem}\label{exer:10.5.19}  
 Solve the initial value problem.
 
 $ {\bf y}'
= \begin{bmatrix}-2&2&1\\-2&2&1\\-3&3&2\end{bmatrix}{\bf y},\quad
{\bf y}(0)= \begin{bmatrix}-6\\-2\\0\end{bmatrix}$
 \end{problem}

 \begin{problem}\label{exer:10.5.20}
 Solve the initial value problem.
 
 $ {\bf y}'
= \begin{bmatrix}-7&-4&4\\-1&0&1\\-9&-5&6\end{bmatrix}{\bf y},\quad\bf
{\bf y}(0)= \begin{bmatrix}-6\\9\\-1\end{bmatrix}$

\begin{solution}
    $ \begin{vmatrix}-7-\lambda&-4&4\\-1&-\lambda&1\\-9&-5&6-\lambda\end{vmatrix}
=-(\lambda+3)(\lambda-1)^2$.
Hence $\lambda_1=-3$ and  $\lambda_2=\lambda_3=1$.
The eigenvectors associated
 with $\lambda_1=-3$ satisfy the system with  augmented matrix
$  \begin{bmatrix}-4&-4&4&\vdots&0\\-1&3&1&
\vdots&0\\-9&-5&9&\vdots&0 \end{bmatrix}$,
which is row equivalent to
$  \begin{bmatrix}1&0&-1&\vdots&0\\0&1&0&
\vdots&0\\0&0&0&\vdots&0 \end{bmatrix}$.
Hence  $x_1=x_3$ and $x_2=0$.  Taking $x_3=1$
yields
${\bf y}_1=  \begin{bmatrix}1\\0\\1\end{bmatrix}e^{-3t}$.
The eigenvectors associated
 with $\lambda_2=1$ satisfy the system with  augmented matrix
$  \begin{bmatrix}-8&-4&4&\vdots&0\\-1&-1&1&
\vdots&0\\-9&-5&-5&\vdots&0 \end{bmatrix}$,
which is row equivalent to
$  \begin{bmatrix}1&0&0&\vdots&0\\0&1&-1&
\vdots&0\\0&0&0&\vdots&0 \end{bmatrix}$.
Hence  $x_1=0$ and $x_2=x_3$.  Taking $x_3=1$
yields
${\bf y}_2=  \begin{bmatrix}0\\1\\1\end{bmatrix}e^t$.
For a third solution we need a vector ${\bf u}$ such that
$  \begin{bmatrix}-8&-4&4\\-1&-1&1\\-9&-5&5\end{bmatrix} \begin{bmatrix}u_1\\u_2\\u_3\end{bmatrix}
=  \begin{bmatrix}0\\1\\1\end{bmatrix}$.
The augmented matrix of this system is row equivalent to
$  \begin{bmatrix}1&0&0&\vdots&1\\0&1&-1&
\vdots&-2\\0&0&0&\vdots&0 \end{bmatrix}$.
Let $u_3=0$, $u_1=1$, and $u_2=-2$. Then ${\bf
y}_3=  \begin{bmatrix}1\\-2\\0\end{bmatrix}e^t+  \begin{bmatrix}0\\1\\1\end{bmatrix}te^{t}$. The general
solution is
${\bf y}=c_1  \begin{bmatrix}1\\0\\1\end{bmatrix}e^{-3t}+ c_2  \begin{bmatrix}0\\1\\1\end{bmatrix}e^t+
c_3\left(  \begin{bmatrix}1\\-2\\0\end{bmatrix}e^t+  \begin{bmatrix}0\\1\\1\end{bmatrix}te^{t}\right)$.
Now ${\bf y}(0)=  \begin{bmatrix}-6\\9\\-1\end{bmatrix}\Rightarrow
c_1  \begin{bmatrix}1\\0\\1\end{bmatrix}+ c_2  \begin{bmatrix}0\\1\\1\end{bmatrix}+
c_3  \begin{bmatrix}1\\-2\\0\end{bmatrix}=  \begin{bmatrix}-6\\9\\-1\end{bmatrix}$,
so $c_1=-2$, $c_2=1$, and $c_3=-4$.  Therefore,
${\bf y}= - \begin{bmatrix}2\\0\\2\end{bmatrix}e^{-3t}+ \begin{bmatrix}-4\\9\\1\end{bmatrix}e^t-
 \begin{bmatrix}0\\4\\4\end{bmatrix}te^t$.
\end{solution}
 \end{problem}

 \begin{problem}\label{exer:10.5.21}
 Solve the initial value problem.
 
 $ {\bf y}'
= \begin{bmatrix}-1&-4&-1\\3&6&1\\-3&-2&3\end{bmatrix}\bf y,\quad\bf
y(0)= \begin{bmatrix}-2\\1\\3\end{bmatrix}$
 \end{problem}

 \begin{problem}\label{exer:10.5.22} 
 Solve the initial value problem.
 
 $ {\bf y}'
= \begin{bmatrix}4&-8&-4\\-3&-1&-3\\1&-1&9\end{bmatrix}{\bf y},\quad
{\bf y}(0)= \begin{bmatrix}-4\\1\\-3\end{bmatrix}$

\begin{solution}
    $ \begin{vmatrix}4-\lambda&-8&-4\\-3&-1-\lambda&-3\\1&-1&9-\lambda\end{vmatrix}
=-(\lambda+4)(\lambda-8)^2$.
Hence $\lambda_1=-4$ and  $\lambda_2=\lambda_3=8$.
The eigenvectors associated
 with $\lambda_1=-4$ satisfy the system with  augmented matrix
$  \begin{bmatrix}8&-8&-4&\vdots&0\\-3&3&-3&
\vdots&0\\1&-1&13&\vdots&0 \end{bmatrix}$,
which is row equivalent to
$  \begin{bmatrix}1&-1&0&\vdots&0\\0&0&1&
\vdots&0\\0&0&0&\vdots&0 \end{bmatrix}$.
Hence  $x_1=x_2$ and $x_3=0$.  Taking $x_2=1$
yields
${\bf y}_1=  \begin{bmatrix}1\\1\\0\end{bmatrix}e^{t}$.
The eigenvectors associated
 with $\lambda_2=8$ satisfy the system with  augmented matrix
$  \begin{bmatrix}-4&-8&-4&\vdots&0\\-3&-9&-3&
\vdots&0\\1&-1&1&\vdots&0 \end{bmatrix}$,
which is row equivalent to
$  \begin{bmatrix}1&0&1&\vdots&0\\0&1&0&
\vdots&0\\0&0&0&\vdots&0 \end{bmatrix}$.
Hence  $x_1=-x_3$ and $x_2=0$.  Taking $x_3=1$
yields
${\bf y}_2=  \begin{bmatrix}-1\\0\\1\end{bmatrix}e^{8t}$.
For a third solution we need a vector ${\bf u}$ such that
$  \begin{bmatrix}-4&-8&-4\\-3&-9&-3\\1&-1&1\end{bmatrix}
 \begin{bmatrix}u_1\\u_2\\u_3\end{bmatrix}=  \begin{bmatrix}-1\\0\\1\end{bmatrix}$.
The augmented matrix of this system is row equivalent to
$  \begin{bmatrix}1&0&1&\vdots&3/4\\0&1&0&
\vdots&-1/4\\0&0&0&\vdots&0 \end{bmatrix}$.
Let $u_3=0$, $u_1= \frac{3}{4}$, and $u_2=- \frac{1}{4}$. Then ${\bf
y}_3=  \begin{bmatrix}3\\-1\\0\end{bmatrix}\frac{e^{8t}}{4}+  \begin{bmatrix}-1\\0\\1\end{bmatrix}te^{8t}$.
The general solution is
$c_1  \begin{bmatrix}1\\1\\0\end{bmatrix}e^{t}+
c_2  \begin{bmatrix}-1\\0\\1\end{bmatrix}e^{8t}+
c_3\left(\begin{bmatrix}3\\-1\\0\end{bmatrix}\frac{e^{8t}}{4}+
  \begin{bmatrix}-1\\0\\1\end{bmatrix}te^{8t}\right)$.
Now ${\bf y}(0)=  \begin{bmatrix}-4\\1\\-3\end{bmatrix}\Rightarrow
c_1  \begin{bmatrix}1\\1\\0\end{bmatrix}+
c_2  \begin{bmatrix}-1\\0\\1\end{bmatrix}+
c_3  \begin{bmatrix}3/4\\-1/4\\0\end{bmatrix}=  \begin{bmatrix}-4\\1\\-3\end{bmatrix}$,
so $c_1=-1$, $c_2=-3$, and $c_3=-8$.  Therefore,
${\bf y}= - \begin{bmatrix}1\\1\\0\end{bmatrix}e^{-4t}+ \begin{bmatrix}-3\\2\\-3\end{bmatrix}e^{8t}+
 \begin{bmatrix}8\\0\\-8\end{bmatrix}te^{8t}$.

\end{solution}
 \end{problem}

 \begin{problem}\label{exer:10.5.23}  
 Solve the initial value problem.
 
 $ {\bf y}'=
 \begin{bmatrix}-5&-1&11\\-7&1&13\\-4&0&8\end{bmatrix}{\bf y},\quad
{\bf y}(0)= \begin{bmatrix}0\\2\\2\end{bmatrix}$
 \end{problem}

 \begin{problem}\label{exer:10.5.24}
 The coefficient matrix in this exercise
has eigenvalue of multiplicity $3$. Find the
general solution.

$ {\bf y}'
= \begin{bmatrix}5&-1&1\\-1&9&-3\\-2&2&4\end{bmatrix}{\bf y}$

\begin{solution}
    $ \begin{vmatrix}5-\lambda&-1&1\\-1&9-\lambda&-3\\-2&2&4-\lambda\end{vmatrix}
=-(\lambda-6)^3$.
Hence $\lambda_1=6$. The eigenvectors
 satisfy the system with  augmented matrix
$  \begin{bmatrix}1&-1&1&\vdots&0\\-1&3&-3&
\vdots&0\\-2&2&-2&\vdots&0 \end{bmatrix}$,
which is row equivalent to
$  \begin{bmatrix}1&0&0&\vdots&0\\0&1&-1&
\vdots&0\\0&0&0&\vdots&0 \end{bmatrix}$.
Hence  $x_1=0$ and $x_2=x_3$.  Taking $x_3=1$
yields
${\bf y}_1=  \begin{bmatrix}0\\1\\1 \end{bmatrix}e^{6t}$.
For a second solution we need a vector ${\bf u}$ such that
$  \begin{bmatrix}1&-1&1\\-1&3&-3\\-2&2&-2 \end{bmatrix} \begin{bmatrix}u_1\\u_2\\u_3 \end{bmatrix}
=  \begin{bmatrix}0\\1\\1 \end{bmatrix}$.
The augmented matrix of this system is row equivalent to
$  \begin{bmatrix}1&0&0&\vdots&-1/4\\0&1&-1&
\vdots&1/4\\0&0&0&\vdots&0 \end{bmatrix}$.
Let $u_3=0$, $u_1=-\frac{1 }{4}$, and $u_2=\frac{1 }{4}$. Then ${\bf
y}_2=  \begin{bmatrix}-1\\1\\0 \end{bmatrix}\frac{e^{6t} }{4}+  \begin{bmatrix}0\\1\\1 \end{bmatrix}te^{6t}$.
For a third solution we need a vector ${\bf v}$ such that
$  \begin{bmatrix}1&-1&1\\-1&3&-3\\-2&2&-2 \end{bmatrix} \begin{bmatrix}v_1\\v_2\\v_3 \end{bmatrix}
=  \begin{bmatrix}-1/4\\1/4\\0 \end{bmatrix}$.
The augmented matrix of this system is row equivalent to
$  \begin{bmatrix}1&0&0&\vdots&1/8\\0&1&-1&
\vdots&1/8\\0&0&0&\vdots&0 \end{bmatrix}$.
Let $v_3=0$, $v_1= \frac{1 }{8}$, and $v_2= \frac{1 }{8}$. Then
${\bf y}_3=   \begin{bmatrix}1\\1\\0 \end{bmatrix}\frac{e^{6t} }{8}+
  \begin{bmatrix}-1\\1\\0 \end{bmatrix}\frac{te^{6t} }{4}+  \begin{bmatrix}0\\1\\1 \end{bmatrix}\frac{t^2e^{6t} }{2}$.
The general solution is
${\bf y}= c_1 \begin{bmatrix}0\\1\\1 \end{bmatrix}e^{6t}+
c_2\left( \begin{bmatrix}-1\\1\\0 \end{bmatrix}\frac{e^{6t} }{4}+ \begin{bmatrix}0\\1\\1 \end{bmatrix}te^{6t}\right)$
$$+c_3\left( \begin{bmatrix}1\\1\\0 \end{bmatrix}\frac{e^{6t} }{8}+ \begin{bmatrix}-1\\1\\0 \end{bmatrix}\frac{t
e^{6t} }{4}+ \begin{bmatrix}0\\1\\1 \end{bmatrix}\frac{t^2e^{6t} }{2}\right).$$


\end{solution}
 \end{problem}

 \begin{problem}\label{exer:10.5.25}
 The coefficient matrix in this exercise
has eigenvalue of multiplicity $3$. Find the
general solution.

$ {\bf y}'
= \begin{bmatrix}1&10&-12\\2&2&3\\2&-1&6\end{bmatrix}{\bf y}$
 \end{problem}

 \begin{problem}\label{exer:10.5.26} 
 The coefficient matrix in this exercise
has eigenvalue of multiplicity $3$. Find the
general solution.

$ {\bf y}'
= \begin{bmatrix}-6 &-4&-4\\2&-1&1\\2&3&1\end{bmatrix}{\bf y}$

\begin{solution}
  $ \begin{vmatrix}-6-\lambda&-4&-4\\2&-1-\lambda&1\\2&3&1-\lambda\end{vmatrix}
=-(\lambda+2)^3$.
Hence $\lambda_1=-2$.
The eigenvectors
 satisfy the system with  augmented matrix
$  \begin{bmatrix}-4&-4&-4&\vdots&0\\2&1&1&
\vdots&0\\2&3&3&\vdots&0 \end{bmatrix}$,
which is row equivalent to
$  \begin{bmatrix}1&0&0&\vdots&0\\0&1&1&
\vdots&0\\0&0&0&\vdots&0 \end{bmatrix}$.
Hence  $x_1=0$ and $x_2=-x_3$.  Taking $x_3=1$
yields
${\bf y}_1=  \begin{bmatrix}0\\-1\\1 \end{bmatrix}e^{-2t}$.
For a second solution we need a vector ${\bf u}$ such that
$  \begin{bmatrix}-4&-4&-4\\2&1&1\\2&3&3 \end{bmatrix} \begin{bmatrix}u_1\\u_2\\u_3 \end{bmatrix}
=  \begin{bmatrix}0\\-1\\1 \end{bmatrix}$.
The augmented matrix of this system is row equivalent to
$  \begin{bmatrix}1&0&0&\vdots&-1\\0&1&1&
\vdots&1\\0&0&0&\vdots&0 \end{bmatrix}$.
Let $u_3=0$, $u_1=-1$, and $u_2=1$. Then
${\bf y}_2= \begin{bmatrix}-1\\1\\0 \end{bmatrix}e^{-2t}+ \begin{bmatrix}0\\-1\\1 \end{bmatrix}te^{-2t}$.
For a third solution we need a vector ${\bf v}$ such that
$  \begin{bmatrix}-4&-4&-4\\2&1&1\\2&3&3 \end{bmatrix} \begin{bmatrix}v_1\\v_2\\v_3 \end{bmatrix}
=  \begin{bmatrix}-1\\1\\0 \end{bmatrix}$.
The augmented matrix of this system is row equivalent to
$  \begin{bmatrix}1&0&0&\vdots&3/4\\0&1&1&
\vdots&-1/2\\0&0&0&\vdots&0 \end{bmatrix}$.
Let $v_3=0$, $v_1=\frac{3 }{4}$, and $v_2=-\frac{1 }{2}$. Then
${\bf y}_3=  \begin{bmatrix}3\\-2\\0 \end{bmatrix} \frac{e^{-2t} }{4}+ \begin{bmatrix}-1\\1\\0 \end{bmatrix}t
e^{-2t}+ \begin{bmatrix}0\\-1\\1 \end{bmatrix}\frac{t^2e^{-2t} }{2}$.
The general solution is
${\bf y}= c_1 \begin{bmatrix}0\\-1\\1 \end{bmatrix}e^{-2t}+
c_2\left( \begin{bmatrix}-1\\1\\0 \end{bmatrix}e^{-2t}+ \begin{bmatrix}0\\-1\\1 \end{bmatrix}te^{-2t}\right)$
$$+c_3\left( \begin{bmatrix}3\\-2\\0 \end{bmatrix}\frac{e^{-2t} }{4}+ \begin{bmatrix}-1\\1\\0 \end{bmatrix}t
e^{-2t}+ \begin{bmatrix}0\\-1\\1 \end{bmatrix}\frac{t^2e^{-2t} }{2}\right)$$


\end{solution}
 \end{problem}

 \begin{problem}\label{exer:10.5.27} 
 The coefficient matrix in this exercise
has eigenvalue of multiplicity $3$. Find the
general solution.

$ {\bf y}'
= \begin{bmatrix}0&2&-2\\-1&5&-3\\1&1&1\end{bmatrix}{\bf y}$
 \end{problem}


 \begin{problem}\label{exer:10.5.28}
 The coefficient matrix in this exercise
has eigenvalue of multiplicity $3$. Find the
general solution.

$ {\bf y}'
= \begin{bmatrix}-2&-12&10\\2&-24&11\\2&-24&8\end{bmatrix}{\bf y}$

\begin{solution}
    $ \begin{vmatrix}-2-\lambda&-12&10\\2&-24-\lambda&11\\2&-24&8-\lambda\end{vmatrix}
=-(\lambda+6)^3$.
Hence $\lambda_1=-6$.
The eigenvectors
 satisfy the system with  augmented matrix
$ \begin{bmatrix}4&-12&10&\vdots&0\\2&-18&11&
\vdots&0\\2&-24&14&\vdots&0\end{bmatrix}$,
which is row equivalent to
$ \begin{bmatrix}1&0&1&\vdots&0\\0&1&-1/2&
\vdots&0\\0&0&0&\vdots&0\end{bmatrix}$.
Hence  $x_1=-x_3$ and $x_2= \frac{x_3}{2}$.  Taking $x_3=2$
yields
${\bf y}_1= \begin{bmatrix}-2\\1\\2\end{bmatrix}e^{-6t}$.
For a second solution we need a vector ${\bf u}$ such that
$ \begin{bmatrix}4&-12&10\\2&-18&11\\2&-24&14\end{bmatrix}\begin{bmatrix}u_1\\u_2\\u_3\end{bmatrix}
= \begin{bmatrix}-2\\1\\2\end{bmatrix}$.
The augmented matrix of this system is row equivalent to
$ \begin{bmatrix}1&0&1&\vdots&-1\\0&1&-1/2&
\vdots&-1/6\\0&0&0&\vdots&0\end{bmatrix}$.
Let $u_3=0$, $u_1=-1$, and $u_2=- \frac{1 }{6}$. Then
${\bf y}_2=-\begin{bmatrix}6\\1\\0\end{bmatrix} \frac{e^{-6t} }{6}+\begin{bmatrix}-2\\1\\2\end{bmatrix}te^{-6t}$.
For a third solution we need a vector ${\bf v}$ such that
$ \begin{bmatrix}4&-12&10\\2&-18&11\\2&-24&14\end{bmatrix}\begin{bmatrix}v_1\\v_2\\v_3\end{bmatrix}
= \begin{bmatrix}-1\\-1/6\\0\end{bmatrix}$.
The augmented matrix of this system is row equivalent to
$ \begin{bmatrix}1&0&1&\vdots&-1/3\\0&1&-1/2&
\vdots&-1/36\\0&0&0&\vdots&0\end{bmatrix}$.
Let $v_3=0$, $v_1=- \frac{1 }{3}$, and $v_2=- \frac{1 }{36}$. Then
${\bf y}_3= -\begin{bmatrix}12\\1\\0\end{bmatrix} \frac{e^{-6t} }{36}-\begin{bmatrix}6\\1\\0\end{bmatrix}\frac{t
e^{-6t} }{6}+\begin{bmatrix}-2\\1\\2\end{bmatrix}\frac{t^2e^{-6t} }{2}$.
The general solution is
 ${\bf y}= c_1\begin{bmatrix}-2\\1\\2\end{bmatrix}e^{-6t}+
c_2\left(-\begin{bmatrix}6\\1\\0\end{bmatrix}\frac{e^{-6t} }{6}+\begin{bmatrix}-2\\1\\2\end{bmatrix}te^{-6t}\right)$
$$+c_3\left(-\begin{bmatrix}12\\1\\0\end{bmatrix}\frac{e^{-6t} }{36}-\begin{bmatrix}6\\1\\0\end{bmatrix}\frac{t
e^{-6t} }{6}+\begin{bmatrix}-2\\1\\2\end{bmatrix}\frac{t^2e^{-6t} }{2}\right).$$

\end{solution}
 \end{problem}

 \begin{problem}\label{exer:10.5.29} 
 The coefficient matrix in this exercise
has eigenvalue of multiplicity $3$. Find the
general solution.

$ {\bf y}'
= \begin{bmatrix}-1&-12&8\\1&-9&4\\1&-6&1\end{bmatrix}{\bf y}$
 \end{problem}

 \begin{problem}\label{exer:10.5.30}  
 The coefficient matrix in this exercise
has eigenvalue of multiplicity $3$. Find the
general solution.

$ {\bf y}'
= \begin{bmatrix}-4&0&-1\\-1&-3&-1\\1&0&-2\end{bmatrix}{\bf y}$

\begin{solution}
    $ \begin{vmatrix}-4-\lambda&0&-1\\-1&-3-\lambda&-1\\1&0&-2-\lambda\end{vmatrix}
=-(\lambda+3)^3$.
Hence $\lambda_1=3$.
The eigenvectors
 satisfy the system with  augmented matrix
$  \begin{bmatrix}-1&0&-1&\vdots&0\\-1&0&-1&
\vdots&0\\1&0&1&\vdots&0 \end{bmatrix}$,
which is row equivalent to
$  \begin{bmatrix}1&0&1&\vdots&0\\0&0&0&
\vdots&0\\0&0&0&\vdots&0 \end{bmatrix}$.
Hence  $x_1=-x_3$ and $x_2$ is arbitrary.  Taking $x_2=0$ and $x_3=1$
yields
${\bf y}_1= \begin{bmatrix}-1\\0\\1 \end{bmatrix}e^{-3t}$.
Taking $x_2=1$ and $x_3=0$
yields  ${\bf y}_2= \begin{bmatrix}0\\1\\0 \end{bmatrix}e^{-3t}$.
For a third solution we need constants $\alpha$ and $\beta$  and a
vector ${\bf u}$ such that
$  \begin{bmatrix}-1&0&-1\\-1&0&-1\\1&0&1 \end{bmatrix} \begin{bmatrix}u_1\\u_2\\u_3 \end{bmatrix}
=\alpha  \begin{bmatrix}-1\\0\\1 \end{bmatrix}+\beta \begin{bmatrix}0\\1\\0 \end{bmatrix}$.
The augmented matrix of this system is row equivalent to
$  \begin{bmatrix}1&0&1&\vdots&\alpha\\0&0&0&
\vdots&\alpha+\beta\\0&0&0&\vdots&0 \end{bmatrix}$;
hence the system
has a solution if $\alpha=-\beta=1$, which yields the eigenvector
${\bf x}_3= \begin{bmatrix}-1\\-1\\1 \end{bmatrix}$. Taking $u_1=1$  and $u_2=u_3=0$
yields the solution
${\bf y}_3= \begin{bmatrix}1\\0\\0 \end{bmatrix}e^{-3t}+ \begin{bmatrix}-1\\-1\\1 \end{bmatrix}te^{-3t}$.
The general solution is
${\bf
y}= c_1 \begin{bmatrix}-1\\0\\1 \end{bmatrix}e^{-3t}+c_2 \begin{bmatrix}0\\1\\0 \end{bmatrix}e^{-3t}+c_3\left
( \begin{bmatrix}1\\0\\0 \end{bmatrix}e^{-3t}+ \begin{bmatrix}-1\\-1\\1 \end{bmatrix}te^{-3t}\right)$
\end{solution}
 \end{problem}

 \begin{problem}\label{exer:10.5.31}  
 The coefficient matrix in this exercise
has eigenvalue of multiplicity $3$. Find the
general solution.

$ {\bf y}'
= \begin{bmatrix}-3&-3&4\\4&5&-8\\2&3&-5\end{bmatrix}\bf y$
 \end{problem}

 \begin{problem}\label{exer:10.5.32}
 The coefficient matrix in this exercise
has eigenvalue of multiplicity $3$. Find the
general solution.

${\bf y}'= \begin{bmatrix}-3&-1&0\\1&-1&0\\-1&-1&-2\end{bmatrix}{\bf y}$

\begin{solution}
    $ \begin{vmatrix}-3-\lambda&-1&0\\1&-1-\lambda&0\\-1&-1&-2-\lambda\end{vmatrix}
=-(\lambda+2)^3$.
Hence $\lambda_1=-2$.
The eigenvectors
 satisfy the system with  augmented matrix
$  \begin{bmatrix}-1&-1&0&\vdots&0\\1&1&0&
\vdots&0\\-1&-1&0&\vdots&0 \end{bmatrix}$,
which is row equivalent to
$  \begin{bmatrix}1&1&0&\vdots&0\\0&0&0&
\vdots&0\\0&0&0&\vdots&0 \end{bmatrix}$.
Hence  $x_1=-x_2$ and $x_3$ is arbitrary.  Taking $x_2=1$ and $x_3=0$
yields
${\bf y}_1= \begin{bmatrix}-1\\1\\0 \end{bmatrix}e^{-2t}$.
Taking  $x_2=0$ and $x_3=1$ yields
  ${\bf y}_2= \begin{bmatrix}0\\0\\1 \end{bmatrix}e^{-2t}$.
For a third solution we need constants $\alpha$ and $\beta$  and a
vector ${\bf u}$ such that
$  \begin{bmatrix}-1&-1&0\\1&1&0\\-1&-1&0 \end{bmatrix} \begin{bmatrix}u_1\\u_2\\u_3 \end{bmatrix}
=\alpha  \begin{bmatrix}-1\\1\\0 \end{bmatrix}+\beta \begin{bmatrix}0\\0\\1 \end{bmatrix}$.
The augmented matrix of this system is row equivalent to
$  \begin{bmatrix}1&1&0&\vdots&\alpha\\0&0&0&
\vdots&\alpha+\beta\\0&0&0&\vdots&0 \end{bmatrix}$;
hence the system
has a solution if $\alpha=-\beta=1$, which yields the eigenvector
${\bf x}_3= \begin{bmatrix}-1\\1\\-1 \end{bmatrix}$. Taking $u_1=1$  and $u_2=u_3=0$
yields the solution
${\bf y}_3=
 \begin{bmatrix}1\\0\\0 \end{bmatrix}e^{-2t}+ \begin{bmatrix}-1\\1\\-1 \end{bmatrix}te^{-2t}$.
The general solution is
 ${\bf y}= 
c_1 \begin{bmatrix}-1\\1\\0 \end{bmatrix}e^{-2t}+c_2 \begin{bmatrix}0\\0\\1 \end{bmatrix}e^{-2t}+c_3\left
( \begin{bmatrix}1\\0\\0 \end{bmatrix}e^{-2t}+ \begin{bmatrix}-1\\1\\-1 \end{bmatrix}te^{-2t}\right)$.
\end{solution}
 \end{problem}

 \begin{problem}\label{exer:10.5.33}
Under the assumptions of Theorem~\ref{thmtype:10.5.1}, suppose
${\bf u}$ and $\hat{\bf u}$ are vectors such that
$$
(A-\lambda_1I){\bf u}={\bf x}\quad\mbox{and }\quad
(A-\lambda_1I)\hat{\bf u}={\bf x},
$$
and let
$$
{\bf y}_2={\bf u}e^{\lambda_1t}+{\bf x}te^{\lambda_1t}
\quad\mbox{and }\quad
\hat{\bf y}_2=\hat{\bf u}e^{\lambda_1t}+{\bf x}te^{\lambda_1t}.
$$
Show that ${\bf y}_2-\hat{\bf y}_2$ is a scalar multiple of
${\bf y}_1={\bf x}e^{\lambda_1t}$.
 \end{problem}

 \begin{problem}\label{exer:10.5.34}
Under the assumptions of Theorem~\ref{thmtype:10.5.2}, let
\begin{eqnarray*}
{\bf y}_1 &=&{\bf x} e^{\lambda_1t},\\
{\bf y}_2&=&{\bf u}e^{\lambda_1t}+{\bf x} te^{\lambda_1t},\mbox{
and }\\
{\bf y}_3&=&{\bf v}e^{\lambda_1t}+{\bf u}te^{\lambda_1t}+{\bf
x} \frac{t^2e^{\lambda_1t} }{2}.
\end{eqnarray*}
Complete the proof of Theorem~\ref{thmtype:10.5.2} by showing
that ${\bf y}_3$ is a solution of ${\bf y}'=A{\bf y}$ and
that $\{{\bf y}_1,{\bf y}_2,{\bf y}_3\}$ is linearly independent.

\begin{solution}
    $$
{\bf y}_3'-A{\bf y}_3=$$
$$=(\lambda_1I-A){\bf v}e^{\lambda_1t}
+(\lambda_1I-A){\bf u}te^{\lambda_1t}+{\bf u}e^{\lambda_1t}+(\lambda_1I-A){\bf x}{t^2e^{\lambda_1t}\over2}+{\bf
x}te^{\lambda_1t}=$$
$$=-{\bf u}e^{\lambda_1t}-{\bf x}te^{\lambda_1t}+{\bf
u}e^{\lambda_1t}+0+{\bf x}te^{\lambda_1t}=0.
$$
Now suppose that $c_1{\bf y}_1+c_2{\bf y}_2+c_3{\bf y}_3={\bf 0}$.
Then
$$
 c_1{\bf x}+c_2({\bf u}+t{\bf x})+c_3\left({\bf
v}+t{\bf u}+{t^2\over2}{\bf x}\right)={\bf 0}.
\quad\quad\text{(A)}
$$
Differentiating this twice yields $c_3{\bf x}=0$, so $c_3=0$ since
${\bf x}\ne{\bf 0}$. Therefore, (A) reduces to
(B) $c_1{\bf x}+c_2({\bf u}+t{\bf x})={\bf 0}$.
Differentiating this yields $c_2{\bf x}=0$, so $c_2=0$ since
${\bf x}\neq{\bf 0}$. Therefore, (B) reduces to $c_3{\bf x}={\bf 0}$,
so $c_1=0$ since
${\bf x}\neq {\bf 0}$. Therefore,${\bf y}_1$, ${\bf y}_2$, and ${\bf
y}_3$ are linearly independendent.
\end{solution}
 \end{problem}

 \begin{problem}\label{exer:10.5.35}
Suppose the matrix
$$
A= \begin{bmatrix}a_{11}&a_{12}\\a_{21}&a_{22}
\end{bmatrix}
$$
has a repeated eigenvalue $\lambda_1$ and the associated eigenspace is
one-dimensional. Let
 ${\bf x}$ be a $\lambda_1$-eigenvector of $A$.
Show that if
$(A-\lambda_1I){\bf u}_1={\bf x}$ and
$(A-\lambda_1I){\bf u}_2={\bf x}$, then ${\bf u}_2-{\bf u}_1$
is parallel to ${\bf x}$. Conclude from this that all vectors ${\bf
u}$
such that $(A-\lambda_1I){\bf u}={\bf x}$ define the same positive and
negative half-planes with respect to the line $L$
 through the origin parallel to ${\bf x}$.
  \end{problem}

 \begin{problem}\label{exer:10.5.36}
 Plot trajectories of the given system.
 
 ${\bf y}'=  \begin{bmatrix}-3&-1\\4&1\end{bmatrix}{\bf y}$
  \end{problem}

 \begin{problem}\label{exer:10.5.37}  
 Plot trajectories of the given system.
 
 ${\bf y}'=  \begin{bmatrix}2&-1\\1&0\end{bmatrix}{\bf y}$
  \end{problem}


 \begin{problem}\label{exer:10.5.38} 
 Plot trajectories of the given system.
 
 ${\bf y}'=  \begin{bmatrix}-1&-3\\3&5\end{bmatrix}{\bf y}$
  \end{problem}

 \begin{problem}\label{exer:10.5.39}  
 Plot trajectories of the given system.
 
 ${\bf y}'=  \begin{bmatrix}-5&3\\-3&1\end{bmatrix}{\bf y}$
 \end{problem}


 \begin{problem}\label{exer:10.5.40} 
 Plot trajectories of the given system.
 
 ${\bf y}'=  \begin{bmatrix}-2&-3\\3&4\end{bmatrix}{\bf y}$
  \end{problem}

 \begin{problem}\label{exer:10.5.41} 
 Plot trajectories of the given system.
 
 ${\bf y}'=  \begin{bmatrix}-4&-3\\3&2\end{bmatrix}{\bf y}$
  \end{problem}



 \begin{problem}\label{exer:10.5.42} 
 Plot trajectories of the given system.
 
 ${\bf y}'=  \begin{bmatrix}0&-1\\1&-2\end{bmatrix}{\bf y}$
  \end{problem}

 \begin{problem}\label{exer:10.5.43} 
 Plot trajectories of the given system.
 
 ${\bf y}'=  \begin{bmatrix}0&1\\-1&2\end{bmatrix}{\bf y}$
 \end{problem}


 \begin{problem}\label{exer:10.5.44} 
 Plot trajectories of the given system.
 
 ${\bf y}'=  \begin{bmatrix}-2&1\\-1&0\end{bmatrix}{\bf y}$
  \end{problem}

 \begin{problem}\label{exer:10.5.45}
 Plot trajectories of the given system.
 
 ${\bf y}'=  \begin{bmatrix}0&-4\\1&-4\end{bmatrix}{\bf y}$
 \end{problem}





\end{document}